\documentclass[11pt]{article}

    \usepackage[breakable]{tcolorbox}
    \usepackage{parskip} % Stop auto-indenting (to mimic markdown behaviour)
    

    % Basic figure setup, for now with no caption control since it's done
    % automatically by Pandoc (which extracts ![](path) syntax from Markdown).
    \usepackage{graphicx}
    % Keep aspect ratio if custom image width or height is specified
    \setkeys{Gin}{keepaspectratio}
    % Maintain compatibility with old templates. Remove in nbconvert 6.0
    \let\Oldincludegraphics\includegraphics
    % Ensure that by default, figures have no caption (until we provide a
    % proper Figure object with a Caption API and a way to capture that
    % in the conversion process - todo).
    \usepackage{caption}
    \DeclareCaptionFormat{nocaption}{}
    \captionsetup{format=nocaption,aboveskip=0pt,belowskip=0pt}

    \usepackage{float}
    \floatplacement{figure}{H} % forces figures to be placed at the correct location
    \usepackage{xcolor} % Allow colors to be defined
    \usepackage{enumerate} % Needed for markdown enumerations to work
    \usepackage{geometry} % Used to adjust the document margins
    \usepackage{amsmath} % Equations
    \usepackage{amssymb} % Equations
    \usepackage{textcomp} % defines textquotesingle
    % Hack from http://tex.stackexchange.com/a/47451/13684:
    \AtBeginDocument{%
        \def\PYZsq{\textquotesingle}% Upright quotes in Pygmentized code
    }
    \usepackage{upquote} % Upright quotes for verbatim code
    \usepackage{eurosym} % defines \euro

    \usepackage{iftex}
    \ifPDFTeX
        \usepackage[T1]{fontenc}
        \IfFileExists{alphabeta.sty}{
              \usepackage{alphabeta}
          }{
              \usepackage[mathletters]{ucs}
              \usepackage[utf8x]{inputenc}
          }
    \else
        \usepackage{fontspec}
        \usepackage{unicode-math}
    \fi

    \usepackage{fancyvrb} % verbatim replacement that allows latex
    \usepackage{grffile} % extends the file name processing of package graphics
                         % to support a larger range
    \makeatletter % fix for old versions of grffile with XeLaTeX
    \@ifpackagelater{grffile}{2019/11/01}
    {
      % Do nothing on new versions
    }
    {
      \def\Gread@@xetex#1{%
        \IfFileExists{"\Gin@base".bb}%
        {\Gread@eps{\Gin@base.bb}}%
        {\Gread@@xetex@aux#1}%
      }
    }
    \makeatother
    \usepackage[Export]{adjustbox} % Used to constrain images to a maximum size
    \adjustboxset{max size={0.9\linewidth}{0.9\paperheight}}

    % The hyperref package gives us a pdf with properly built
    % internal navigation ('pdf bookmarks' for the table of contents,
    % internal cross-reference links, web links for URLs, etc.)
    \usepackage{hyperref}
    % The default LaTeX title has an obnoxious amount of whitespace. By default,
    % titling removes some of it. It also provides customization options.
    \usepackage{titling}
    \usepackage{longtable} % longtable support required by pandoc >1.10
    \usepackage{booktabs}  % table support for pandoc > 1.12.2
    \usepackage{array}     % table support for pandoc >= 2.11.3
    \usepackage{calc}      % table minipage width calculation for pandoc >= 2.11.1
    \usepackage[inline]{enumitem} % IRkernel/repr support (it uses the enumerate* environment)
    \usepackage[normalem]{ulem} % ulem is needed to support strikethroughs (\sout)
                                % normalem makes italics be italics, not underlines
    \usepackage{soul}      % strikethrough (\st) support for pandoc >= 3.0.0
    \usepackage{mathrsfs}
    

    
    % Colors for the hyperref package
    \definecolor{urlcolor}{rgb}{0,.145,.698}
    \definecolor{linkcolor}{rgb}{.71,0.21,0.01}
    \definecolor{citecolor}{rgb}{.12,.54,.11}

    % ANSI colors
    \definecolor{ansi-black}{HTML}{3E424D}
    \definecolor{ansi-black-intense}{HTML}{282C36}
    \definecolor{ansi-red}{HTML}{E75C58}
    \definecolor{ansi-red-intense}{HTML}{B22B31}
    \definecolor{ansi-green}{HTML}{00A250}
    \definecolor{ansi-green-intense}{HTML}{007427}
    \definecolor{ansi-yellow}{HTML}{DDB62B}
    \definecolor{ansi-yellow-intense}{HTML}{B27D12}
    \definecolor{ansi-blue}{HTML}{208FFB}
    \definecolor{ansi-blue-intense}{HTML}{0065CA}
    \definecolor{ansi-magenta}{HTML}{D160C4}
    \definecolor{ansi-magenta-intense}{HTML}{A03196}
    \definecolor{ansi-cyan}{HTML}{60C6C8}
    \definecolor{ansi-cyan-intense}{HTML}{258F8F}
    \definecolor{ansi-white}{HTML}{C5C1B4}
    \definecolor{ansi-white-intense}{HTML}{A1A6B2}
    \definecolor{ansi-default-inverse-fg}{HTML}{FFFFFF}
    \definecolor{ansi-default-inverse-bg}{HTML}{000000}

    % common color for the border for error outputs.
    \definecolor{outerrorbackground}{HTML}{FFDFDF}

    % commands and environments needed by pandoc snippets
    % extracted from the output of `pandoc -s`
    \providecommand{\tightlist}{%
      \setlength{\itemsep}{0pt}\setlength{\parskip}{0pt}}
    \DefineVerbatimEnvironment{Highlighting}{Verbatim}{commandchars=\\\{\}}
    % Add ',fontsize=\small' for more characters per line
    \newenvironment{Shaded}{}{}
    \newcommand{\KeywordTok}[1]{\textcolor[rgb]{0.00,0.44,0.13}{\textbf{{#1}}}}
    \newcommand{\DataTypeTok}[1]{\textcolor[rgb]{0.56,0.13,0.00}{{#1}}}
    \newcommand{\DecValTok}[1]{\textcolor[rgb]{0.25,0.63,0.44}{{#1}}}
    \newcommand{\BaseNTok}[1]{\textcolor[rgb]{0.25,0.63,0.44}{{#1}}}
    \newcommand{\FloatTok}[1]{\textcolor[rgb]{0.25,0.63,0.44}{{#1}}}
    \newcommand{\CharTok}[1]{\textcolor[rgb]{0.25,0.44,0.63}{{#1}}}
    \newcommand{\StringTok}[1]{\textcolor[rgb]{0.25,0.44,0.63}{{#1}}}
    \newcommand{\CommentTok}[1]{\textcolor[rgb]{0.38,0.63,0.69}{\textit{{#1}}}}
    \newcommand{\OtherTok}[1]{\textcolor[rgb]{0.00,0.44,0.13}{{#1}}}
    \newcommand{\AlertTok}[1]{\textcolor[rgb]{1.00,0.00,0.00}{\textbf{{#1}}}}
    \newcommand{\FunctionTok}[1]{\textcolor[rgb]{0.02,0.16,0.49}{{#1}}}
    \newcommand{\RegionMarkerTok}[1]{{#1}}
    \newcommand{\ErrorTok}[1]{\textcolor[rgb]{1.00,0.00,0.00}{\textbf{{#1}}}}
    \newcommand{\NormalTok}[1]{{#1}}

    % Additional commands for more recent versions of Pandoc
    \newcommand{\ConstantTok}[1]{\textcolor[rgb]{0.53,0.00,0.00}{{#1}}}
    \newcommand{\SpecialCharTok}[1]{\textcolor[rgb]{0.25,0.44,0.63}{{#1}}}
    \newcommand{\VerbatimStringTok}[1]{\textcolor[rgb]{0.25,0.44,0.63}{{#1}}}
    \newcommand{\SpecialStringTok}[1]{\textcolor[rgb]{0.73,0.40,0.53}{{#1}}}
    \newcommand{\ImportTok}[1]{{#1}}
    \newcommand{\DocumentationTok}[1]{\textcolor[rgb]{0.73,0.13,0.13}{\textit{{#1}}}}
    \newcommand{\AnnotationTok}[1]{\textcolor[rgb]{0.38,0.63,0.69}{\textbf{\textit{{#1}}}}}
    \newcommand{\CommentVarTok}[1]{\textcolor[rgb]{0.38,0.63,0.69}{\textbf{\textit{{#1}}}}}
    \newcommand{\VariableTok}[1]{\textcolor[rgb]{0.10,0.09,0.49}{{#1}}}
    \newcommand{\ControlFlowTok}[1]{\textcolor[rgb]{0.00,0.44,0.13}{\textbf{{#1}}}}
    \newcommand{\OperatorTok}[1]{\textcolor[rgb]{0.40,0.40,0.40}{{#1}}}
    \newcommand{\BuiltInTok}[1]{{#1}}
    \newcommand{\ExtensionTok}[1]{{#1}}
    \newcommand{\PreprocessorTok}[1]{\textcolor[rgb]{0.74,0.48,0.00}{{#1}}}
    \newcommand{\AttributeTok}[1]{\textcolor[rgb]{0.49,0.56,0.16}{{#1}}}
    \newcommand{\InformationTok}[1]{\textcolor[rgb]{0.38,0.63,0.69}{\textbf{\textit{{#1}}}}}
    \newcommand{\WarningTok}[1]{\textcolor[rgb]{0.38,0.63,0.69}{\textbf{\textit{{#1}}}}}


    % Define a nice break command that doesn't care if a line doesn't already
    % exist.
    \def\br{\hspace*{\fill} \\* }
    % Math Jax compatibility definitions
    \def\gt{>}
    \def\lt{<}
    \let\Oldtex\TeX
    \let\Oldlatex\LaTeX
    \renewcommand{\TeX}{\textrm{\Oldtex}}
    \renewcommand{\LaTeX}{\textrm{\Oldlatex}}
    % Document parameters
    % Document title
    \title{Cauchy\_theorem}
    
    
    
    
    
    
    
% Pygments definitions
\makeatletter
\def\PY@reset{\let\PY@it=\relax \let\PY@bf=\relax%
    \let\PY@ul=\relax \let\PY@tc=\relax%
    \let\PY@bc=\relax \let\PY@ff=\relax}
\def\PY@tok#1{\csname PY@tok@#1\endcsname}
\def\PY@toks#1+{\ifx\relax#1\empty\else%
    \PY@tok{#1}\expandafter\PY@toks\fi}
\def\PY@do#1{\PY@bc{\PY@tc{\PY@ul{%
    \PY@it{\PY@bf{\PY@ff{#1}}}}}}}
\def\PY#1#2{\PY@reset\PY@toks#1+\relax+\PY@do{#2}}

\@namedef{PY@tok@w}{\def\PY@tc##1{\textcolor[rgb]{0.73,0.73,0.73}{##1}}}
\@namedef{PY@tok@c}{\let\PY@it=\textit\def\PY@tc##1{\textcolor[rgb]{0.24,0.48,0.48}{##1}}}
\@namedef{PY@tok@cp}{\def\PY@tc##1{\textcolor[rgb]{0.61,0.40,0.00}{##1}}}
\@namedef{PY@tok@k}{\let\PY@bf=\textbf\def\PY@tc##1{\textcolor[rgb]{0.00,0.50,0.00}{##1}}}
\@namedef{PY@tok@kp}{\def\PY@tc##1{\textcolor[rgb]{0.00,0.50,0.00}{##1}}}
\@namedef{PY@tok@kt}{\def\PY@tc##1{\textcolor[rgb]{0.69,0.00,0.25}{##1}}}
\@namedef{PY@tok@o}{\def\PY@tc##1{\textcolor[rgb]{0.40,0.40,0.40}{##1}}}
\@namedef{PY@tok@ow}{\let\PY@bf=\textbf\def\PY@tc##1{\textcolor[rgb]{0.67,0.13,1.00}{##1}}}
\@namedef{PY@tok@nb}{\def\PY@tc##1{\textcolor[rgb]{0.00,0.50,0.00}{##1}}}
\@namedef{PY@tok@nf}{\def\PY@tc##1{\textcolor[rgb]{0.00,0.00,1.00}{##1}}}
\@namedef{PY@tok@nc}{\let\PY@bf=\textbf\def\PY@tc##1{\textcolor[rgb]{0.00,0.00,1.00}{##1}}}
\@namedef{PY@tok@nn}{\let\PY@bf=\textbf\def\PY@tc##1{\textcolor[rgb]{0.00,0.00,1.00}{##1}}}
\@namedef{PY@tok@ne}{\let\PY@bf=\textbf\def\PY@tc##1{\textcolor[rgb]{0.80,0.25,0.22}{##1}}}
\@namedef{PY@tok@nv}{\def\PY@tc##1{\textcolor[rgb]{0.10,0.09,0.49}{##1}}}
\@namedef{PY@tok@no}{\def\PY@tc##1{\textcolor[rgb]{0.53,0.00,0.00}{##1}}}
\@namedef{PY@tok@nl}{\def\PY@tc##1{\textcolor[rgb]{0.46,0.46,0.00}{##1}}}
\@namedef{PY@tok@ni}{\let\PY@bf=\textbf\def\PY@tc##1{\textcolor[rgb]{0.44,0.44,0.44}{##1}}}
\@namedef{PY@tok@na}{\def\PY@tc##1{\textcolor[rgb]{0.41,0.47,0.13}{##1}}}
\@namedef{PY@tok@nt}{\let\PY@bf=\textbf\def\PY@tc##1{\textcolor[rgb]{0.00,0.50,0.00}{##1}}}
\@namedef{PY@tok@nd}{\def\PY@tc##1{\textcolor[rgb]{0.67,0.13,1.00}{##1}}}
\@namedef{PY@tok@s}{\def\PY@tc##1{\textcolor[rgb]{0.73,0.13,0.13}{##1}}}
\@namedef{PY@tok@sd}{\let\PY@it=\textit\def\PY@tc##1{\textcolor[rgb]{0.73,0.13,0.13}{##1}}}
\@namedef{PY@tok@si}{\let\PY@bf=\textbf\def\PY@tc##1{\textcolor[rgb]{0.64,0.35,0.47}{##1}}}
\@namedef{PY@tok@se}{\let\PY@bf=\textbf\def\PY@tc##1{\textcolor[rgb]{0.67,0.36,0.12}{##1}}}
\@namedef{PY@tok@sr}{\def\PY@tc##1{\textcolor[rgb]{0.64,0.35,0.47}{##1}}}
\@namedef{PY@tok@ss}{\def\PY@tc##1{\textcolor[rgb]{0.10,0.09,0.49}{##1}}}
\@namedef{PY@tok@sx}{\def\PY@tc##1{\textcolor[rgb]{0.00,0.50,0.00}{##1}}}
\@namedef{PY@tok@m}{\def\PY@tc##1{\textcolor[rgb]{0.40,0.40,0.40}{##1}}}
\@namedef{PY@tok@gh}{\let\PY@bf=\textbf\def\PY@tc##1{\textcolor[rgb]{0.00,0.00,0.50}{##1}}}
\@namedef{PY@tok@gu}{\let\PY@bf=\textbf\def\PY@tc##1{\textcolor[rgb]{0.50,0.00,0.50}{##1}}}
\@namedef{PY@tok@gd}{\def\PY@tc##1{\textcolor[rgb]{0.63,0.00,0.00}{##1}}}
\@namedef{PY@tok@gi}{\def\PY@tc##1{\textcolor[rgb]{0.00,0.52,0.00}{##1}}}
\@namedef{PY@tok@gr}{\def\PY@tc##1{\textcolor[rgb]{0.89,0.00,0.00}{##1}}}
\@namedef{PY@tok@ge}{\let\PY@it=\textit}
\@namedef{PY@tok@gs}{\let\PY@bf=\textbf}
\@namedef{PY@tok@ges}{\let\PY@bf=\textbf\let\PY@it=\textit}
\@namedef{PY@tok@gp}{\let\PY@bf=\textbf\def\PY@tc##1{\textcolor[rgb]{0.00,0.00,0.50}{##1}}}
\@namedef{PY@tok@go}{\def\PY@tc##1{\textcolor[rgb]{0.44,0.44,0.44}{##1}}}
\@namedef{PY@tok@gt}{\def\PY@tc##1{\textcolor[rgb]{0.00,0.27,0.87}{##1}}}
\@namedef{PY@tok@err}{\def\PY@bc##1{{\setlength{\fboxsep}{\string -\fboxrule}\fcolorbox[rgb]{1.00,0.00,0.00}{1,1,1}{\strut ##1}}}}
\@namedef{PY@tok@kc}{\let\PY@bf=\textbf\def\PY@tc##1{\textcolor[rgb]{0.00,0.50,0.00}{##1}}}
\@namedef{PY@tok@kd}{\let\PY@bf=\textbf\def\PY@tc##1{\textcolor[rgb]{0.00,0.50,0.00}{##1}}}
\@namedef{PY@tok@kn}{\let\PY@bf=\textbf\def\PY@tc##1{\textcolor[rgb]{0.00,0.50,0.00}{##1}}}
\@namedef{PY@tok@kr}{\let\PY@bf=\textbf\def\PY@tc##1{\textcolor[rgb]{0.00,0.50,0.00}{##1}}}
\@namedef{PY@tok@bp}{\def\PY@tc##1{\textcolor[rgb]{0.00,0.50,0.00}{##1}}}
\@namedef{PY@tok@fm}{\def\PY@tc##1{\textcolor[rgb]{0.00,0.00,1.00}{##1}}}
\@namedef{PY@tok@vc}{\def\PY@tc##1{\textcolor[rgb]{0.10,0.09,0.49}{##1}}}
\@namedef{PY@tok@vg}{\def\PY@tc##1{\textcolor[rgb]{0.10,0.09,0.49}{##1}}}
\@namedef{PY@tok@vi}{\def\PY@tc##1{\textcolor[rgb]{0.10,0.09,0.49}{##1}}}
\@namedef{PY@tok@vm}{\def\PY@tc##1{\textcolor[rgb]{0.10,0.09,0.49}{##1}}}
\@namedef{PY@tok@sa}{\def\PY@tc##1{\textcolor[rgb]{0.73,0.13,0.13}{##1}}}
\@namedef{PY@tok@sb}{\def\PY@tc##1{\textcolor[rgb]{0.73,0.13,0.13}{##1}}}
\@namedef{PY@tok@sc}{\def\PY@tc##1{\textcolor[rgb]{0.73,0.13,0.13}{##1}}}
\@namedef{PY@tok@dl}{\def\PY@tc##1{\textcolor[rgb]{0.73,0.13,0.13}{##1}}}
\@namedef{PY@tok@s2}{\def\PY@tc##1{\textcolor[rgb]{0.73,0.13,0.13}{##1}}}
\@namedef{PY@tok@sh}{\def\PY@tc##1{\textcolor[rgb]{0.73,0.13,0.13}{##1}}}
\@namedef{PY@tok@s1}{\def\PY@tc##1{\textcolor[rgb]{0.73,0.13,0.13}{##1}}}
\@namedef{PY@tok@mb}{\def\PY@tc##1{\textcolor[rgb]{0.40,0.40,0.40}{##1}}}
\@namedef{PY@tok@mf}{\def\PY@tc##1{\textcolor[rgb]{0.40,0.40,0.40}{##1}}}
\@namedef{PY@tok@mh}{\def\PY@tc##1{\textcolor[rgb]{0.40,0.40,0.40}{##1}}}
\@namedef{PY@tok@mi}{\def\PY@tc##1{\textcolor[rgb]{0.40,0.40,0.40}{##1}}}
\@namedef{PY@tok@il}{\def\PY@tc##1{\textcolor[rgb]{0.40,0.40,0.40}{##1}}}
\@namedef{PY@tok@mo}{\def\PY@tc##1{\textcolor[rgb]{0.40,0.40,0.40}{##1}}}
\@namedef{PY@tok@ch}{\let\PY@it=\textit\def\PY@tc##1{\textcolor[rgb]{0.24,0.48,0.48}{##1}}}
\@namedef{PY@tok@cm}{\let\PY@it=\textit\def\PY@tc##1{\textcolor[rgb]{0.24,0.48,0.48}{##1}}}
\@namedef{PY@tok@cpf}{\let\PY@it=\textit\def\PY@tc##1{\textcolor[rgb]{0.24,0.48,0.48}{##1}}}
\@namedef{PY@tok@c1}{\let\PY@it=\textit\def\PY@tc##1{\textcolor[rgb]{0.24,0.48,0.48}{##1}}}
\@namedef{PY@tok@cs}{\let\PY@it=\textit\def\PY@tc##1{\textcolor[rgb]{0.24,0.48,0.48}{##1}}}

\def\PYZbs{\char`\\}
\def\PYZus{\char`\_}
\def\PYZob{\char`\{}
\def\PYZcb{\char`\}}
\def\PYZca{\char`\^}
\def\PYZam{\char`\&}
\def\PYZlt{\char`\<}
\def\PYZgt{\char`\>}
\def\PYZsh{\char`\#}
\def\PYZpc{\char`\%}
\def\PYZdl{\char`\$}
\def\PYZhy{\char`\-}
\def\PYZsq{\char`\'}
\def\PYZdq{\char`\"}
\def\PYZti{\char`\~}
% for compatibility with earlier versions
\def\PYZat{@}
\def\PYZlb{[}
\def\PYZrb{]}
\makeatother


    % For linebreaks inside Verbatim environment from package fancyvrb.
    \makeatletter
        \newbox\Wrappedcontinuationbox
        \newbox\Wrappedvisiblespacebox
        \newcommand*\Wrappedvisiblespace {\textcolor{red}{\textvisiblespace}}
        \newcommand*\Wrappedcontinuationsymbol {\textcolor{red}{\llap{\tiny$\m@th\hookrightarrow$}}}
        \newcommand*\Wrappedcontinuationindent {3ex }
        \newcommand*\Wrappedafterbreak {\kern\Wrappedcontinuationindent\copy\Wrappedcontinuationbox}
        % Take advantage of the already applied Pygments mark-up to insert
        % potential linebreaks for TeX processing.
        %        {, <, #, %, $, ' and ": go to next line.
        %        _, }, ^, &, >, - and ~: stay at end of broken line.
        % Use of \textquotesingle for straight quote.
        \newcommand*\Wrappedbreaksatspecials {%
            \def\PYGZus{\discretionary{\char`\_}{\Wrappedafterbreak}{\char`\_}}%
            \def\PYGZob{\discretionary{}{\Wrappedafterbreak\char`\{}{\char`\{}}%
            \def\PYGZcb{\discretionary{\char`\}}{\Wrappedafterbreak}{\char`\}}}%
            \def\PYGZca{\discretionary{\char`\^}{\Wrappedafterbreak}{\char`\^}}%
            \def\PYGZam{\discretionary{\char`\&}{\Wrappedafterbreak}{\char`\&}}%
            \def\PYGZlt{\discretionary{}{\Wrappedafterbreak\char`\<}{\char`\<}}%
            \def\PYGZgt{\discretionary{\char`\>}{\Wrappedafterbreak}{\char`\>}}%
            \def\PYGZsh{\discretionary{}{\Wrappedafterbreak\char`\#}{\char`\#}}%
            \def\PYGZpc{\discretionary{}{\Wrappedafterbreak\char`\%}{\char`\%}}%
            \def\PYGZdl{\discretionary{}{\Wrappedafterbreak\char`\$}{\char`\$}}%
            \def\PYGZhy{\discretionary{\char`\-}{\Wrappedafterbreak}{\char`\-}}%
            \def\PYGZsq{\discretionary{}{\Wrappedafterbreak\textquotesingle}{\textquotesingle}}%
            \def\PYGZdq{\discretionary{}{\Wrappedafterbreak\char`\"}{\char`\"}}%
            \def\PYGZti{\discretionary{\char`\~}{\Wrappedafterbreak}{\char`\~}}%
        }
        % Some characters . , ; ? ! / are not pygmentized.
        % This macro makes them "active" and they will insert potential linebreaks
        \newcommand*\Wrappedbreaksatpunct {%
            \lccode`\~`\.\lowercase{\def~}{\discretionary{\hbox{\char`\.}}{\Wrappedafterbreak}{\hbox{\char`\.}}}%
            \lccode`\~`\,\lowercase{\def~}{\discretionary{\hbox{\char`\,}}{\Wrappedafterbreak}{\hbox{\char`\,}}}%
            \lccode`\~`\;\lowercase{\def~}{\discretionary{\hbox{\char`\;}}{\Wrappedafterbreak}{\hbox{\char`\;}}}%
            \lccode`\~`\:\lowercase{\def~}{\discretionary{\hbox{\char`\:}}{\Wrappedafterbreak}{\hbox{\char`\:}}}%
            \lccode`\~`\?\lowercase{\def~}{\discretionary{\hbox{\char`\?}}{\Wrappedafterbreak}{\hbox{\char`\?}}}%
            \lccode`\~`\!\lowercase{\def~}{\discretionary{\hbox{\char`\!}}{\Wrappedafterbreak}{\hbox{\char`\!}}}%
            \lccode`\~`\/\lowercase{\def~}{\discretionary{\hbox{\char`\/}}{\Wrappedafterbreak}{\hbox{\char`\/}}}%
            \catcode`\.\active
            \catcode`\,\active
            \catcode`\;\active
            \catcode`\:\active
            \catcode`\?\active
            \catcode`\!\active
            \catcode`\/\active
            \lccode`\~`\~
        }
    \makeatother

    \let\OriginalVerbatim=\Verbatim
    \makeatletter
    \renewcommand{\Verbatim}[1][1]{%
        %\parskip\z@skip
        \sbox\Wrappedcontinuationbox {\Wrappedcontinuationsymbol}%
        \sbox\Wrappedvisiblespacebox {\FV@SetupFont\Wrappedvisiblespace}%
        \def\FancyVerbFormatLine ##1{\hsize\linewidth
            \vtop{\raggedright\hyphenpenalty\z@\exhyphenpenalty\z@
                \doublehyphendemerits\z@\finalhyphendemerits\z@
                \strut ##1\strut}%
        }%
        % If the linebreak is at a space, the latter will be displayed as visible
        % space at end of first line, and a continuation symbol starts next line.
        % Stretch/shrink are however usually zero for typewriter font.
        \def\FV@Space {%
            \nobreak\hskip\z@ plus\fontdimen3\font minus\fontdimen4\font
            \discretionary{\copy\Wrappedvisiblespacebox}{\Wrappedafterbreak}
            {\kern\fontdimen2\font}%
        }%

        % Allow breaks at special characters using \PYG... macros.
        \Wrappedbreaksatspecials
        % Breaks at punctuation characters . , ; ? ! and / need catcode=\active
        \OriginalVerbatim[#1,codes*=\Wrappedbreaksatpunct]%
    }
    \makeatother

    % Exact colors from NB
    \definecolor{incolor}{HTML}{303F9F}
    \definecolor{outcolor}{HTML}{D84315}
    \definecolor{cellborder}{HTML}{CFCFCF}
    \definecolor{cellbackground}{HTML}{F7F7F7}

    % prompt
    \makeatletter
    \newcommand{\boxspacing}{\kern\kvtcb@left@rule\kern\kvtcb@boxsep}
    \makeatother
    \newcommand{\prompt}[4]{
        {\ttfamily\llap{{\color{#2}[#3]:\hspace{3pt}#4}}\vspace{-\baselineskip}}
    }
    

    
    % Prevent overflowing lines due to hard-to-break entities
    \sloppy
    % Setup hyperref package
    \hypersetup{
      breaklinks=true,  % so long urls are correctly broken across lines
      colorlinks=true,
      urlcolor=urlcolor,
      linkcolor=linkcolor,
      citecolor=citecolor,
      }
    % Slightly bigger margins than the latex defaults
    
    \geometry{verbose,tmargin=1in,bmargin=1in,lmargin=1in,rmargin=1in}
    
    

\begin{document}
    
    \maketitle
    
    

    
    \begin{tcolorbox}[breakable, size=fbox, boxrule=1pt, pad at break*=1mm,colback=cellbackground, colframe=cellborder]
\prompt{In}{incolor}{1}{\boxspacing}
\begin{Verbatim}[commandchars=\\\{\}]
\PY{c}{\PYZsh{}\PYZsh{}\PYZsh{} A Pluto.jl notebook \PYZsh{}\PYZsh{}\PYZsh{}}
\PY{c}{\PYZsh{} v0.20.3}
\PY{c}{\PYZsh{}\PYZgt{} [frontmatter]}
\PY{c}{\PYZsh{}\PYZgt{} title = \PYZdq{}Cauchy Theorem\PYZdq{}}
\PY{c}{\PYZsh{}\PYZgt{} date = \PYZdq{}2025\PYZhy{}01\PYZhy{}09\PYZdq{}}
\PY{c}{\PYZsh{}\PYZgt{} tags = [\PYZdq{}Complex\PYZdq{}, \PYZdq{}analysis\PYZdq{}, \PYZdq{}cauchy\PYZdq{}, \PYZdq{}julia\PYZdq{}]}
\PY{c}{\PYZsh{}\PYZgt{} description = \PYZdq{}Complex Analysic notes\PYZdq{}}
\PY{c}{\PYZsh{}\PYZgt{} }
\PY{c}{\PYZsh{}\PYZgt{}     [[frontmatter.author]]}
\PY{c}{\PYZsh{}\PYZgt{}     name = \PYZdq{}Nima Poshitban\PYZdq{}}
\PY{k}{using}\PY{+w}{ }\PY{n}{Markdown}
\PY{k}{using}\PY{+w}{ }\PY{n}{InteractiveUtils}
\end{Verbatim}
\end{tcolorbox}

    \begin{tcolorbox}[breakable, size=fbox, boxrule=1pt, pad at break*=1mm,colback=cellbackground, colframe=cellborder]
\prompt{In}{incolor}{2}{\boxspacing}
\begin{Verbatim}[commandchars=\\\{\}]
\PY{k}{begin}
\PY{+w}{	}\PY{k}{using}\PY{+w}{ }\PY{n}{PlutoExtras}
\PY{+w}{    }\PY{k}{import}\PY{+w}{ }\PY{n}{Pkg}
\PY{+w}{    }\PY{c}{\PYZsh{} careful: this is \PYZus{}not\PYZus{} a reproducible environment}
\PY{+w}{    }\PY{c}{\PYZsh{} activate the global environment}
\PY{+w}{    }\PY{n}{Pkg}\PY{o}{.}\PY{n}{activate}\PY{p}{(}\PY{p}{)}
\PY{+w}{    }\PY{k}{import}\PY{+w}{ }\PY{n}{GLMakie}
\PY{+w}{	}\PY{k}{using}\PY{+w}{ }\PY{n}{LaTeXStrings}
\PY{k}{end}
\end{Verbatim}
\end{tcolorbox}

    \begin{Verbatim}[commandchars=\\\{\}]
\textcolor{ansi-green-intense}{\textbf{  Activating}} project at
`C:\textbackslash{}Users\textbackslash{}nima\textbackslash{}.julia\textbackslash{}environments\textbackslash{}v1.11`
    \end{Verbatim}

    \begin{tcolorbox}[breakable, size=fbox, boxrule=1pt, pad at break*=1mm,colback=cellbackground, colframe=cellborder]
\prompt{In}{incolor}{3}{\boxspacing}
\begin{Verbatim}[commandchars=\\\{\}]
\PY{n}{initialize\PYZus{}eqref}\PY{p}{(}\PY{p}{)}
\end{Verbatim}
\end{tcolorbox}

            \begin{tcolorbox}[breakable, size=fbox, boxrule=.5pt, pad at break*=1mm, opacityfill=0]
\prompt{Out}{outcolor}{3}{\boxspacing}
\begin{Verbatim}[commandchars=\\\{\}]
<link rel="stylesheet"
href="https://cdn.jsdelivr.net/npm/katex@0.16.8/dist/katex.min.css" integrity="s
ha384-GvrOXuhMATgEsSwCs4smul74iXGOixntILdUW9XmUC6+HX0sLNAK3q71HotJqlAn"
crossorigin="anonymous">

<style>
a.eq\_href \{
        text-decoration: none;
\}
</style>

<script src="https://cdn.jsdelivr.net/npm/katex@0.16.8/dist/katex.min.js" integr
ity="sha384-cpW21h6RZv/phavutF+AuVYrr+dA8xD9zs6FwLpaCct6O9ctzYFfFr4dgmgccOTx"
crossorigin="anonymous"></script>

<script id="katex-eqnum-script">


const a\_vec = [] // This will hold the list of a tags with custom click, used
for cleaning listeners up upon invalidation

const eqrefClick = (e) => \{
        e.preventDefault() // This prevent normal scrolling to link
        const a = e.target
        const eq\_id = a.getAttribute('eq\_id')
        const eq = document.getElementById(eq\_id)
        history.pushState(\{\},'') // This is to allow going back to the previous
position in the page after scroll with History Back
        eq.scrollIntoView(\{
                behavior: 'smooth',
                block: 'center',
        \})
\}

// We make a function to compute the vertical offset (from the top) of an object
to the
// closest parent containing the katex-html class. This is used to find the
equation number
// that is closest to the label
const findOffsetTop = obj => \{
        let offset = 0
        let keepGoing = true
        while (keepGoing) \{
                offset += obj.offsetTop
                // Check if the current offsetParent is the containing katex-
html
                if (obj.offsetParent.classList.contains('katex-html')) \{
                        keepGoing = false
                \} else \{
                        obj = obj.offsetParent
                \}
        \}
        return offset
\}


// The katex equation numbers are wrapped in spans containing the class 'eqn-
num'. WHen you
// assign a label, another class ('enclosing') is assigned to some parts of the
rendered
// html containing the equation line. This means that equation containing labels
will have
// both 'eqn-num' and 'enclosing'. The new approach is to go through all the
katex math
// equations one by one and analyze how many numbered lines they contain by
counting the
// 'eqn-num' instances.
const updateCallback = () => \{
a\_vec.splice(0,a\_vec.length) // Reset the array
const katex\_blocks = document.querySelectorAll('.katex-html') // This selects
all the environments we created with texeq
let i = 0;
for (let blk of katex\_blocks) \{
        // Find the number of numbered equation in each sub-block
        let numeqs = blk.querySelectorAll('.eqn-num')
        let eqlen = numeqs.length
        if (eqlen == 0) \{
                continue // There is nothing to do here since no equation is
numbered
        \}
        let labeleqs = blk.querySelectorAll('.enclosing')
        if (labeleqs.length == 0) \{
                // There is no label, so we just have to increase the counter
                i += eqlen
                continue
        \}
        // Find the offset from the katex-html parent of each equation number,
the assumption
        // here is that the span containing the label tag has the same (or
almost the same) offset as the related equation number
        let eqoffsets = Array.from(numeqs,findOffsetTop)


        for (let item of labeleqs) \{
                const labelOffset = findOffsetTop(item)
                let prevDiff = -Infinity
                let currentOffset = eqoffsets.shift()
                let currentDiff = currentOffset - labelOffset
                i += 1
                while (eqoffsets.length > 0 \&\& currentDiff < 0) \{ // if
currentOffset >= 0, it means that the current equ-num is lower than the label
(or at the same height)
                        prevDiff = currentDiff
                        currentOffset = eqoffsets.shift()
                        currentDiff = currentOffset - labelOffset
                        i += 1
                \}
                // Now we have to check whether the previous number with offset
< 0 or the first with offset > 0 is the closest to the label offset
                if (Math.abs(currentDiff) > Math.abs(prevDiff)) \{
                        // The previous entry was closer, so we reduce i by one
and put back the last shifted element in the offset array
                        i -= 1
                        eqoffsets.unshift(currentOffset)
                \}
                // We now update all the links that refer to this label
                const id = item.id
                const a\_vals = document.querySelectorAll(`[eq\_id=\$\{id\}]`)
                a\_vals !== null \&\& a\_vals.forEach(a => \{
                        a\_vec.push(a) // Add this to the vector
                        a.innerText = `(\$\{i\})`
                        a.addEventListener('click',eqrefClick)
                \})
        \}
\}
\}

const notebook = document.querySelector("pluto-notebook")

// We have a mutationobserver for each cell:
const observers = \{
        current: [],
\}

const createCellObservers = () => \{
        observers.current.forEach((o) => o.disconnect())
        observers.current = Array.from(notebook.querySelectorAll("pluto-
cell")).map(el => \{
                const o = new MutationObserver(updateCallback)
                o.observe(el, \{attributeFilter: ["class"]\})
                return o
        \})
\}
createCellObservers()

// And one for the notebook's child list, which updates our cell observers:
const notebookObserver = new MutationObserver(() => \{
        updateCallback()
        createCellObservers()
\})
notebookObserver.observe(notebook, \{childList: true\})

invalidation.then(() => \{
        notebookObserver.disconnect()
        observers.current.forEach((o) => o.disconnect())
        a\_vec.forEach(a => a.removeEventListener('click',eqrefClick))
\})
</script>

\end{Verbatim}
\end{tcolorbox}
        
    \section{Cauchy Theorem}\label{cauchy-theorem}

    \subsection{Cauchy Integral Formulas}\label{cauchy-integral-formulas}

\textbf{C} is a toy contour

\textbf{Toy contour}: any closed curve where the notion of the interior
is obvious

\[\large f(\mathcal{z})\,=\,\frac{1}{2\pi i} \oint_{\mathcal{C}}{\dfrac{f(\zeta)}{\zeta\,-\mathcal{z}}d\varsigma}\]

    \section{\texorpdfstring{Cauchy integral formula for the \emph{nth}
derivative of
\(f(z)\)}{Cauchy integral formula for the nth derivative of f(z)}}\label{cauchy-integral-formula-for-the-nth-derivative-of-fz}

\(\large f^{(n)}(\mathcal{z})\,=\,\frac{n!}{2\pi i}\oint_{\mathcal{C}}{\dfrac{f(\zeta)}{(\zeta\,-\mathcal{z})^{n+1}}d\zeta}\)
for all \(z\,\) in the interior of \(C\)

    \subsection{Poisson Integral}\label{poisson-integral}

\(\large u(r,\theta)\,=\frac{1}{2\pi}\int_{0}^{2\pi}{P_{r}(\theta-\varphi)u(1,\varphi)d\varphi}\)

    \subsection{Cauchy inequalities}\label{cauchy-inequalities}

\(f\longrightarrow holomorphic\) in an open set \(\Omega\)

disk \(D\) centered at \(z_{0}\) and has a radius of \(R\)

then:

\(\left|f^{(n)}(z_{0}) \right| \, \le \, \dfrac{n!\left\|f\right\|_{\mathcal{C}}}{R^{n}}\)

where
\(\left\|f\right|_{\mathcal{C}}\,=\,\underset{z\in \mathcal{C}}{sup}|f(\mathcal{z})|\)
denotes the supremum of \(|f|\) on the boundary cirlce \(\mathcal{C}\)

    \subsection{Power Series Expansion}\label{power-series-expansion}

if \(f \longrightarrow holomorphic\) in an open set \(\Omega\)

and disk \(D\) centered at \(z_{0}\) where closure is contained in
\(\Omega\)

\(\implies f\quad\) has a power series expansion at \(z_{0}\)

\(f(\mathcal{z})\,=\,\sum_{n=0}^{\infty}{a_{n}(z-z_{0})^{n}}\) for all
\(\mathcal{z} \in D\)

and the \textbf{coefficients} are given by:

\(\large a_{n}\,=\,\dfrac{f^{(n)}(z_{0})}{n!}\)

\[\Large \implies f(z)\,=\,\sum_{n=0}^{\infty}{\left(\frac{1}{2\pi i}\oint_{\mathcal{C}}{\dfrac{f(\zeta)}{(\zeta-z_{0})^{n+1}}d\zeta}\right)\cdot(z-z_{0})^{n}}\]

    \textbf{Example of Cauchy Integral}

    \begin{tcolorbox}[breakable, size=fbox, boxrule=1pt, pad at break*=1mm,colback=cellbackground, colframe=cellborder]
\prompt{In}{incolor}{4}{\boxspacing}
\begin{Verbatim}[commandchars=\\\{\}]
\PY{n}{C}\PY{+w}{ }\PY{o}{=}\PY{+w}{ }\PY{n}{range}\PY{p}{(}\PY{l+m+mi}{0}\PY{p}{,}\PY{l+m+mi}{2}\PY{n+nb}{π}\PY{p}{,}\PY{l+m+mi}{360}\PY{p}{)}
\end{Verbatim}
\end{tcolorbox}

            \begin{tcolorbox}[breakable, size=fbox, boxrule=.5pt, pad at break*=1mm, opacityfill=0]
\prompt{Out}{outcolor}{4}{\boxspacing}
\begin{Verbatim}[commandchars=\\\{\}]
0.0:0.01750190893364787:6.283185307179586
\end{Verbatim}
\end{tcolorbox}
        
    \begin{tcolorbox}[breakable, size=fbox, boxrule=1pt, pad at break*=1mm,colback=cellbackground, colframe=cellborder]
\prompt{In}{incolor}{5}{\boxspacing}
\begin{Verbatim}[commandchars=\\\{\}]
\PY{n}{fvarsigma}\PY{+w}{ }\PY{o}{=}\PY{+w}{ }\PY{n}{t}\PY{+w}{ }\PY{o}{\PYZhy{}\PYZgt{}}\PY{+w}{ }\PY{n}{exp}\PY{p}{(}\PY{l+m+mi}{1}\PY{p}{)}\PY{o}{.\PYZca{}}\PY{p}{(}\PY{n+nb}{im}\PY{o}{.*}\PY{n}{t}\PY{p}{)}
\end{Verbatim}
\end{tcolorbox}

            \begin{tcolorbox}[breakable, size=fbox, boxrule=.5pt, pad at break*=1mm, opacityfill=0]
\prompt{Out}{outcolor}{5}{\boxspacing}
\begin{Verbatim}[commandchars=\\\{\}]
\#1 (generic function with 1 method)
\end{Verbatim}
\end{tcolorbox}
        
    \begin{tcolorbox}[breakable, size=fbox, boxrule=1pt, pad at break*=1mm,colback=cellbackground, colframe=cellborder]
\prompt{In}{incolor}{6}{\boxspacing}
\begin{Verbatim}[commandchars=\\\{\}]
\PY{c}{\PYZsh{} Evaluate the contour}
\PY{n}{contour}\PY{+w}{ }\PY{o}{=}\PY{+w}{ }\PY{n}{fvarsigma}\PY{o}{.}\PY{p}{(}\PY{n}{C}\PY{p}{)}
\end{Verbatim}
\end{tcolorbox}

            \begin{tcolorbox}[breakable, size=fbox, boxrule=.5pt, pad at break*=1mm, opacityfill=0]
\prompt{Out}{outcolor}{6}{\boxspacing}
\begin{Verbatim}[commandchars=\\\{\}]
360-element Vector\{ComplexF64\}:
                1.0 + 0.0im
 0.9998468455013823 + 0.017501015425828787im
 0.9993874289181298 + 0.034996670133171884im
 0.9986218909736757 + 0.052481605045579915im
 0.9975504661591799 + 0.0699504643701731im
 0.9961734826617031 + 0.08739789723816892im
 0.9944913622636795 + 0.10481855934390122im
 0.9925046202137215 + 0.1222071145818292im
 0.9902138650687943 + 0.13955823668103456im
  0.987619798507809 + 0.15686661083670625im
 0.9847232151166929 + 0.17412693533811274im
 0.9815250021449998 + 0.19133392319256415im
 0.9780261392341382 + 0.20848230374486532im
                    ⋮
 0.9815250021449997 - 0.19133392319256481im
 0.9847232151166929 - 0.17412693533811274im
  0.987619798507809 - 0.15686661083670642im
 0.9902138650687943 - 0.13955823668103498im
 0.9925046202137214 - 0.1222071145818298im
 0.9944913622636795 - 0.10481855934390114im
  0.996173482661703 - 0.08739789723816904im
 0.9975504661591799 - 0.06995046437017344im
 0.9986218909736756 - 0.052481605045580436im
 0.9993874289181298 - 0.034996670133171724im
 0.9998468455013823 - 0.01750101542582883im
                1.0 - 2.4492935982947064e-16im
\end{Verbatim}
\end{tcolorbox}
        
    \begin{tcolorbox}[breakable, size=fbox, boxrule=1pt, pad at break*=1mm,colback=cellbackground, colframe=cellborder]
\prompt{In}{incolor}{7}{\boxspacing}
\begin{Verbatim}[commandchars=\\\{\}]
\PY{k}{begin}
\PY{n}{f}\PY{+w}{ }\PY{o}{=}\PY{+w}{ }\PY{n}{GLMakie}\PY{o}{.}\PY{n}{Figure}\PY{p}{(}\PY{p}{)}
\PY{n}{ax}\PY{+w}{ }\PY{o}{=}\PY{+w}{ }\PY{n}{GLMakie}\PY{o}{.}\PY{n}{Axis}\PY{p}{(}\PY{n}{f}\PY{p}{[}\PY{l+m+mi}{1}\PY{p}{,}\PY{+w}{ }\PY{l+m+mi}{1}\PY{p}{]}\PY{p}{,}\PY{+w}{ }\PY{n}{xlabel}\PY{+w}{ }\PY{o}{=}\PY{+w}{ }\PY{l+s}{\PYZdq{}}\PY{l+s}{Re(z)}\PY{l+s}{\PYZdq{}}\PY{p}{,}\PY{+w}{ }\PY{n}{ylabel}\PY{+w}{ }\PY{o}{=}\PY{+w}{ }\PY{l+s}{\PYZdq{}}\PY{l+s}{Im(z)}\PY{l+s}{\PYZdq{}}\PY{p}{,}\PY{+w}{ }\PY{n}{title}\PY{+w}{ }\PY{o}{=}\PY{+w}{ }\PY{l+s+sa}{L}\PY{l+s}{\PYZdq{}\PYZdq{}\PYZdq{}}\PY{l+s}{ e\PYZca{}\PYZob{}z\PYZcb{}}\PY{l+s}{\PYZbs{}}\PY{l+s}{quad }\PY{l+s+se}{\PYZbs{}t}\PY{l+s}{ext\PYZob{}is entire\PYZcb{}}\PY{l+s}{\PYZdq{}\PYZdq{}\PYZdq{}}\PY{p}{)}
\PY{n}{GLMakie}\PY{o}{.}\PY{n}{lines!}\PY{p}{(}\PY{n}{ax}\PY{p}{,}\PY{n}{C}\PY{p}{,}\PY{n}{imag}\PY{p}{(}\PY{n}{contour}\PY{p}{)}\PY{p}{,}\PY{+w}{ }\PY{n}{color}\PY{o}{=}\PY{l+s+ss}{:red}\PY{p}{,}\PY{n}{label}\PY{o}{=}\PY{l+s+sa}{L}\PY{l+s}{\PYZdq{}\PYZdq{}\PYZdq{}}\PY{l+s}{Im(e\PYZca{}\PYZob{}z\PYZcb{})}\PY{l+s}{\PYZdq{}\PYZdq{}\PYZdq{}}\PY{p}{)}
\PY{n}{GLMakie}\PY{o}{.}\PY{n}{lines!}\PY{p}{(}\PY{n}{ax}\PY{p}{,}\PY{n}{C}\PY{p}{,}\PY{n}{real}\PY{p}{(}\PY{n}{contour}\PY{p}{)}\PY{p}{,}\PY{n}{color}\PY{o}{=}\PY{l+s+ss}{:blue}\PY{p}{,}\PY{n}{label}\PY{o}{=}\PY{l+s+sa}{L}\PY{l+s}{\PYZdq{}\PYZdq{}\PYZdq{}}\PY{l+s}{Re(e\PYZca{}\PYZob{}z\PYZcb{})}\PY{l+s}{\PYZdq{}\PYZdq{}\PYZdq{}}\PY{p}{)}
\PY{n}{GLMakie}\PY{o}{.}\PY{n}{lines!}\PY{p}{(}\PY{n}{ax}\PY{p}{,}\PY{+w}{ }\PY{n}{real}\PY{p}{(}\PY{n}{contour}\PY{p}{)}\PY{p}{,}\PY{+w}{ }\PY{n}{imag}\PY{p}{(}\PY{n}{contour}\PY{p}{)}\PY{p}{,}\PY{+w}{ }\PY{n}{zeros}\PY{p}{(}\PY{n}{length}\PY{p}{(}\PY{n}{C}\PY{p}{)}\PY{p}{)}\PY{p}{,}\PY{+w}{ }\PY{n}{color}\PY{o}{=}\PY{l+s+ss}{:green}\PY{p}{,}\PY{+w}{ }\PY{n}{label}\PY{o}{=}\PY{l+s}{\PYZdq{}}\PY{l+s}{A Toy Contour}\PY{l+s}{\PYZdq{}}\PY{p}{,}\PY{+w}{ }\PY{n}{linewidth}\PY{o}{=}\PY{l+m+mi}{4}\PY{p}{)}
\PY{n}{GLMakie}\PY{o}{.}\PY{n}{axislegend}\PY{p}{(}\PY{n}{ax}\PY{p}{)}
\PY{k}{end}
\end{Verbatim}
\end{tcolorbox}

            \begin{tcolorbox}[breakable, size=fbox, boxrule=.5pt, pad at break*=1mm, opacityfill=0]
\prompt{Out}{outcolor}{7}{\boxspacing}
\begin{Verbatim}[commandchars=\\\{\}]
Makie.Legend()
\end{Verbatim}
\end{tcolorbox}
        
    \begin{tcolorbox}[breakable, size=fbox, boxrule=1pt, pad at break*=1mm,colback=cellbackground, colframe=cellborder]
\prompt{In}{incolor}{8}{\boxspacing}
\begin{Verbatim}[commandchars=\\\{\}]
\PY{n}{f}
\end{Verbatim}
\end{tcolorbox}
 
            
\prompt{Out}{outcolor}{8}{}
    
    \begin{center}
    \adjustimage{max size={0.9\linewidth}{0.9\paperheight}}{output_14_0.png}
    \end{center}
    { \hspace*{\fill} \\}
    

    \subsection{Liouville's theorem}\label{liouvilles-theorem}

\begin{itemize}
\tightlist
\item
  if \(f\) is \textbf{entire} and \textbf{bounded} then the \(f\) is
  constant:
\end{itemize}

\[\left|f^{'}(z_0)\right|\,\le \frac{B}{R} \text{where}\,\, B\,\,\text{is bounded},\,\,R>0\,\,\text{letting}\,\,R\to\infty\]

Every \textbf{non-constant} polynominal
\(P(z)\,=\,a_{n}z^{n}+\cdots+a_{0}\,\) with complex coefficient has a
\textbf{root} in \(\mathbb{C}\)

Every polynominal \(P(z)\,=\,a_{n}z^{n}+\cdots+a_{0}\,\) of degree
\(n\ge1\) has precisely \(n\) roots in \(\mathbb{C}\,\) if the roots
denoted by \(w_{1},\cdots w_{n}\) then \(P\) can be factored as:

\[P(z)\,=\,c(z-w_{1})(z-w_{2})\cdots (z-w_{n})\,\, \text{for some}\, c \in \mathbb{C},\,\,c=a_{n}\]

    \subsubsection{Extra notes}\label{extra-notes}

If \(f\) is \textbf{holomorphic} in a \textbf{region} \(\Omega\) that
vanishes on a sequence of distinct points with a limit point in
\(\Omega\,\) \(\implies f\) is identically 0

given two holomorphic functions \(f,g\) in a region \(\Omega\) and
\(f(z)\,=\,g(z)\) for all \(z\) in \textbf{some non-empty} subset of
\(\,\Omega\,\implies\) \(f(z)=g(z)\)

suppose given two functions \(f\) and \(F\) analytic in region
\(\,\Omega\,\) and \(\,\Omega^{\prime}\,\) with
\(\,\Omega\,\subset\Omega^{\prime}\,\,\) if \(f\) and \(F\) agree on the
smaller set \(\Omega\,\),then \(F\) is analytic continuation of \(f\)
into the region \(\Omega^{\prime}\).

Then there can be \textbf{only one} such analytic continuation since
\(F\) is \textbf{uniquely determined} by \(f\)

    \subsection{Morera's Theorem:}\label{moreras-theorem}

If \(f\) is a continuous complex function on an open set \(\Omega\) in
the complex plane, and if for every closed curve \(T\) in \(\Omega\,\)
then

\[\oint_{T}{f_{n}(z)\,dz\,=\,0}\]

If \(\left\{f_{n}\right\}_{n=1}^{\infty}\) is a sequence of
\textbf{holomorphic} functions that converges uniformely to a function
\(f\) in every compact subset of \(\Omega\) then \(f\) is
\textbf{holomorphic} in \(\Omega\)

\[\oint_{T}{f_{n}(z)\,dz\,=\,0}\quad\text{for all}\,\,n \implies \oint_{T}{f(z)\,dz}\]

    \subsubsection{Extra notes}\label{extra-notes}

given \(\delta>0\), let \(\Omega_{\delta}\) denote the subset of
\(\Omega\) defined by:

\[\Omega_{\delta}\,=\left\{\,z\in\Omega\,\,:\,\,\overline{D_{\delta}}(z) \subset \Omega\right\}\]

\[\implies \large{\underset{z \in \Omega}{sup}} \left|F^{\prime}(z)\right|\le \frac{1}{\delta}\,\large{\underset{\zeta \in \Omega}{sup}} \left|F(\zeta)\right|\]

whenever \(F\) is holomorphic in \(\,\Omega\,\) then \(\,F=f_{n}-f\)

This info helps to construct holomorphic functions as a series:

\[F(z)\,=\sum_{n=1}^{\infty}{f_{n}(z)}\]

    \subsection{Holomorphic functions defined in terms of
integrals}\label{holomorphic-functions-defined-in-terms-of-integrals}

\[f(z)\,=\int_{b}^{a}{F(z,s)ds}\]

where \(F\) is \textbf{holomorphic} in the \textbf{first argument \(z\)}
and is \textbf{continues in the second argument \(s\)}

    \subsection{Symmetry principle}\label{symmetry-principle}

if \(I\) denotes the interior part of the boundry
\(\Omega^{+}\,\)and\(\,\Omega^{-}\) then

\[\Omega^{+}\,\cup\,I\,\cup\,\Omega^{-}\,=\Omega\]

if \(f^{+}\) and \(f^{-}\) are holomorphic functions in \(\Omega^{+}\)
and \(\Omega^{-}\) respectively that extend continuously to \(I\) and

\[f^{+}(x)\,=f^{-}(x)\quad\text{for all}\,\, x\in I\]

\(f\) is defined on \(\Omega\) by:

\[f(z)\,=\,\left\{ \begin{array}{cl}f^{+}(z)&: z \in \Omega^{+} \\
f^{+}(z)=f^{-}(z)&:\,\,z\in I \\ 
f^{-}(z)&: z\in\Omega^{-}\end{array}\right.\]

Then \(f\) is holomorphic on all of the \(\Omega\)

    \subsection{Shwartz reflection
principle}\label{shwartz-reflection-principle}

if \(f\) is a \textbf{holomorphic} function in \(\Omega^{+}\) that
extends continuously to \(I\) such that \(f\) is real-valued on
\(I\,\,\), then there exists a function \(F\) \textbf{holomorphic} in
all of \(\Omega\) such that \(F\,f\,\) on \(\Omega^{+}\)

\[F(z)=\sum \overline{a_{n}}(z-z_{0})^{n},\,\,\,\,\overline{f(x)}=f(x)\,\,\text{whenever}\,\, x\in I\]

    \subsubsection{Extra notes}\label{extra-notes}

\textbf{singularity} : points where the function is not holomorphic and
are ``poles''

any function holomorphic in a neighborhood of compact set \(K\) can be
approximated uniformly on \(K\) by rational functions whose
\textbf{singularities} are in \(K^{c}\)

If \(K^{c}\) is connected, any function holomorphic on a neighborhood of
\(K\) can be approximated uniformely on \(K\) by polynominals.

Suppose \(f\) is holomorphic on \(\Omega\) and \(K\subset \Omega\) is
compact, then there exists finitely many segments
\(\gamma_{1}\cdots\gamma_{n}\,\) in \(\Omega-K\) such that

\[f(z)=\sum_{n=1}^{N}{\frac{1}{2\pi i}} \oint_{\gamma_{n}}{\frac{f(\zeta)}{\zeta-z}d\zeta}\quad\text{for all}\,\,z\in K\]

for any line segment \(\gamma\) entirely contained in \(\Omega-K\,\),
there exists a sequence of rational functions with singularities on
\(\gamma\) that approximate the integral
\(\oint_{\gamma}{\frac{f(\zeta)}{(\zeta-z)}d\zeta}\) uniformly on \(K\),
if \(K^{c}\) is connected and \(z_{0}\not\in K\) then the function

\[\dfrac{1}{z-z_{0}}\]

can be approximated uniformly on \(K\) by polynominals


    % Add a bibliography block to the postdoc
    
    
    
\end{document}
