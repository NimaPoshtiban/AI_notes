\documentclass[11pt]{article}

    \usepackage[breakable]{tcolorbox}
    \usepackage{parskip} % Stop auto-indenting (to mimic markdown behaviour)
    

    % Basic figure setup, for now with no caption control since it's done
    % automatically by Pandoc (which extracts ![](path) syntax from Markdown).
    \usepackage{graphicx}
    % Keep aspect ratio if custom image width or height is specified
    \setkeys{Gin}{keepaspectratio}
    % Maintain compatibility with old templates. Remove in nbconvert 6.0
    \let\Oldincludegraphics\includegraphics
    % Ensure that by default, figures have no caption (until we provide a
    % proper Figure object with a Caption API and a way to capture that
    % in the conversion process - todo).
    \usepackage{caption}
    \DeclareCaptionFormat{nocaption}{}
    \captionsetup{format=nocaption,aboveskip=0pt,belowskip=0pt}

    \usepackage{float}
    \floatplacement{figure}{H} % forces figures to be placed at the correct location
    \usepackage{xcolor} % Allow colors to be defined
    \usepackage{enumerate} % Needed for markdown enumerations to work
    \usepackage{geometry} % Used to adjust the document margins
    \usepackage{amsmath} % Equations
    \usepackage{amssymb} % Equations
    \usepackage{textcomp} % defines textquotesingle
    % Hack from http://tex.stackexchange.com/a/47451/13684:
    \AtBeginDocument{%
        \def\PYZsq{\textquotesingle}% Upright quotes in Pygmentized code
    }
    \usepackage{upquote} % Upright quotes for verbatim code
    \usepackage{eurosym} % defines \euro

    \usepackage{iftex}
    \ifPDFTeX
        \usepackage[T1]{fontenc}
        \IfFileExists{alphabeta.sty}{
              \usepackage{alphabeta}
          }{
              \usepackage[mathletters]{ucs}
              \usepackage[utf8x]{inputenc}
          }
    \else
        \usepackage{fontspec}
        \usepackage{unicode-math}
    \fi

    \usepackage{fancyvrb} % verbatim replacement that allows latex
    \usepackage{grffile} % extends the file name processing of package graphics
                         % to support a larger range
    \makeatletter % fix for old versions of grffile with XeLaTeX
    \@ifpackagelater{grffile}{2019/11/01}
    {
      % Do nothing on new versions
    }
    {
      \def\Gread@@xetex#1{%
        \IfFileExists{"\Gin@base".bb}%
        {\Gread@eps{\Gin@base.bb}}%
        {\Gread@@xetex@aux#1}%
      }
    }
    \makeatother
    \usepackage[Export]{adjustbox} % Used to constrain images to a maximum size
    \adjustboxset{max size={0.9\linewidth}{0.9\paperheight}}

    % The hyperref package gives us a pdf with properly built
    % internal navigation ('pdf bookmarks' for the table of contents,
    % internal cross-reference links, web links for URLs, etc.)
    \usepackage{hyperref}
    % The default LaTeX title has an obnoxious amount of whitespace. By default,
    % titling removes some of it. It also provides customization options.
    \usepackage{titling}
    \usepackage{longtable} % longtable support required by pandoc >1.10
    \usepackage{booktabs}  % table support for pandoc > 1.12.2
    \usepackage{array}     % table support for pandoc >= 2.11.3
    \usepackage{calc}      % table minipage width calculation for pandoc >= 2.11.1
    \usepackage[inline]{enumitem} % IRkernel/repr support (it uses the enumerate* environment)
    \usepackage[normalem]{ulem} % ulem is needed to support strikethroughs (\sout)
                                % normalem makes italics be italics, not underlines
    \usepackage{soul}      % strikethrough (\st) support for pandoc >= 3.0.0
    \usepackage{mathrsfs}
    

    
    % Colors for the hyperref package
    \definecolor{urlcolor}{rgb}{0,.145,.698}
    \definecolor{linkcolor}{rgb}{.71,0.21,0.01}
    \definecolor{citecolor}{rgb}{.12,.54,.11}

    % ANSI colors
    \definecolor{ansi-black}{HTML}{3E424D}
    \definecolor{ansi-black-intense}{HTML}{282C36}
    \definecolor{ansi-red}{HTML}{E75C58}
    \definecolor{ansi-red-intense}{HTML}{B22B31}
    \definecolor{ansi-green}{HTML}{00A250}
    \definecolor{ansi-green-intense}{HTML}{007427}
    \definecolor{ansi-yellow}{HTML}{DDB62B}
    \definecolor{ansi-yellow-intense}{HTML}{B27D12}
    \definecolor{ansi-blue}{HTML}{208FFB}
    \definecolor{ansi-blue-intense}{HTML}{0065CA}
    \definecolor{ansi-magenta}{HTML}{D160C4}
    \definecolor{ansi-magenta-intense}{HTML}{A03196}
    \definecolor{ansi-cyan}{HTML}{60C6C8}
    \definecolor{ansi-cyan-intense}{HTML}{258F8F}
    \definecolor{ansi-white}{HTML}{C5C1B4}
    \definecolor{ansi-white-intense}{HTML}{A1A6B2}
    \definecolor{ansi-default-inverse-fg}{HTML}{FFFFFF}
    \definecolor{ansi-default-inverse-bg}{HTML}{000000}

    % common color for the border for error outputs.
    \definecolor{outerrorbackground}{HTML}{FFDFDF}

    % commands and environments needed by pandoc snippets
    % extracted from the output of `pandoc -s`
    \providecommand{\tightlist}{%
      \setlength{\itemsep}{0pt}\setlength{\parskip}{0pt}}
    \DefineVerbatimEnvironment{Highlighting}{Verbatim}{commandchars=\\\{\}}
    % Add ',fontsize=\small' for more characters per line
    \newenvironment{Shaded}{}{}
    \newcommand{\KeywordTok}[1]{\textcolor[rgb]{0.00,0.44,0.13}{\textbf{{#1}}}}
    \newcommand{\DataTypeTok}[1]{\textcolor[rgb]{0.56,0.13,0.00}{{#1}}}
    \newcommand{\DecValTok}[1]{\textcolor[rgb]{0.25,0.63,0.44}{{#1}}}
    \newcommand{\BaseNTok}[1]{\textcolor[rgb]{0.25,0.63,0.44}{{#1}}}
    \newcommand{\FloatTok}[1]{\textcolor[rgb]{0.25,0.63,0.44}{{#1}}}
    \newcommand{\CharTok}[1]{\textcolor[rgb]{0.25,0.44,0.63}{{#1}}}
    \newcommand{\StringTok}[1]{\textcolor[rgb]{0.25,0.44,0.63}{{#1}}}
    \newcommand{\CommentTok}[1]{\textcolor[rgb]{0.38,0.63,0.69}{\textit{{#1}}}}
    \newcommand{\OtherTok}[1]{\textcolor[rgb]{0.00,0.44,0.13}{{#1}}}
    \newcommand{\AlertTok}[1]{\textcolor[rgb]{1.00,0.00,0.00}{\textbf{{#1}}}}
    \newcommand{\FunctionTok}[1]{\textcolor[rgb]{0.02,0.16,0.49}{{#1}}}
    \newcommand{\RegionMarkerTok}[1]{{#1}}
    \newcommand{\ErrorTok}[1]{\textcolor[rgb]{1.00,0.00,0.00}{\textbf{{#1}}}}
    \newcommand{\NormalTok}[1]{{#1}}

    % Additional commands for more recent versions of Pandoc
    \newcommand{\ConstantTok}[1]{\textcolor[rgb]{0.53,0.00,0.00}{{#1}}}
    \newcommand{\SpecialCharTok}[1]{\textcolor[rgb]{0.25,0.44,0.63}{{#1}}}
    \newcommand{\VerbatimStringTok}[1]{\textcolor[rgb]{0.25,0.44,0.63}{{#1}}}
    \newcommand{\SpecialStringTok}[1]{\textcolor[rgb]{0.73,0.40,0.53}{{#1}}}
    \newcommand{\ImportTok}[1]{{#1}}
    \newcommand{\DocumentationTok}[1]{\textcolor[rgb]{0.73,0.13,0.13}{\textit{{#1}}}}
    \newcommand{\AnnotationTok}[1]{\textcolor[rgb]{0.38,0.63,0.69}{\textbf{\textit{{#1}}}}}
    \newcommand{\CommentVarTok}[1]{\textcolor[rgb]{0.38,0.63,0.69}{\textbf{\textit{{#1}}}}}
    \newcommand{\VariableTok}[1]{\textcolor[rgb]{0.10,0.09,0.49}{{#1}}}
    \newcommand{\ControlFlowTok}[1]{\textcolor[rgb]{0.00,0.44,0.13}{\textbf{{#1}}}}
    \newcommand{\OperatorTok}[1]{\textcolor[rgb]{0.40,0.40,0.40}{{#1}}}
    \newcommand{\BuiltInTok}[1]{{#1}}
    \newcommand{\ExtensionTok}[1]{{#1}}
    \newcommand{\PreprocessorTok}[1]{\textcolor[rgb]{0.74,0.48,0.00}{{#1}}}
    \newcommand{\AttributeTok}[1]{\textcolor[rgb]{0.49,0.56,0.16}{{#1}}}
    \newcommand{\InformationTok}[1]{\textcolor[rgb]{0.38,0.63,0.69}{\textbf{\textit{{#1}}}}}
    \newcommand{\WarningTok}[1]{\textcolor[rgb]{0.38,0.63,0.69}{\textbf{\textit{{#1}}}}}


    % Define a nice break command that doesn't care if a line doesn't already
    % exist.
    \def\br{\hspace*{\fill} \\* }
    % Math Jax compatibility definitions
    \def\gt{>}
    \def\lt{<}
    \let\Oldtex\TeX
    \let\Oldlatex\LaTeX
    \renewcommand{\TeX}{\textrm{\Oldtex}}
    \renewcommand{\LaTeX}{\textrm{\Oldlatex}}
    % Document parameters
    % Document title
    \title{Complex\_numbers}
    
    
    
    
    
    
    
% Pygments definitions
\makeatletter
\def\PY@reset{\let\PY@it=\relax \let\PY@bf=\relax%
    \let\PY@ul=\relax \let\PY@tc=\relax%
    \let\PY@bc=\relax \let\PY@ff=\relax}
\def\PY@tok#1{\csname PY@tok@#1\endcsname}
\def\PY@toks#1+{\ifx\relax#1\empty\else%
    \PY@tok{#1}\expandafter\PY@toks\fi}
\def\PY@do#1{\PY@bc{\PY@tc{\PY@ul{%
    \PY@it{\PY@bf{\PY@ff{#1}}}}}}}
\def\PY#1#2{\PY@reset\PY@toks#1+\relax+\PY@do{#2}}

\@namedef{PY@tok@w}{\def\PY@tc##1{\textcolor[rgb]{0.73,0.73,0.73}{##1}}}
\@namedef{PY@tok@c}{\let\PY@it=\textit\def\PY@tc##1{\textcolor[rgb]{0.24,0.48,0.48}{##1}}}
\@namedef{PY@tok@cp}{\def\PY@tc##1{\textcolor[rgb]{0.61,0.40,0.00}{##1}}}
\@namedef{PY@tok@k}{\let\PY@bf=\textbf\def\PY@tc##1{\textcolor[rgb]{0.00,0.50,0.00}{##1}}}
\@namedef{PY@tok@kp}{\def\PY@tc##1{\textcolor[rgb]{0.00,0.50,0.00}{##1}}}
\@namedef{PY@tok@kt}{\def\PY@tc##1{\textcolor[rgb]{0.69,0.00,0.25}{##1}}}
\@namedef{PY@tok@o}{\def\PY@tc##1{\textcolor[rgb]{0.40,0.40,0.40}{##1}}}
\@namedef{PY@tok@ow}{\let\PY@bf=\textbf\def\PY@tc##1{\textcolor[rgb]{0.67,0.13,1.00}{##1}}}
\@namedef{PY@tok@nb}{\def\PY@tc##1{\textcolor[rgb]{0.00,0.50,0.00}{##1}}}
\@namedef{PY@tok@nf}{\def\PY@tc##1{\textcolor[rgb]{0.00,0.00,1.00}{##1}}}
\@namedef{PY@tok@nc}{\let\PY@bf=\textbf\def\PY@tc##1{\textcolor[rgb]{0.00,0.00,1.00}{##1}}}
\@namedef{PY@tok@nn}{\let\PY@bf=\textbf\def\PY@tc##1{\textcolor[rgb]{0.00,0.00,1.00}{##1}}}
\@namedef{PY@tok@ne}{\let\PY@bf=\textbf\def\PY@tc##1{\textcolor[rgb]{0.80,0.25,0.22}{##1}}}
\@namedef{PY@tok@nv}{\def\PY@tc##1{\textcolor[rgb]{0.10,0.09,0.49}{##1}}}
\@namedef{PY@tok@no}{\def\PY@tc##1{\textcolor[rgb]{0.53,0.00,0.00}{##1}}}
\@namedef{PY@tok@nl}{\def\PY@tc##1{\textcolor[rgb]{0.46,0.46,0.00}{##1}}}
\@namedef{PY@tok@ni}{\let\PY@bf=\textbf\def\PY@tc##1{\textcolor[rgb]{0.44,0.44,0.44}{##1}}}
\@namedef{PY@tok@na}{\def\PY@tc##1{\textcolor[rgb]{0.41,0.47,0.13}{##1}}}
\@namedef{PY@tok@nt}{\let\PY@bf=\textbf\def\PY@tc##1{\textcolor[rgb]{0.00,0.50,0.00}{##1}}}
\@namedef{PY@tok@nd}{\def\PY@tc##1{\textcolor[rgb]{0.67,0.13,1.00}{##1}}}
\@namedef{PY@tok@s}{\def\PY@tc##1{\textcolor[rgb]{0.73,0.13,0.13}{##1}}}
\@namedef{PY@tok@sd}{\let\PY@it=\textit\def\PY@tc##1{\textcolor[rgb]{0.73,0.13,0.13}{##1}}}
\@namedef{PY@tok@si}{\let\PY@bf=\textbf\def\PY@tc##1{\textcolor[rgb]{0.64,0.35,0.47}{##1}}}
\@namedef{PY@tok@se}{\let\PY@bf=\textbf\def\PY@tc##1{\textcolor[rgb]{0.67,0.36,0.12}{##1}}}
\@namedef{PY@tok@sr}{\def\PY@tc##1{\textcolor[rgb]{0.64,0.35,0.47}{##1}}}
\@namedef{PY@tok@ss}{\def\PY@tc##1{\textcolor[rgb]{0.10,0.09,0.49}{##1}}}
\@namedef{PY@tok@sx}{\def\PY@tc##1{\textcolor[rgb]{0.00,0.50,0.00}{##1}}}
\@namedef{PY@tok@m}{\def\PY@tc##1{\textcolor[rgb]{0.40,0.40,0.40}{##1}}}
\@namedef{PY@tok@gh}{\let\PY@bf=\textbf\def\PY@tc##1{\textcolor[rgb]{0.00,0.00,0.50}{##1}}}
\@namedef{PY@tok@gu}{\let\PY@bf=\textbf\def\PY@tc##1{\textcolor[rgb]{0.50,0.00,0.50}{##1}}}
\@namedef{PY@tok@gd}{\def\PY@tc##1{\textcolor[rgb]{0.63,0.00,0.00}{##1}}}
\@namedef{PY@tok@gi}{\def\PY@tc##1{\textcolor[rgb]{0.00,0.52,0.00}{##1}}}
\@namedef{PY@tok@gr}{\def\PY@tc##1{\textcolor[rgb]{0.89,0.00,0.00}{##1}}}
\@namedef{PY@tok@ge}{\let\PY@it=\textit}
\@namedef{PY@tok@gs}{\let\PY@bf=\textbf}
\@namedef{PY@tok@ges}{\let\PY@bf=\textbf\let\PY@it=\textit}
\@namedef{PY@tok@gp}{\let\PY@bf=\textbf\def\PY@tc##1{\textcolor[rgb]{0.00,0.00,0.50}{##1}}}
\@namedef{PY@tok@go}{\def\PY@tc##1{\textcolor[rgb]{0.44,0.44,0.44}{##1}}}
\@namedef{PY@tok@gt}{\def\PY@tc##1{\textcolor[rgb]{0.00,0.27,0.87}{##1}}}
\@namedef{PY@tok@err}{\def\PY@bc##1{{\setlength{\fboxsep}{\string -\fboxrule}\fcolorbox[rgb]{1.00,0.00,0.00}{1,1,1}{\strut ##1}}}}
\@namedef{PY@tok@kc}{\let\PY@bf=\textbf\def\PY@tc##1{\textcolor[rgb]{0.00,0.50,0.00}{##1}}}
\@namedef{PY@tok@kd}{\let\PY@bf=\textbf\def\PY@tc##1{\textcolor[rgb]{0.00,0.50,0.00}{##1}}}
\@namedef{PY@tok@kn}{\let\PY@bf=\textbf\def\PY@tc##1{\textcolor[rgb]{0.00,0.50,0.00}{##1}}}
\@namedef{PY@tok@kr}{\let\PY@bf=\textbf\def\PY@tc##1{\textcolor[rgb]{0.00,0.50,0.00}{##1}}}
\@namedef{PY@tok@bp}{\def\PY@tc##1{\textcolor[rgb]{0.00,0.50,0.00}{##1}}}
\@namedef{PY@tok@fm}{\def\PY@tc##1{\textcolor[rgb]{0.00,0.00,1.00}{##1}}}
\@namedef{PY@tok@vc}{\def\PY@tc##1{\textcolor[rgb]{0.10,0.09,0.49}{##1}}}
\@namedef{PY@tok@vg}{\def\PY@tc##1{\textcolor[rgb]{0.10,0.09,0.49}{##1}}}
\@namedef{PY@tok@vi}{\def\PY@tc##1{\textcolor[rgb]{0.10,0.09,0.49}{##1}}}
\@namedef{PY@tok@vm}{\def\PY@tc##1{\textcolor[rgb]{0.10,0.09,0.49}{##1}}}
\@namedef{PY@tok@sa}{\def\PY@tc##1{\textcolor[rgb]{0.73,0.13,0.13}{##1}}}
\@namedef{PY@tok@sb}{\def\PY@tc##1{\textcolor[rgb]{0.73,0.13,0.13}{##1}}}
\@namedef{PY@tok@sc}{\def\PY@tc##1{\textcolor[rgb]{0.73,0.13,0.13}{##1}}}
\@namedef{PY@tok@dl}{\def\PY@tc##1{\textcolor[rgb]{0.73,0.13,0.13}{##1}}}
\@namedef{PY@tok@s2}{\def\PY@tc##1{\textcolor[rgb]{0.73,0.13,0.13}{##1}}}
\@namedef{PY@tok@sh}{\def\PY@tc##1{\textcolor[rgb]{0.73,0.13,0.13}{##1}}}
\@namedef{PY@tok@s1}{\def\PY@tc##1{\textcolor[rgb]{0.73,0.13,0.13}{##1}}}
\@namedef{PY@tok@mb}{\def\PY@tc##1{\textcolor[rgb]{0.40,0.40,0.40}{##1}}}
\@namedef{PY@tok@mf}{\def\PY@tc##1{\textcolor[rgb]{0.40,0.40,0.40}{##1}}}
\@namedef{PY@tok@mh}{\def\PY@tc##1{\textcolor[rgb]{0.40,0.40,0.40}{##1}}}
\@namedef{PY@tok@mi}{\def\PY@tc##1{\textcolor[rgb]{0.40,0.40,0.40}{##1}}}
\@namedef{PY@tok@il}{\def\PY@tc##1{\textcolor[rgb]{0.40,0.40,0.40}{##1}}}
\@namedef{PY@tok@mo}{\def\PY@tc##1{\textcolor[rgb]{0.40,0.40,0.40}{##1}}}
\@namedef{PY@tok@ch}{\let\PY@it=\textit\def\PY@tc##1{\textcolor[rgb]{0.24,0.48,0.48}{##1}}}
\@namedef{PY@tok@cm}{\let\PY@it=\textit\def\PY@tc##1{\textcolor[rgb]{0.24,0.48,0.48}{##1}}}
\@namedef{PY@tok@cpf}{\let\PY@it=\textit\def\PY@tc##1{\textcolor[rgb]{0.24,0.48,0.48}{##1}}}
\@namedef{PY@tok@c1}{\let\PY@it=\textit\def\PY@tc##1{\textcolor[rgb]{0.24,0.48,0.48}{##1}}}
\@namedef{PY@tok@cs}{\let\PY@it=\textit\def\PY@tc##1{\textcolor[rgb]{0.24,0.48,0.48}{##1}}}

\def\PYZbs{\char`\\}
\def\PYZus{\char`\_}
\def\PYZob{\char`\{}
\def\PYZcb{\char`\}}
\def\PYZca{\char`\^}
\def\PYZam{\char`\&}
\def\PYZlt{\char`\<}
\def\PYZgt{\char`\>}
\def\PYZsh{\char`\#}
\def\PYZpc{\char`\%}
\def\PYZdl{\char`\$}
\def\PYZhy{\char`\-}
\def\PYZsq{\char`\'}
\def\PYZdq{\char`\"}
\def\PYZti{\char`\~}
% for compatibility with earlier versions
\def\PYZat{@}
\def\PYZlb{[}
\def\PYZrb{]}
\makeatother


    % For linebreaks inside Verbatim environment from package fancyvrb.
    \makeatletter
        \newbox\Wrappedcontinuationbox
        \newbox\Wrappedvisiblespacebox
        \newcommand*\Wrappedvisiblespace {\textcolor{red}{\textvisiblespace}}
        \newcommand*\Wrappedcontinuationsymbol {\textcolor{red}{\llap{\tiny$\m@th\hookrightarrow$}}}
        \newcommand*\Wrappedcontinuationindent {3ex }
        \newcommand*\Wrappedafterbreak {\kern\Wrappedcontinuationindent\copy\Wrappedcontinuationbox}
        % Take advantage of the already applied Pygments mark-up to insert
        % potential linebreaks for TeX processing.
        %        {, <, #, %, $, ' and ": go to next line.
        %        _, }, ^, &, >, - and ~: stay at end of broken line.
        % Use of \textquotesingle for straight quote.
        \newcommand*\Wrappedbreaksatspecials {%
            \def\PYGZus{\discretionary{\char`\_}{\Wrappedafterbreak}{\char`\_}}%
            \def\PYGZob{\discretionary{}{\Wrappedafterbreak\char`\{}{\char`\{}}%
            \def\PYGZcb{\discretionary{\char`\}}{\Wrappedafterbreak}{\char`\}}}%
            \def\PYGZca{\discretionary{\char`\^}{\Wrappedafterbreak}{\char`\^}}%
            \def\PYGZam{\discretionary{\char`\&}{\Wrappedafterbreak}{\char`\&}}%
            \def\PYGZlt{\discretionary{}{\Wrappedafterbreak\char`\<}{\char`\<}}%
            \def\PYGZgt{\discretionary{\char`\>}{\Wrappedafterbreak}{\char`\>}}%
            \def\PYGZsh{\discretionary{}{\Wrappedafterbreak\char`\#}{\char`\#}}%
            \def\PYGZpc{\discretionary{}{\Wrappedafterbreak\char`\%}{\char`\%}}%
            \def\PYGZdl{\discretionary{}{\Wrappedafterbreak\char`\$}{\char`\$}}%
            \def\PYGZhy{\discretionary{\char`\-}{\Wrappedafterbreak}{\char`\-}}%
            \def\PYGZsq{\discretionary{}{\Wrappedafterbreak\textquotesingle}{\textquotesingle}}%
            \def\PYGZdq{\discretionary{}{\Wrappedafterbreak\char`\"}{\char`\"}}%
            \def\PYGZti{\discretionary{\char`\~}{\Wrappedafterbreak}{\char`\~}}%
        }
        % Some characters . , ; ? ! / are not pygmentized.
        % This macro makes them "active" and they will insert potential linebreaks
        \newcommand*\Wrappedbreaksatpunct {%
            \lccode`\~`\.\lowercase{\def~}{\discretionary{\hbox{\char`\.}}{\Wrappedafterbreak}{\hbox{\char`\.}}}%
            \lccode`\~`\,\lowercase{\def~}{\discretionary{\hbox{\char`\,}}{\Wrappedafterbreak}{\hbox{\char`\,}}}%
            \lccode`\~`\;\lowercase{\def~}{\discretionary{\hbox{\char`\;}}{\Wrappedafterbreak}{\hbox{\char`\;}}}%
            \lccode`\~`\:\lowercase{\def~}{\discretionary{\hbox{\char`\:}}{\Wrappedafterbreak}{\hbox{\char`\:}}}%
            \lccode`\~`\?\lowercase{\def~}{\discretionary{\hbox{\char`\?}}{\Wrappedafterbreak}{\hbox{\char`\?}}}%
            \lccode`\~`\!\lowercase{\def~}{\discretionary{\hbox{\char`\!}}{\Wrappedafterbreak}{\hbox{\char`\!}}}%
            \lccode`\~`\/\lowercase{\def~}{\discretionary{\hbox{\char`\/}}{\Wrappedafterbreak}{\hbox{\char`\/}}}%
            \catcode`\.\active
            \catcode`\,\active
            \catcode`\;\active
            \catcode`\:\active
            \catcode`\?\active
            \catcode`\!\active
            \catcode`\/\active
            \lccode`\~`\~
        }
    \makeatother

    \let\OriginalVerbatim=\Verbatim
    \makeatletter
    \renewcommand{\Verbatim}[1][1]{%
        %\parskip\z@skip
        \sbox\Wrappedcontinuationbox {\Wrappedcontinuationsymbol}%
        \sbox\Wrappedvisiblespacebox {\FV@SetupFont\Wrappedvisiblespace}%
        \def\FancyVerbFormatLine ##1{\hsize\linewidth
            \vtop{\raggedright\hyphenpenalty\z@\exhyphenpenalty\z@
                \doublehyphendemerits\z@\finalhyphendemerits\z@
                \strut ##1\strut}%
        }%
        % If the linebreak is at a space, the latter will be displayed as visible
        % space at end of first line, and a continuation symbol starts next line.
        % Stretch/shrink are however usually zero for typewriter font.
        \def\FV@Space {%
            \nobreak\hskip\z@ plus\fontdimen3\font minus\fontdimen4\font
            \discretionary{\copy\Wrappedvisiblespacebox}{\Wrappedafterbreak}
            {\kern\fontdimen2\font}%
        }%

        % Allow breaks at special characters using \PYG... macros.
        \Wrappedbreaksatspecials
        % Breaks at punctuation characters . , ; ? ! and / need catcode=\active
        \OriginalVerbatim[#1,codes*=\Wrappedbreaksatpunct]%
    }
    \makeatother

    % Exact colors from NB
    \definecolor{incolor}{HTML}{303F9F}
    \definecolor{outcolor}{HTML}{D84315}
    \definecolor{cellborder}{HTML}{CFCFCF}
    \definecolor{cellbackground}{HTML}{F7F7F7}

    % prompt
    \makeatletter
    \newcommand{\boxspacing}{\kern\kvtcb@left@rule\kern\kvtcb@boxsep}
    \makeatother
    \newcommand{\prompt}[4]{
        {\ttfamily\llap{{\color{#2}[#3]:\hspace{3pt}#4}}\vspace{-\baselineskip}}
    }
    

    
    % Prevent overflowing lines due to hard-to-break entities
    \sloppy
    % Setup hyperref package
    \hypersetup{
      breaklinks=true,  % so long urls are correctly broken across lines
      colorlinks=true,
      urlcolor=urlcolor,
      linkcolor=linkcolor,
      citecolor=citecolor,
      }
    % Slightly bigger margins than the latex defaults
    
    \geometry{verbose,tmargin=1in,bmargin=1in,lmargin=1in,rmargin=1in}
    
    

\begin{document}
    
    \maketitle
    
    

    
    \begin{tcolorbox}[breakable, size=fbox, boxrule=1pt, pad at break*=1mm,colback=cellbackground, colframe=cellborder]
\prompt{In}{incolor}{3}{\boxspacing}
\begin{Verbatim}[commandchars=\\\{\}]
\PY{k}{import}\PY{+w}{ }\PY{n}{Pkg}
\PY{k}{import}\PY{+w}{ }\PY{n}{Base}
\end{Verbatim}
\end{tcolorbox}

    \section{Complex Numbers}\label{complex-numbers}

    \subsection{A preliminary to complex
numbers}\label{a-preliminary-to-complex-numbers}

\begin{enumerate}
\def\labelenumi{\arabic{enumi})}
\tightlist
\item
  \textbf{\emph{Definition of a complex number}}\\
  A complex number \(\mathit{z}\) is defined in this form:
  \(\mathit{z}\,=\,x+iy\) where \$ x \$is the \(Real\, part\) and \(y\)
  is the \(imaginary\,\,part\,,y\ne0\)
\end{enumerate}

    \begin{tcolorbox}[breakable, size=fbox, boxrule=1pt, pad at break*=1mm,colback=cellbackground, colframe=cellborder]
\prompt{In}{incolor}{4}{\boxspacing}
\begin{Verbatim}[commandchars=\\\{\}]
\PY{n}{z}\PY{+w}{ }\PY{o}{=}\PY{+w}{ }\PY{l+m+mi}{2}\PY{+w}{ }\PY{o}{+}\PY{+w}{ }\PY{l+m+mi}{2}\PY{n+nb}{im}
\end{Verbatim}
\end{tcolorbox}

            \begin{tcolorbox}[breakable, size=fbox, boxrule=.5pt, pad at break*=1mm, opacityfill=0]
\prompt{Out}{outcolor}{4}{\boxspacing}
\begin{Verbatim}[commandchars=\\\{\}]
2 + 2im
\end{Verbatim}
\end{tcolorbox}
        
    \begin{enumerate}
\def\labelenumi{\arabic{enumi})}
\setcounter{enumi}{1}
\tightlist
\item
  \textbf{\emph{Absolute Value of a complex number}}\\
  \(|\mathit{z}\,| = (x^{2}+y^{2})^{\frac{1}{2}}\)
\end{enumerate}

    \begin{tcolorbox}[breakable, size=fbox, boxrule=1pt, pad at break*=1mm,colback=cellbackground, colframe=cellborder]
\prompt{In}{incolor}{5}{\boxspacing}
\begin{Verbatim}[commandchars=\\\{\}]
\PY{n}{abs}\PY{p}{(}\PY{n}{z}\PY{p}{)}
\end{Verbatim}
\end{tcolorbox}

            \begin{tcolorbox}[breakable, size=fbox, boxrule=.5pt, pad at break*=1mm, opacityfill=0]
\prompt{Out}{outcolor}{5}{\boxspacing}
\begin{Verbatim}[commandchars=\\\{\}]
2.8284271247461903
\end{Verbatim}
\end{tcolorbox}
        
    \begin{enumerate}
\def\labelenumi{\arabic{enumi})}
\setcounter{enumi}{2}
\tightlist
\item
  \textbf{\emph{Complex Conjucate}}\\
  \$ ,\bar\{z\},=,x-iy\$\\
  if \$ z,=,\bar\{z\} ,\$ then \$ z \in \Re \$\\
  if \$ z,=,-\bar\{z\}, \$ then \(z\) is purely imaginary
  \(z \notin \Re\)
\end{enumerate}

    \begin{tcolorbox}[breakable, size=fbox, boxrule=1pt, pad at break*=1mm,colback=cellbackground, colframe=cellborder]
\prompt{In}{incolor}{6}{\boxspacing}
\begin{Verbatim}[commandchars=\\\{\}]
\PY{n}{z̄}\PY{o}{=}\PY{n}{conj}\PY{p}{(}\PY{n}{z}\PY{p}{)}
\end{Verbatim}
\end{tcolorbox}

            \begin{tcolorbox}[breakable, size=fbox, boxrule=.5pt, pad at break*=1mm, opacityfill=0]
\prompt{Out}{outcolor}{6}{\boxspacing}
\begin{Verbatim}[commandchars=\\\{\}]
2 - 2im
\end{Verbatim}
\end{tcolorbox}
        
    \begin{enumerate}
\def\labelenumi{\arabic{enumi})}
\setcounter{enumi}{3}
\tightlist
\item
  \textbf{\emph{Complex numbers real and imaginary parts}}\\
  \(\left|\,z\,\right|^{2}\,\,=\,\,z\bar{z}\) \(\implies\)
  \(\frac{1}{z}\,\,=\,\,\frac{\bar{z}}{\left|\,z\,\right|^{2}}\) where
  \(z\ne0\,\,\,\,\) \(Re(z)\,\,= \huge \,\,\frac{z+\bar{z}}{2}\,\,\) and
  \(\,\,\,Im(z)\,=\huge \,\frac{z-\bar{z}}{2i}\)
\end{enumerate}

    \begin{tcolorbox}[breakable, size=fbox, boxrule=1pt, pad at break*=1mm,colback=cellbackground, colframe=cellborder]
\prompt{In}{incolor}{7}{\boxspacing}
\begin{Verbatim}[commandchars=\\\{\}]
\PY{n}{abs2}\PY{p}{(}\PY{n}{z}\PY{p}{)}
\end{Verbatim}
\end{tcolorbox}

            \begin{tcolorbox}[breakable, size=fbox, boxrule=.5pt, pad at break*=1mm, opacityfill=0]
\prompt{Out}{outcolor}{7}{\boxspacing}
\begin{Verbatim}[commandchars=\\\{\}]
8
\end{Verbatim}
\end{tcolorbox}
        
    \begin{enumerate}
\def\labelenumi{\arabic{enumi})}
\setcounter{enumi}{4}
\tightlist
\item
  \textbf{\emph{Polar form of Complex numbers}}\\
  \(\huge z\,=\,re^{i\theta}\,\,,\) where \$ r \textgreater{} 0,,\$and
  \(\huge r\) is abosolute value of \(z\,\,\,\) and
  \(\theta\,\in\mathbb{R}\, , \, \theta\,=\,\arctan(Im(z)/Re(z))\)\\
  \(\huge e^{i\theta}\,=\,\cos{\theta}+\,i\sin{\theta}\)

  If\(\huge \,\,\,z=\,re^{i\theta}\,\,\) and
  \(\huge\,\,w=se^{i\varphi}\,\,\) then
  \(\huge\,\,\,zw\,=\,rse^{i(\theta+\varphi)}\)
\end{enumerate}

    \begin{tcolorbox}[breakable, size=fbox, boxrule=1pt, pad at break*=1mm,colback=cellbackground, colframe=cellborder]
\prompt{In}{incolor}{15}{\boxspacing}
\begin{Verbatim}[commandchars=\\\{\}]
\PY{k}{using}\PY{+w}{ }\PY{n}{LaTeXStrings}

\PY{n}{println}\PY{p}{(}\PY{l+s}{\PYZdq{}}\PY{l+s}{Demonstration of Polar form if a Complex number}\PY{l+s}{\PYZdq{}}\PY{p}{)}

\PY{n}{θ}\PY{+w}{ }\PY{o}{=}\PY{+w}{ }\PY{n+nb}{π}\PY{+w}{ }\PY{o}{/}\PY{+w}{ }\PY{l+m+mi}{4}\PY{+w}{  }


\PY{n}{arc}\PY{+w}{ }\PY{o}{=}\PY{+w}{ }\PY{n}{range}\PY{p}{(}\PY{l+m+mi}{0}\PY{p}{,}\PY{+w}{ }\PY{n}{stop}\PY{o}{=}\PY{n}{θ}\PY{p}{,}\PY{+w}{ }\PY{n}{length}\PY{o}{=}\PY{l+m+mi}{1000}\PY{p}{)}
\PY{n}{x\PYZus{}arc}\PY{+w}{ }\PY{o}{=}\PY{+w}{ }\PY{n}{cos}\PY{o}{.}\PY{p}{(}\PY{n}{arc}\PY{p}{)}
\PY{n}{y\PYZus{}arc}\PY{+w}{ }\PY{o}{=}\PY{+w}{ }\PY{n}{sin}\PY{o}{.}\PY{p}{(}\PY{n}{arc}\PY{p}{)}


\PY{n}{Plots}\PY{o}{.}\PY{n}{plot}\PY{p}{(}\PY{n}{x\PYZus{}arc}\PY{p}{,}\PY{+w}{ }\PY{n}{y\PYZus{}arc}\PY{p}{,}\PY{+w}{ }\PY{n}{label}\PY{o}{=}\PY{l+s}{\PYZdq{}}\PY{l+s}{θ}\PY{l+s}{\PYZdq{}}\PY{p}{,}\PY{+w}{ }\PY{n}{legend}\PY{o}{=}\PY{l+s+ss}{:right}\PY{p}{,}\PY{+w}{ }\PY{n}{title}\PY{o}{=}\PY{l+s}{\PYZdq{}}\PY{l+s}{z}\PY{l+s}{\PYZdq{}}\PY{p}{,}\PY{+w}{ }\PY{n}{xlabel}\PY{o}{=}\PY{l+s+sa}{L}\PY{l+s}{\PYZdq{}\PYZdq{}\PYZdq{}}\PY{l+s}{\PYZbs{}}\PY{l+s}{mathbb\PYZob{}R\PYZcb{}}\PY{l+s}{\PYZdq{}\PYZdq{}\PYZdq{}}\PY{p}{,}\PY{+w}{ }\PY{n}{ylabel}\PY{o}{=}\PY{l+s+sa}{L}\PY{l+s}{\PYZdq{}\PYZdq{}\PYZdq{}}\PY{l+s}{\PYZbs{}}\PY{l+s}{mathbf\PYZob{}i\PYZcb{}}\PY{l+s}{\PYZdq{}\PYZdq{}\PYZdq{}}\PY{p}{,}\PY{+w}{ }\PY{n}{line}\PY{o}{=}\PY{p}{(}\PY{l+s+ss}{:solid}\PY{p}{,}\PY{+w}{ }\PY{l+s+ss}{:blue}\PY{p}{)}\PY{p}{)}


\PY{n}{Plots}\PY{o}{.}\PY{n}{plot!}\PY{p}{(}\PY{p}{[}\PY{l+m+mi}{0}\PY{p}{,}\PY{+w}{ }\PY{l+m+mi}{10}\PY{p}{]}\PY{p}{,}\PY{+w}{ }\PY{p}{[}\PY{l+m+mi}{0}\PY{p}{,}\PY{+w}{ }\PY{l+m+mi}{10}\PY{p}{]}\PY{p}{,}\PY{+w}{ }\PY{n}{label}\PY{o}{=}\PY{l+s}{\PYZdq{}}\PY{l+s}{r}\PY{l+s}{\PYZdq{}}\PY{p}{,}\PY{+w}{ }\PY{n}{line}\PY{o}{=}\PY{p}{(}\PY{l+s+ss}{:solid}\PY{p}{,}\PY{+w}{ }\PY{l+s+ss}{:red}\PY{p}{)}\PY{p}{)}


\PY{n}{plot!}\PY{p}{(}\PY{p}{[}\PY{l+m+mi}{0}\PY{p}{,}\PY{+w}{ }\PY{l+m+mi}{10}\PY{p}{]}\PY{p}{,}\PY{+w}{ }\PY{p}{[}\PY{l+m+mi}{0}\PY{p}{,}\PY{+w}{ }\PY{l+m+mi}{0}\PY{p}{]}\PY{p}{,}\PY{+w}{ }\PY{n}{label}\PY{o}{=}\PY{l+s}{\PYZdq{}}\PY{l+s}{Real\PYZhy{}axis}\PY{l+s}{\PYZdq{}}\PY{p}{,}\PY{+w}{ }\PY{n}{line}\PY{o}{=}\PY{p}{(}\PY{l+s+ss}{:dash}\PY{p}{,}\PY{+w}{ }\PY{l+s+ss}{:black}\PY{p}{)}\PY{p}{)}
\PY{n}{scatter!}\PY{p}{(}\PY{p}{[}\PY{l+m+mi}{10}\PY{p}{]}\PY{p}{,}\PY{p}{[}\PY{l+m+mi}{10}\PY{p}{]}\PY{p}{,}\PY{+w}{ }\PY{n}{label}\PY{o}{=}\PY{l+s+sa}{L}\PY{l+s}{\PYZdq{}\PYZdq{}\PYZdq{}}\PY{l+s}{re\PYZca{}\PYZob{}iθ\PYZcb{}}\PY{l+s}{\PYZdq{}\PYZdq{}\PYZdq{}}\PY{p}{,}\PY{+w}{ }\PY{n}{marker}\PY{o}{=}\PY{l+s+ss}{:circle}\PY{p}{,}\PY{+w}{ }\PY{n}{markersize}\PY{o}{=}\PY{l+m+mi}{5}\PY{p}{,}\PY{+w}{ }\PY{n}{color}\PY{o}{=}\PY{l+s+ss}{:green}\PY{p}{)}
\end{Verbatim}
\end{tcolorbox}

    \begin{Verbatim}[commandchars=\\\{\}]
Demonstration of Polar form if a Complex number
    \end{Verbatim}
 
            
\prompt{Out}{outcolor}{15}{}
    
    \begin{center}
    \adjustimage{max size={0.9\linewidth}{0.9\paperheight}}{output_11_1.pdf}
    \end{center}
    { \hspace*{\fill} \\}
    

    \begin{tcolorbox}[breakable, size=fbox, boxrule=1pt, pad at break*=1mm,colback=cellbackground, colframe=cellborder]
\prompt{In}{incolor}{9}{\boxspacing}
\begin{Verbatim}[commandchars=\\\{\}]
\PY{n}{w}\PY{+w}{ }\PY{o}{=}\PY{+w}{ }\PY{l+m+mi}{3}\PY{o}{\PYZhy{}}\PY{l+m+mi}{1}\PY{n+nb}{im}
\PY{n}{printstyled}\PY{p}{(}\PY{l+s}{\PYZdq{}}\PY{l+s}{zw = }\PY{l+s+si}{\PYZdl{}}\PY{p}{(}\PY{n}{z}\PY{o}{*}\PY{n}{w}\PY{p}{)}\PY{l+s+se}{\PYZbs{}n}\PY{l+s}{\PYZdq{}}\PY{p}{,}\PY{n}{color}\PY{o}{=}\PY{l+s+ss}{:blue}\PY{p}{)}
\PY{n}{r}\PY{+w}{ }\PY{o}{=}\PY{+w}{ }\PY{n}{abs}\PY{p}{(}\PY{n}{z}\PY{p}{)}
\PY{n}{s}\PY{+w}{ }\PY{o}{=}\PY{+w}{ }\PY{n}{abs}\PY{p}{(}\PY{n}{w}\PY{p}{)}
\PY{n}{θ}\PY{+w}{ }\PY{o}{=}\PY{+w}{ }\PY{n}{atan}\PY{p}{(}\PY{n}{z}\PY{o}{.}\PY{n+nb}{im}\PY{o}{/}\PY{n}{z}\PY{o}{.}\PY{n}{re}\PY{p}{)}
\PY{n}{φ}\PY{+w}{ }\PY{o}{=}\PY{+w}{ }\PY{n}{atan}\PY{p}{(}\PY{n}{w}\PY{o}{.}\PY{n+nb}{im}\PY{o}{/}\PY{n}{w}\PY{o}{.}\PY{n}{re}\PY{p}{)}
\PY{n}{polarform}\PY{+w}{ }\PY{o}{=}\PY{+w}{ }\PY{n}{r}\PY{o}{*}\PY{n}{s}\PY{o}{*}\PY{n}{exp}\PY{p}{(}\PY{l+m+mi}{1}\PY{p}{)}\PY{o}{\PYZca{}}\PY{p}{(}\PY{l+m+mi}{1}\PY{n+nb}{im}\PY{o}{*}\PY{p}{(}\PY{n}{θ}\PY{+w}{ }\PY{o}{+}\PY{+w}{ }\PY{n}{φ}\PY{p}{)}\PY{p}{)}
\PY{n}{printstyled}\PY{p}{(}\PY{l+s}{\PYZdq{}}\PY{l+s}{rse\PYZca{}i(θ+φ) = }\PY{l+s+si}{\PYZdl{}polarform}\PY{l+s}{\PYZdq{}}\PY{p}{,}\PY{n}{color}\PY{o}{=}\PY{l+s+ss}{:red}\PY{p}{)}
\end{Verbatim}
\end{tcolorbox}

    \begin{Verbatim}[commandchars=\\\{\}]
\textcolor{ansi-blue}{zw = 8 + 4im}
\textcolor{ansi-red}{rse\^{}i(θ+φ) = 8.0 + 4.0im}
    \end{Verbatim}

    \subsubsection{Convergence}\label{convergence}

A sequence \(\,\,\left\{z_{1}\,,z_{2}\,,...\right\}\) of complex numbers
is said to converge to \(\,\,\mathit{w}\,\in\,\mathbb{C}\,\,\) if
\[ lim_{n\to\infty\,}{|\,z_{n}\,-\,\mathit{w}\,\,|}\,\,=0\,\, \,\,\text{and we write} \,\,\,\mathit{w}\,=\,lim_{n\to\infty}{z_{n}} \]
\textbf{\emph{Cauchy sequence (or simply Cauchy)}}\\
A sequence \$,\left\{,z\_\{n\},\right\} \$ is said to be a
\textbf{Cauchy} if
\[ \left|\,z_{n}\,\,-\,\,z_{m}\,\right|\,\to\,0\,\,\,\, \text{as\,\,} n,m\,\to\infty\]
\textbf{Every Cauchy sequence in \(\mathbb{C}\,\,\)has a limit in
\(\,\mathbb{C}\,\)}

    \subsubsection{Sets in the Complex
Plain}\label{sets-in-the-complex-plain}

\textbf{\emph{if}} \(\,z_{0}\,\in\,\mathbb{C}\,\,\) and
\(\,\,r\,>\,0\,\) we define an open disk
\(\mathbf{D}_{r}\left(z_{0}\right)\,\) that is centered at \(z_{0}\) and
has a radius of \(r\)

An \textbf{\emph{open disk}} is defined \$ ,,
\mathbf{D}\emph{\{r\}\left(z}\{0\}\right),=, \left\{,z,\in,\mathbb{C}, :
, \textbar,z,-,z\_\{0\},\textbar{} , \textless,r,\right\} ,\$

A \textbf{\emph{close disk}} is defined \$ ,,
\mathbf{\overline{D}}\emph{\{r\}\left(z}\{0\}\right),=,
\left\{,z,\in,\mathbb{C}, : , \textbar,z,-,z\_\{0\},\textbar{}
,\le,r,\right\} \$

\textbf{\emph{And the boundary of the either closed or open disk is the
circle:}}
\(\,\mathbf{C}_{r}(z_{0})\,=\,\left\{\,\,z\,\in\,\mathbb{C}\, : \, | \,z\,-\,z_{0}\,\,|\,=\,r\,\right\}\)

\textbf{\emph{Unit Disk}} is centered at origin and of radius \textbf{1}
\(\,\mathbb{D}\,=\,\left\{\,z\,\in\,\mathbb{C}\, : \, |\,z\,|\,<\,1\right\}\)

    \begin{tcolorbox}[breakable, size=fbox, boxrule=1pt, pad at break*=1mm,colback=cellbackground, colframe=cellborder]
\prompt{In}{incolor}{16}{\boxspacing}
\begin{Verbatim}[commandchars=\\\{\}]
\PY{k}{using}\PY{+w}{ }\PY{n}{ComplexPlots}\PY{p}{,}\PY{+w}{ }\PY{n}{Plots}\PY{p}{,}\PY{+w}{ }\PY{n}{ComplexValues}
\PY{n}{zc}\PY{+w}{ }\PY{o}{=}\PY{+w}{ }\PY{n}{cispi}\PY{o}{.}\PY{p}{(}\PY{l+m+mi}{2}\PY{o}{*}\PY{p}{(}\PY{l+m+mi}{0}\PY{o}{:}\PY{l+m+mi}{100}\PY{p}{)}\PY{+w}{ }\PY{o}{/}\PY{+w}{ }\PY{l+m+mi}{100}\PY{p}{)}\PY{p}{;}
\PY{n}{Plots}\PY{o}{.}\PY{n}{plot}\PY{p}{(}\PY{n+nd}{@.}\PY{+w}{ }\PY{n}{Polar}\PY{p}{(}\PY{n}{zc}\PY{p}{)}\PY{p}{;}\PY{n}{label}\PY{o}{=}\PY{l+s}{\PYZdq{}}\PY{l+s}{Unit DisK}\PY{l+s}{\PYZdq{}}\PY{p}{,}\PY{p}{)}
\end{Verbatim}
\end{tcolorbox}
 
            
\prompt{Out}{outcolor}{16}{}
    
    \begin{center}
    \adjustimage{max size={0.9\linewidth}{0.9\paperheight}}{output_15_0.pdf}
    \end{center}
    { \hspace*{\fill} \\}
    

    Given a set \(\,\Omega\,\,\subset\,\mathbb{C}\,\), a point \(z_{0}\,\)
is \textbf{\emph{an interior point}} of \(\Omega\) if there exists
\(\,r\,>\,0\) such that \[ \mathbf{D}_{r}(z_{0})\,\subset\,\Omega\]

The interior of \(\Omega\) consists of all its interior points.

A set \(\Omega\) is \textbf{\emph{open}} if every point in that set is
an interior point of \(\Omega\)

A set \(\Omega\) is \textbf{\emph{closed}} if its complement
\(\,\Omega^{\small{c}}\,=\,\mathbb{C}\,-\,\Omega\) is open. this
property can be reformulated in terms of \emph{limit points}. A point
\(\,z\in\,\mathbb{C}\) is said to be a \emph{limit point} of the set
\(\Omega\) if there exists a sequence of points \(\,z_{n}\,\in\,\Omega\)
such that \(\,z_{n}\,\ne\,z\,\) and
\(\,\lim_{n\to\infty}{\,z_{n}\,=\,z}\)

The \textbf{\emph{closure}} of any set \(\Omega\) is the union of
\(\Omega\) and its \emph{limit points}, and is often denoted by
\(\huge \overline{\Omega}\)

The boundary of a set \(\Omega\) is equal to its closure minus its
interior, and is often denoted by \(\huge\partial\Omega\).

A set \(\Omega\) is \textbf{\emph{bounded}} if there exists
\(\,M\,>\,0\,\) such that \(\,|\,z\,|\,<\,M\) whenever
\(\,z\,\in\,\Omega\). In other words, the set \(\Omega\) is contained in
some larger disk.

If \(\Omega\) \textbf{\emph{bounded}}, we define its diameter by:
\[ diam(\Omega)\,=\,\underset{z,w\,\in\,\Omega}{sup}\,|\,z\,-\,w\,|\]

A set \(\Omega\) is said to be \textbf{\emph{compact}} if it is closed
and bounded.

    \begin{tcolorbox}[breakable, size=fbox, boxrule=1pt, pad at break*=1mm,colback=cellbackground, colframe=cellborder]
\prompt{In}{incolor}{11}{\boxspacing}
\begin{Verbatim}[commandchars=\\\{\}]
\PY{n}{zc}\PY{+w}{ }\PY{o}{=}\PY{+w}{ }\PY{n}{cispi}\PY{o}{.}\PY{p}{(}\PY{l+m+mi}{2}\PY{o}{*}\PY{p}{(}\PY{l+m+mi}{0}\PY{o}{:}\PY{l+m+mi}{100}\PY{p}{)}\PY{+w}{ }\PY{o}{/}\PY{+w}{ }\PY{l+m+mi}{100}\PY{p}{)}\PY{p}{;}
\PY{n}{gr}\PY{p}{(}\PY{p}{)}
\PY{n}{Plots}\PY{o}{.}\PY{n}{plot}\PY{p}{(}\PY{n+nd}{@.}\PY{+w}{ }\PY{n}{Polar}\PY{p}{(}\PY{n}{zc}\PY{p}{)}\PY{p}{;}\PY{n}{label}\PY{o}{=}\PY{l+s+sa}{L}\PY{l+s}{\PYZdq{}\PYZdq{}\PYZdq{}}\PY{l+s}{Compact}\PY{l+s}{\PYZbs{}}\PY{l+s}{,}\PY{l+s}{\PYZbs{}}\PY{l+s}{,}\PY{l+s}{\PYZbs{}}\PY{l+s}{Omega}\PY{l+s}{\PYZbs{}}\PY{l+s}{,}\PY{l+s}{\PYZbs{}}\PY{l+s}{, set}\PY{l+s}{\PYZdq{}\PYZdq{}\PYZdq{}}\PY{p}{,}\PY{p}{)}
\PY{n}{Plots}\PY{o}{.}\PY{n}{plot!}\PY{p}{(}\PY{n+nd}{@.}\PY{+w}{ }\PY{n}{Polar}\PY{p}{(}\PY{n}{zc}\PY{o}{+}\PY{n}{zc}\PY{p}{)}\PY{p}{;}\PY{n}{label}\PY{o}{=}\PY{l+s+sa}{L}\PY{l+s}{\PYZdq{}\PYZdq{}\PYZdq{}}\PY{l+s}{Superset}\PY{l+s}{\PYZbs{}}\PY{l+s}{,}\PY{l+s}{\PYZbs{}}\PY{l+s}{, of}\PY{l+s}{\PYZbs{}}\PY{l+s}{,}\PY{l+s}{\PYZbs{}}\PY{l+s}{,}\PY{l+s}{\PYZbs{}}\PY{l+s}{, }\PY{l+s}{\PYZbs{}}\PY{l+s}{Omega}\PY{l+s}{\PYZdq{}\PYZdq{}\PYZdq{}}\PY{p}{,}\PY{n}{legend}\PY{o}{=}\PY{l+s+ss}{:topleft}\PY{p}{)}
\end{Verbatim}
\end{tcolorbox}
 
            
\prompt{Out}{outcolor}{11}{}
    
    \begin{center}
    \adjustimage{max size={0.9\linewidth}{0.9\paperheight}}{output_17_0.pdf}
    \end{center}
    { \hspace*{\fill} \\}
    

    \textbf{\emph{Set Connectedness}}

if an open set \(\Omega\,\in\,\mathbb{C}\) \textbf{\emph{Can not}} be
written as the union of two disjoint open sets:
\(\Omega\,=\,\Omega_{1}\cup\,\Omega_{2}\,\).

it is a connected set and will be called \textbf{\emph{Region}}

a closed set \(F\) is connected if it \textbf{\emph{can not}} be written
as the union of two disjoint closed sets: \(F\,=\,F_{1}\cup\,F_{2}\,\)

Note: an open set \(\Omega\) is connected if any 2 points on \(\Omega\)
can be joined by a curve \(\gamma\) entirely contained in \(\Omega\)

    \begin{tcolorbox}[breakable, size=fbox, boxrule=1pt, pad at break*=1mm,colback=cellbackground, colframe=cellborder]
\prompt{In}{incolor}{19}{\boxspacing}
\begin{Verbatim}[commandchars=\\\{\}]
\PY{k}{import}\PY{+w}{ }\PY{n}{GLMakie}

\PY{c}{\PYZsh{} Parameters for the annular region Omega}
\PY{n}{c}\PY{+w}{ }\PY{o}{=}\PY{+w}{ }\PY{l+m+mf}{0.0}\PY{+w}{ }\PY{o}{+}\PY{+w}{ }\PY{l+m+mf}{0.0}\PY{n+nb}{im}\PY{+w}{  }\PY{c}{\PYZsh{} Center of the annulus in the complex plane}
\PY{n}{r1}\PY{+w}{ }\PY{o}{=}\PY{+w}{ }\PY{l+m+mf}{1.0}\PY{+w}{         }\PY{c}{\PYZsh{} Outer radius}
\PY{n}{r2}\PY{+w}{ }\PY{o}{=}\PY{+w}{ }\PY{l+m+mf}{0.5}\PY{+w}{         }\PY{c}{\PYZsh{} Inner radius}

\PY{c}{\PYZsh{} Parameters for the closed set F (for example, a small circle inside Omega)}
\PY{n}{r\PYZus{}F}\PY{+w}{ }\PY{o}{=}\PY{+w}{ }\PY{l+m+mf}{0.3}\PY{+w}{        }\PY{c}{\PYZsh{} Radius of the closed set F}
\PY{n}{center\PYZus{}F}\PY{+w}{ }\PY{o}{=}\PY{+w}{ }\PY{l+m+mf}{0.5}\PY{+w}{ }\PY{o}{+}\PY{+w}{ }\PY{l+m+mf}{0.5}\PY{n+nb}{im}\PY{+w}{  }\PY{c}{\PYZsh{} Center of the closed set F}

\PY{c}{\PYZsh{} Create a grid of points}
\PY{n}{x}\PY{+w}{ }\PY{o}{=}\PY{+w}{ }\PY{n}{GLMakie}\PY{o}{.}\PY{k+kt}{LinRange}\PY{p}{(}\PY{o}{\PYZhy{}}\PY{l+m+mf}{1.5}\PY{p}{,}\PY{+w}{ }\PY{l+m+mf}{1.5}\PY{p}{,}\PY{+w}{ }\PY{l+m+mi}{400}\PY{p}{)}
\PY{n}{y}\PY{+w}{ }\PY{o}{=}\PY{+w}{ }\PY{n}{GLMakie}\PY{o}{.}\PY{k+kt}{LinRange}\PY{p}{(}\PY{o}{\PYZhy{}}\PY{l+m+mf}{1.5}\PY{p}{,}\PY{+w}{ }\PY{l+m+mf}{1.5}\PY{p}{,}\PY{+w}{ }\PY{l+m+mi}{400}\PY{p}{)}


\PY{n}{xx}\PY{+w}{ }\PY{o}{=}\PY{+w}{ }\PY{n}{repeat}\PY{p}{(}\PY{n}{x}\PY{p}{,}\PY{+w}{ }\PY{n}{length}\PY{p}{(}\PY{n}{y}\PY{p}{)}\PY{p}{)}\PY{+w}{ }\PY{c}{\PYZsh{} Create a 1D vector and repeat it for all y values}
\PY{n}{yy}\PY{+w}{ }\PY{o}{=}\PY{+w}{ }\PY{n}{repeat}\PY{p}{(}\PY{n}{y}\PY{p}{,}\PY{+w}{ }\PY{n}{inner}\PY{+w}{ }\PY{o}{=}\PY{+w}{ }\PY{n}{length}\PY{p}{(}\PY{n}{x}\PY{p}{)}\PY{p}{)}\PY{+w}{ }\PY{c}{\PYZsh{} Repeat y values for all x values}

\PY{c}{\PYZsh{} Reshape xx and yy into matrices to form the mesh grid}
\PY{n}{xx}\PY{+w}{ }\PY{o}{=}\PY{+w}{ }\PY{n}{reshape}\PY{p}{(}\PY{n}{xx}\PY{p}{,}\PY{+w}{ }\PY{n}{length}\PY{p}{(}\PY{n}{x}\PY{p}{)}\PY{p}{,}\PY{+w}{ }\PY{n}{length}\PY{p}{(}\PY{n}{y}\PY{p}{)}\PY{p}{)}
\PY{n}{yy}\PY{+w}{ }\PY{o}{=}\PY{+w}{ }\PY{n}{reshape}\PY{p}{(}\PY{n}{yy}\PY{p}{,}\PY{+w}{ }\PY{n}{length}\PY{p}{(}\PY{n}{x}\PY{p}{)}\PY{p}{,}\PY{+w}{ }\PY{n}{length}\PY{p}{(}\PY{n}{y}\PY{p}{)}\PY{p}{)}

\PY{c}{\PYZsh{} Define the complex grid zz}
\PY{n}{zz}\PY{+w}{ }\PY{o}{=}\PY{+w}{ }\PY{n}{xx}\PY{+w}{ }\PY{o}{.+}\PY{+w}{ }\PY{n+nb}{im}\PY{+w}{ }\PY{o}{.*}\PY{+w}{ }\PY{n}{yy}

\PY{c}{\PYZsh{} Define the region Omega (annulus: r2 \PYZlt{} |z \PYZhy{} c| \PYZlt{} r1)}
\PY{n}{omega}\PY{+w}{ }\PY{o}{=}\PY{+w}{ }\PY{p}{(}\PY{n}{abs}\PY{o}{.}\PY{p}{(}\PY{n}{zz}\PY{+w}{ }\PY{o}{.\PYZhy{}}\PY{+w}{ }\PY{n}{c}\PY{p}{)}\PY{+w}{ }\PY{o}{.\PYZgt{}}\PY{+w}{ }\PY{n}{r2}\PY{p}{)}\PY{+w}{ }\PY{o}{.\PYZam{}}\PY{+w}{ }\PY{p}{(}\PY{n}{abs}\PY{o}{.}\PY{p}{(}\PY{n}{zz}\PY{+w}{ }\PY{o}{.\PYZhy{}}\PY{+w}{ }\PY{n}{c}\PY{p}{)}\PY{+w}{ }\PY{o}{.\PYZlt{}}\PY{+w}{ }\PY{n}{r1}\PY{p}{)}

\PY{c}{\PYZsh{} Plotting the region Omega}
\PY{n}{fig}\PY{+w}{ }\PY{o}{=}\PY{+w}{ }\PY{n}{GLMakie}\PY{o}{.}\PY{n}{Figure}\PY{p}{(}\PY{n}{size}\PY{+w}{ }\PY{o}{=}\PY{+w}{ }\PY{p}{(}\PY{l+m+mi}{800}\PY{p}{,}\PY{+w}{ }\PY{l+m+mi}{600}\PY{p}{)}\PY{p}{)}
\PY{n}{ax}\PY{+w}{ }\PY{o}{=}\PY{+w}{ }\PY{n}{GLMakie}\PY{o}{.}\PY{n}{Axis}\PY{p}{(}\PY{n}{fig}\PY{p}{[}\PY{l+m+mi}{1}\PY{p}{,}\PY{+w}{ }\PY{l+m+mi}{1}\PY{p}{]}\PY{p}{,}\PY{+w}{ }\PY{n}{title}\PY{+w}{ }\PY{o}{=}\PY{+w}{ }\PY{l+s}{\PYZdq{}}\PY{l+s}{Region Ω and Set F in the Complex Plane}\PY{l+s}{\PYZdq{}}\PY{p}{)}

\PY{c}{\PYZsh{} Plot Omega using scatter}
\PY{n}{GLMakie}\PY{o}{.}\PY{n}{scatter!}\PY{p}{(}\PY{n}{ax}\PY{p}{,}\PY{+w}{ }\PY{n}{real}\PY{p}{(}\PY{n}{zz}\PY{p}{[}\PY{n}{omega}\PY{p}{]}\PY{p}{)}\PY{p}{,}\PY{+w}{ }\PY{n}{imag}\PY{p}{(}\PY{n}{zz}\PY{p}{[}\PY{n}{omega}\PY{p}{]}\PY{p}{)}\PY{p}{,}\PY{+w}{ }\PY{n}{color}\PY{+w}{ }\PY{o}{=}\PY{+w}{ }\PY{l+s+ss}{:blue}\PY{p}{,}\PY{+w}{ }\PY{n}{alpha}\PY{+w}{ }\PY{o}{=}\PY{+w}{ }\PY{l+m+mf}{0.3}\PY{p}{,}\PY{+w}{ }\PY{n}{label}\PY{+w}{ }\PY{o}{=}\PY{+w}{ }\PY{l+s}{\PYZdq{}}\PY{l+s}{Ω}\PY{l+s}{\PYZdq{}}\PY{p}{)}

\PY{c}{\PYZsh{} Plot the closed set F (a circle)}
\PY{n}{theta}\PY{+w}{ }\PY{o}{=}\PY{+w}{ }\PY{n}{GLMakie}\PY{o}{.}\PY{k+kt}{LinRange}\PY{p}{(}\PY{l+m+mi}{0}\PY{p}{,}\PY{+w}{ }\PY{l+m+mi}{2}\PY{+w}{ }\PY{o}{*}\PY{+w}{ }\PY{n+nb}{π}\PY{p}{,}\PY{+w}{ }\PY{l+m+mi}{200}\PY{p}{)}
\PY{n}{x\PYZus{}F}\PY{+w}{ }\PY{o}{=}\PY{+w}{ }\PY{n}{real}\PY{p}{(}\PY{n}{center\PYZus{}F}\PY{p}{)}\PY{+w}{ }\PY{o}{.+}\PY{+w}{ }\PY{n}{r\PYZus{}F}\PY{+w}{ }\PY{o}{*}\PY{+w}{ }\PY{n}{cos}\PY{o}{.}\PY{p}{(}\PY{n}{theta}\PY{p}{)}
\PY{n}{y\PYZus{}F}\PY{+w}{ }\PY{o}{=}\PY{+w}{ }\PY{n}{imag}\PY{p}{(}\PY{n}{center\PYZus{}F}\PY{p}{)}\PY{+w}{ }\PY{o}{.+}\PY{+w}{ }\PY{n}{r\PYZus{}F}\PY{+w}{ }\PY{o}{*}\PY{+w}{ }\PY{n}{sin}\PY{o}{.}\PY{p}{(}\PY{n}{theta}\PY{p}{)}

\PY{c}{\PYZsh{} Plot F}
\PY{n}{GLMakie}\PY{o}{.}\PY{n}{plot!}\PY{p}{(}\PY{n}{ax}\PY{p}{,}\PY{+w}{ }\PY{n}{x\PYZus{}F}\PY{p}{,}\PY{+w}{ }\PY{n}{y\PYZus{}F}\PY{p}{,}\PY{+w}{ }\PY{n}{color}\PY{+w}{ }\PY{o}{=}\PY{+w}{ }\PY{l+s+ss}{:red}\PY{p}{,}\PY{+w}{ }\PY{n}{label}\PY{+w}{ }\PY{o}{=}\PY{+w}{ }\PY{l+s}{\PYZdq{}}\PY{l+s}{Set F}\PY{l+s}{\PYZdq{}}\PY{p}{)}

\PY{c}{\PYZsh{} Add axis labels}
\PY{n}{ax}\PY{o}{.}\PY{n}{xlabel}\PY{+w}{ }\PY{o}{=}\PY{+w}{ }\PY{l+s}{\PYZdq{}}\PY{l+s}{Re(z)}\PY{l+s}{\PYZdq{}}
\PY{n}{ax}\PY{o}{.}\PY{n}{ylabel}\PY{+w}{ }\PY{o}{=}\PY{+w}{ }\PY{l+s}{\PYZdq{}}\PY{l+s}{Im(z)}\PY{l+s}{\PYZdq{}}

\PY{c}{\PYZsh{} Display the figure}
\PY{n}{fig}
\end{Verbatim}
\end{tcolorbox}
 
            
\prompt{Out}{outcolor}{19}{}
    
    \begin{center}
    \adjustimage{max size={0.9\linewidth}{0.9\paperheight}}{output_19_0.png}
    \end{center}
    { \hspace*{\fill} \\}
    

    \paragraph{Continues Complex Function}\label{continues-complex-function}

\$ \left\{,z\_\{1\},z\_\{2\},.,.,.,. \right\},
\subset \Omega,,,\lim\emph{\{z}\{n\}\} = z\_\{0\},,\$ then
\(\,\,\lim\,f(z_{n})\,=\,f(z_{0})\,\) \(f\) is continues on \(\Omega\)
if it is continues at every point of \(\Omega\).

Sums and Products of continues functions are also continues.

    \subsubsection{Holomorphic Functions}\label{holomorphic-functions}

Let \(\Omega\) be an open set in \(\mathbb{C}\) and \(\,f\,\) is a
complex-valued function on \(\Omega\).

The function \(\,f\,\) is holomorphic at the point
\(\,z_{0}\,\in\Omega\) if the quotient \$
\frac{f(z_{0}+h)-f(z_{0})}{h},,\text{and},,,,h\ne0, , , h
\in \mathbb{C}\$ converges to a limit when \(h\to 0\).

The limit of quotient when exists, is denoted by \(\,f\acute(z_{0})\)
and is called \textbf{\emph{the derivetive of}} \(\,f\,\) at \(\,z_{0}\)

\[ f\acute(z_{0})\,=lim_{h\to0}{\frac{f(z_{0}+h)-f(z_{0})}{h}}\]

\(f\) is said to be \textbf{\emph{holomorphic}} on \(\Omega\) if \(f\)
is \textbf{\emph{holomorphic}} every point of \(\Omega\)

if \(\mathcal{C}\) is a closed subset of \(\mathbb{C}\), \(f\) is
\textbf{\emph{holomorphic}} in some open set containing \(\mathcal{C}\).

if \(f\) is \textbf{\emph{holomorphic}} in \textbf{all} of
\(\mathbb{C}\,,\) we say that \(f\) is \textbf{\emph{entire}}.

Sometimes the terms \textbf{regular} or \textbf{complex-differentiable}
are used instead of \textbf{\emph{holomorphic}}.

\textbf{\emph{holomorphic}} function is \textbf{\emph{infinitely
differentiable}}, that is the existence of the \textbf{first derivative}
will guarantee the existence of derivatives of any order.

Every \textbf{\emph{holomorphic}} function is analytic in the sense that
it has a power series expansion near every point. So the term analytic
is a synonym for \textbf{\emph{holomorphic}}

It is clear \(f\) is \textbf{\emph{holomorphic}} at
\(\,z_{0}\,\in\Omega\) if and only if there exists a complex number
\(a\) such that

\[f(z_{0}\,+h)\,-f(z_{0})\,-ah\,=\,h\psi(h)\] where \(\psi\) is a
function defined for all small \(h\) and
\(\,lim_{h\to0}{\,\,\,\,\psi(h)}\,=\,0\). Ofcourse we have
\(a\,=\,f\acute(z_{0})\)

if \(f\) and \(g\) are \textbf{\emph{holomorphic}} in \(\Omega\) then:

\begin{enumerate}
\def\labelenumi{\arabic{enumi}.}
\tightlist
\item
  \(f\,+\,g\,\) is \textbf{\emph{holomorphic}} in \(\Omega\) and \$(f,+,
  g)\acute{} ,=,\acute{f} , + \acute{g} \$
\item
  \(fg\,\,\) is \textbf{\emph{holomorphic}} in \(\Omega\) and
  \((fg)\acute{} \,=\acute{f}g\,+\,f\acute{g}\,\)
\item
  if \(g(z_{0}) \,\ne\, 0\,\,\) then \(f/g\) is
  \textbf{\emph{holomorphic}} at \(z_{0}\,\) and
  \[(f/g)\acute{}\,= \,\frac{\acute{f}g\,+\,f\acute{g}}{g^{2}}\]
\end{enumerate}

    \subparagraph{Reminder}\label{reminder}

\textbf{\emph{Gamma Function}}

\begin{enumerate}
\def\labelenumi{\arabic{enumi})}
\item
  \textbf{Main Definition} :
  \[\Huge\Gamma{(z)}\,=\,\Huge \int_{0}^{\infty}{t^{z-1}e^{-t}\, dt}\,\quad \text{where} \,\,\,\Huge \Re{(z)} > 0\,\,\, \]
\item
  \textbf{Euler's definition as an infinite product} :
  \[\Huge \lim_{n\to\infty\,\,}{\dfrac{n!(n+1)^{z}}{(n+z)!}} \quad \Huge \implies \, \Gamma{(z)}\,=\,\tfrac{1}{z}\prod_{n=1}^{ \infty} {\left[\dfrac{1}{1+\frac{z}{n}}\left(1+\dfrac{1}{n}\right)^{\mathcal{z}}\,\,\right]}\]
\end{enumerate}

\begin{enumerate}
\def\labelenumi{\arabic{enumi})}
\setcounter{enumi}{2}
\tightlist
\item
  \textbf{Euler's Reflection formula} :
\end{enumerate}

\[\Huge\Gamma{(1-z)}\Gamma{(z)}\,=\,\dfrac{\pi}{\sin{\pi z}},\,\,\, z \not\in\,\, \mathbb{Z}\,\, \implies \,\Gamma{(z-n)}\,=\,(-1)^{n-1}\dfrac{\Gamma{(-z)}\Gamma{(z+1)}}{\Gamma{(n+1-z)}},\,\,\, n \in \mathbb{Z}\,\,\,\]

\begin{enumerate}
\def\labelenumi{\arabic{enumi})}
\setcounter{enumi}{4}
\item
  \textbf{Weierstrass's definition} :
  \[\Huge \Gamma{(z)}\,=\dfrac{e^{-\gamma z}}{z}\prod_{n=1}^{\infty}\left(1+\dfrac{z}{n}\right)^{-1}\,\mathcal{e}^{z/n},\quad \text{where}\quad \gamma \approx \, 0.577216 \quad \,\,{ \text{is\, the \textbf{Euler–Mascheroni} constant.} }\]
\item
  \textbf{Legendre duplication formula} :\\
  \[\Huge \Gamma{(z)}\Gamma{(z+\dfrac{1}{2})}\,=\,2^{1-2z}\sqrt{\pi}\,\Gamma{(2z)}\]
\end{enumerate}

\textbf{Properties of Gamma Function}

\begin{itemize}
\item
  \[\Huge\Gamma{(z+1)}\,=\,z\Gamma{(z)} \implies \Gamma{(z)}\,=\,\dfrac{\Gamma{(z+1)}}{\large z}\]
\item
  \[\Huge \overline{\Gamma{(z)}}\,=\,\Gamma{(\overline{z})}\quad \implies \quad \Gamma{(z)}\Gamma{(\overline{z})} \in \mathbb{R} \]
\item
  \[\huge \Gamma{(\tfrac{1}{2})}\,=\,\sqrt{\pi} \]
\item
  \[\huge \text{For}\quad \Re(z) > 0,\quad \dfrac{d^{\,n}}{dz^{n}}\Gamma{(z)}\,=\,\int_{0}^{\infty}{t^{z-1}\,\mathcal{e}^{-t}(\log{t})^{n}dt} \]
  \textbf{\emph{ψ Function Definition}}
\end{itemize}

\[\Huge \psi{(z)}\,=\,\dfrac{d}{dz}\ln \left(\Gamma{(z)}\right) \]

    \subsubsection{Complex-valued functions as
mappings}\label{complex-valued-functions-as-mappings}

A complex-valued function
\(\quad\mathcal{f}\,= \mathcal{u}+\mathcal{iv}\quad\) the mapping
\(\quad F(x,y)\,=(\mathcal{u}(x,y),\mathcal{v}(x,y))\quad\) from
\(\quad\mathbb{R}^{2}\,\to\,\mathbb{R}^2\)

\(F\) is differentiable at point \(P_{0}\) if there exists a linear
transformation

\(J\,:\,\mathbb{R}^{2}\,\to\,\mathbb{R}^2\) such that
\[ \dfrac{|F(P_{0}+H)-F(P_{0})-J(H)|}{|H|}\,\to 0 \quad \text{as}\,\,|H|\to 0,\,\,\,\,H\in\mathbb{R}^{2}\]
which can be written as
\[F(P_{0}+h)-F(P_{0})\,=\,J(H)+\,|H|\Psi(H)\quad \text{with}\quad |\Psi(H)|\to 0\quad \text{as}\,\,H\to 0\]

Linear transformation \(\,J\,\) is unique and is called the derivative
of \(\,\,F\,\,\) at \(\,\,P_{0}\)

if \(F\) is defferntiable, the partial derivative of \(\,u\,\) and
\(\,v\,\) exists.
\[\huge J\,=\,J_{F}\left(x,\,y\right)\,=\begin{pmatrix}\quad \dfrac{\partial{u}}{\partial{x}}\quad &  \dfrac{\partial{u}}{\partial{y}} \\  \quad\dfrac{\partial{v}}{\partial{x}}\quad &  \quad\dfrac{\partial{v}}{\partial{y}}\quad \end{pmatrix}\]

\subparagraph{Cauchy--Riemann equation}\label{cauchyriemann-equation}

\[\huge \frac{\partial u}{\partial x}\,=\,\frac{\partial v}{\partial y},\quad \frac{\partial u}{\partial y}\,=\,-\frac{\partial v}{\partial x}\]

or
\[\huge \frac{\partial}{\partial z}\,=\,\frac{1}{2}\left(\frac{\partial}{\partial x}\,+\,\frac{1}{i}\frac{\partial}{\partial y}\right),\quad \frac{\partial}{\partial \overline{z}}\,=\,\frac{1}{2}\left(\frac{\partial}{\partial x}\,-\,\frac{1}{i}\frac{\partial}{\partial y}\right) \]

By Using \textbf{\emph{Cauchy--Riemann equation}} we find the value of
\(\large{f^{'}}(z)\)

\(\large f(z+h)-f(z)\,=\,\left(\frac{\partial u}{\partial x} - i\frac{\partial u}{\partial y}\right)\left(h_{1}+i\,h_{2}\right)\,+|h|\psi{(h)}\quad\text{where}\quad\psi(h)\,=\,\psi(h1)+\psi(h2)\to 0,\,\,h\to 0 \quad \implies
\,\,f^{'}(z)\,=\,2\frac{\partial u}{\partial z}\,=\,\frac{\partial f}{\partial z}\)
\[\huge f^{'}(z)\,=\frac{\partial f}{\partial z}\]

    \paragraph{Power Series}\label{power-series}

\begin{enumerate}
\def\labelenumi{\arabic{enumi})}
\tightlist
\item
  The prime example of a power series is the complex
  \textbf{exponential} function, which is defined for
  \(\large \,\,z\in\mathbb{C}\)
  \[\huge e^{z}\,=\,\sum_{n=0}^{\infty}{\frac{z^{n}}{n!}}\] note that
  \[\Large \left|\frac{z^{n}}{n!}\right|\,=\frac{\left|z\right|^{n}}{n!}\]
  \(\Large e^{z}\) is holomorphic in all of \(\,\mathbb{C}\,\)
  \textbf{(it is entire)}.
\end{enumerate}

\textbf{Its derivative can be found by differentiating the series, term
by term:}
\[\Large\left(e^{z}\right)^{'}\,=\,\sum_{n=0}^{\infty}{n\dfrac{z^{n-1}}{n!}}\,=\,\sum_{m=0}^{\infty}{\frac{z^{m}}{m!}}\,=\,e^{z}\]

In general, a power series is an expansion of the form:
\[\Large\sum_{n=0}^{\infty}{a_{n}{z}^{n}}\] there exists a
\(\large 0\le\,R\,\le\infty\) such that:

\begin{itemize}
\tightlist
\item
  if \(\large\,|z|<R\quad\) the series converges absolutely.
\item
  if \(\large\,|z|>R\quad\) the series diverges.
\end{itemize}

\(R\) is given from Hadamard's formula:
\[\large 1/R\,=\,\lim{sup}\,|a_{n}|^{1/n}\]

the \(\,\large R\,\) is called the \textbf{radius of convergence} of the
power series, and the \textbf{region} \$
\large,\textbar z\textbar\textless R\quad\$ the disk of convergence.
\(\quad\large R\,=\,\infty\quad\) in exponential functions and
\(\large R\,=\,1\,\,\) in geometric series.

\[f(z)\,=\,\sum_{n=0}^{\infty}{a_{n}z^{n}}\quad\text{is a holomorphic function in its disk of convergence and the derivatives of the}\,\,f(z)\quad \text{have the same radius of convergence as}\,\,f(z)\]

\textbf{A power series is infinitely complex-differentiable in its disk
of convergence}

a power series centered at \(\,z_{0}\in\mathbb{C}\,\) is an expression
of form:
\[\Large f(z)\,=\,\sum_{n=0}^{\infty}{a_{n}\left(z-z_{0}\right)^{n}}\] a
function defined on an open set \(\Omega\) is said to be
\textbf{analytic} at a point \(\,z_{0}\in\mathbb{C}\,\) if there exists
a power series \(\,\sum{a_{n}(z-z_{0})^{n}}\,\) centered at
\(\,z_{0}\,\) with positive radius of convergence (\(R\)) such that
\[\Large f(z)\,=\,\sum_{n=0}^{\infty}{a_{n}(z-z_{0})^{n}}\quad\text{for all}\,\,z\,\,\text{in a neighborhood of}\,\,z_{0}\]
if\(\,\,f\,\,\)has a power series expansion at every point in
\(\large\Omega\,\), then \(f\) is \textbf{analytic} on \(\large\Omega\)

    \begin{tcolorbox}[breakable, size=fbox, boxrule=1pt, pad at break*=1mm,colback=cellbackground, colframe=cellborder]
\prompt{In}{incolor}{55}{\boxspacing}
\begin{Verbatim}[commandchars=\\\{\}]
\PY{k}{import}\PY{+w}{ }\PY{n}{GLMakie}
\PY{k}{using}\PY{+w}{ }\PY{n}{GLMakie}
\PY{n}{f}\PY{+w}{ }\PY{o}{=}\PY{+w}{ }\PY{n}{GLMakie}\PY{o}{.}\PY{n}{Figure}\PY{p}{(}\PY{n}{size}\PY{+w}{ }\PY{o}{=}\PY{+w}{ }\PY{p}{(}\PY{l+m+mi}{800}\PY{p}{,}\PY{+w}{ }\PY{l+m+mi}{300}\PY{p}{)}\PY{p}{)}

\PY{n}{ax}\PY{+w}{ }\PY{o}{=}\PY{+w}{ }\PY{n}{GLMakie}\PY{o}{.}\PY{n}{PolarAxis}\PY{p}{(}\PY{n}{f}\PY{p}{[}\PY{l+m+mi}{1}\PY{p}{,}\PY{+w}{ }\PY{l+m+mi}{1}\PY{p}{]}\PY{p}{,}\PY{+w}{ }\PY{n}{title}\PY{+w}{ }\PY{o}{=}\PY{+w}{ }\PY{l+s}{\PYZdq{}}\PY{l+s}{sin(z)}\PY{l+s}{\PYZdq{}}\PY{p}{,}\PY{n}{theta\PYZus{}as\PYZus{}x}\PY{+w}{ }\PY{o}{=}\PY{+w}{ }\PY{n+nb}{false}\PY{p}{)}
\PY{n}{lineobject}\PY{+w}{ }\PY{o}{=}\PY{+w}{ }\PY{n}{GLMakie}\PY{o}{.}\PY{n}{lines!}\PY{p}{(}\PY{n}{ax}\PY{p}{,}\PY{+w}{ }\PY{l+m+mi}{0}\PY{o}{..}\PY{l+m+mi}{2}\PY{n+nb}{pi}\PY{p}{,}\PY{+w}{ }\PY{n}{sinh}\PY{p}{,}\PY{+w}{ }\PY{n}{color}\PY{+w}{ }\PY{o}{=}\PY{+w}{ }\PY{l+s+ss}{:red}\PY{p}{,}\PY{p}{)}

\PY{n}{ax}\PY{+w}{ }\PY{o}{=}\PY{+w}{ }\PY{n}{GLMakie}\PY{o}{.}\PY{n}{PolarAxis}\PY{p}{(}\PY{n}{f}\PY{p}{[}\PY{l+m+mi}{1}\PY{p}{,}\PY{+w}{ }\PY{l+m+mi}{2}\PY{p}{]}\PY{p}{,}\PY{+w}{ }\PY{n}{title}\PY{+w}{ }\PY{o}{=}\PY{+w}{ }\PY{l+s}{\PYZdq{}}\PY{l+s}{cos(z)}\PY{l+s}{\PYZdq{}}\PY{p}{,}\PY{+w}{ }\PY{n}{theta\PYZus{}as\PYZus{}x}\PY{+w}{ }\PY{o}{=}\PY{+w}{ }\PY{n+nb}{false}\PY{p}{)}
\PY{n}{lineobject}\PY{+w}{ }\PY{o}{=}\PY{+w}{ }\PY{n}{GLMakie}\PY{o}{.}\PY{n}{lines!}\PY{p}{(}\PY{n}{ax}\PY{p}{,}\PY{l+m+mi}{0}\PY{o}{..}\PY{l+m+mi}{2}\PY{n+nb}{pi}\PY{p}{,}\PY{+w}{ }\PY{n}{cosh}\PY{p}{,}\PY{+w}{ }\PY{n}{color}\PY{+w}{ }\PY{o}{=}\PY{+w}{ }\PY{l+s+ss}{:orange}\PY{p}{)}

\PY{n}{ax}\PY{+w}{ }\PY{o}{=}\PY{+w}{ }\PY{n}{GLMakie}\PY{o}{.}\PY{n}{PolarAxis}\PY{p}{(}\PY{n}{f}\PY{p}{[}\PY{l+m+mi}{1}\PY{p}{,}\PY{+w}{ }\PY{l+m+mi}{3}\PY{p}{]}\PY{p}{,}\PY{+w}{ }\PY{n}{title}\PY{+w}{ }\PY{o}{=}\PY{+w}{ }\PY{l+s+sa}{L}\PY{l+s}{\PYZdq{}\PYZdq{}\PYZdq{}}\PY{l+s}{\PYZbs{}}\PY{l+s}{mathcal\PYZob{}e\PYZcb{}\PYZca{}\PYZob{}z\PYZcb{}}\PY{l+s}{\PYZdq{}\PYZdq{}\PYZdq{}}\PY{p}{,}\PY{+w}{ }\PY{n}{theta\PYZus{}as\PYZus{}x}\PY{+w}{ }\PY{o}{=}\PY{+w}{ }\PY{n+nb}{false}\PY{p}{)}
\PY{n}{ez}\PY{+w}{ }\PY{o}{=}\PY{+w}{ }\PY{n}{z}\PY{+w}{ }\PY{o}{\PYZhy{}\PYZgt{}}\PY{+w}{ }\PY{n}{exp}\PY{p}{(}\PY{l+m+mi}{1}\PY{p}{)}\PY{o}{\PYZca{}}\PY{n}{z}
\PY{n}{lineobject}\PY{+w}{ }\PY{o}{=}\PY{+w}{ }\PY{n}{GLMakie}\PY{o}{.}\PY{n}{lines!}\PY{p}{(}\PY{n}{ax}\PY{p}{,}\PY{l+m+mi}{0}\PY{o}{..}\PY{l+m+mi}{2}\PY{n+nb}{pi}\PY{p}{,}\PY{+w}{ }\PY{n}{ez}\PY{p}{,}\PY{+w}{ }\PY{n}{color}\PY{+w}{ }\PY{o}{=}\PY{+w}{ }\PY{l+s+ss}{:blue}\PY{p}{)}
\PY{n}{f}
\end{Verbatim}
\end{tcolorbox}
 
            
\prompt{Out}{outcolor}{55}{}
    
    \begin{center}
    \adjustimage{max size={0.9\linewidth}{0.9\paperheight}}{output_25_0.png}
    \end{center}
    { \hspace*{\fill} \\}
    

    \subsubsection{Integration along curves}\label{integration-along-curves}

    \paragraph{Parameterized Curve}\label{parameterized-curve}

A function \(\,\mathcal{z}(t)\,\) that \textbf{maps a closed interval}
\(\left[a,b\right]\,\in \mathbb{R}\,\) to a \textbf{complex plane}.

\paragraph{Smooth Curve}\label{smooth-curve}

If \(\,\mathcal{z^{'}}(t)\,\) exists and is continues on
\(\left[a,b\right],\,\mathcal{z^{'}}(t)\,\ne 0\,\,\) for
\(\,\,t\,\in\,\left[a,b\right]\,\)

\begin{enumerate}
\def\labelenumi{\arabic{enumi})}
\item
  \(t\,=\,a\) : left-sided limit
  \(\,\,\mathcal{z^{'}}(a)\,=\,\lim_{\underset{h>0}{h\to 0}}{\dfrac{\mathcal{z}(a+h)-\mathcal{z}(a)}{h}}\,\,\)
\item
  \(t\,=\,b\) : right-sided limit
  \(\,\,\mathcal{z^{'}}(b)\,=\,\lim_{\underset{h<0}{h\to 0}}{\dfrac{\mathcal{z}(b+h)-\mathcal{z}(b)}{h}}\,\,\)
\end{enumerate}

\paragraph{Piecewise-smooth Curve}\label{piecewise-smooth-curve}

if \(\,\mathcal{z}\,\) is continues on \(\left[a,b\right]\,\) and if
there exists points:\[a=a_{0}<a_{1}<a_{2}<\cdots<a_{n}=b\] where
\(\,\mathcal{z}(t)\) is smooth in the intervals
\(\,\left[a_{k},a_{k+1}\right]\,\) and \(a_{k}\) right-sided limit may
differ from left-sided

\paragraph{Two Parameterization
Equality}\label{two-parameterization-equality}

\[ \mathcal{z}\,:\,\left[a,b\right]\to \mathbb{C},\,\,\mathcal{\overline{z}}\,:\,\left[c,d\right]\to \mathbb{C}\]
If there exists a continuously differentiable bijection
\(\,s\mapsto\,t(s)\,\, \text{from}\,\,\left[c,d\right]\,\,\text{to}\,\,\left[a,b\right],\,\,t^{'}(s)>0\)
\[ \mathcal{\overline{z}}(s)\,=\,\mathcal{z}(t(s)) \]

    \paragraph{γ Curve}\label{ux3b3-curve}

\(\large\gamma\,\) is a smooth curved in \(\mathbb{C}\,\) (image of
\(\left[a,b\right]\,\)) under \(\mathcal{z}\) with \(\mathcal{z}\)
orientation, where \(\,t\,\) travels from \(a\) to \(b\)

\(\large\gamma^{\overline{}}\,\) is reversed oriented \(\large\gamma\)

\(\large\gamma\,\) begins at \(\,\mathcal{z}(a)\,\) and ends at
\(\,\mathcal{z}(b)\)

\(\Large\mathcal{z^{\overline{}}}\,:\,\left[a,b\right]\,\mapsto\mathbb{R}^{2}\)
\[\large\implies\,\mathcal{z^{\overline{}}}(t)\,=\,\mathcal{z}(b+a-t)\]

\paragraph{Closed Curve}\label{closed-curve}

Definition: if \(\,\mathcal{z}(a)\,=\mathcal{z}(b)\,\) for every its
\textbf{parameterizations}.

\paragraph{Simple Curve}\label{simple-curve}

if
\(\mathcal{z}(t)\ne\mathcal{z}(s)\,\,\text{unless}\,\,t=s\,\,\,\text{or}\,\,\, s=a,\,\,\ \text{and}\,\,\,t=b\)

if \(f(\mathcal{z})\,\) is a \textbf{holomorphic} function in the
interior of a \textbf{closed curved} \(\gamma\) then:
\[ \int_{\gamma}{f(\mathcal{z})\,\mathcal{dz}}\,=\,0\]

    \paragraph{Integral of f along γ}\label{integral-of-f-along-ux3b3}

Given a smooth curved \(\gamma\) in \(\,\mathbb{C}\,\) parameterized by
\(\,\mathcal{z}\,:\,\left[a,b\right]\to \mathbb{C}\,\,\) and \(f\) a
continuous function on \(\gamma\),
\[\large \int_{\gamma}{f(\mathcal{z})\,\mathcal{dz}}\,=\,\int_{a}^{b}{f\left(\mathcal{z}\left(t\right)\right)\,z^{'}(t)\,\mathcal{dt}}\]

\paragraph{Piecewise-Smooth Integral}\label{piecewise-smooth-integral}

\[\large \int_{\gamma}{f(\mathcal{z})\,\mathcal{dz}}\,=\,\sum_{k=0}^{n-1}{{\int_{a_{k}}^{a_{k+1}}{f\left(\mathcal{z}\left(t\right)\right)\,z^{'}(t)\,\mathcal{dt}}}}\]

\paragraph{Length of a smooth curve}\label{length-of-a-smooth-curve}

\[\Large length(\gamma)\,=\,\int_{a}^{b}{\left|\mathcal{z}^{'}\left(t\right)\right|\,\mathcal{dt}} \]

\paragraph{Integration along Curve
Properties}\label{integration-along-curve-properties}

\begin{enumerate}
\def\labelenumi{\roman{enumi})}
\tightlist
\item
  Its linear,that is, if \(\alpha,\beta\,\in\mathbb{C}\) then:
  \[\Large\int_{\gamma}{\left(\alpha f(z)\,+\,\beta g(z)\right)\,dz}\,=\,\alpha\int_{\gamma}{f(z)\,dz}\,+\,\beta\int_{\gamma}{g(z)\,dz}\]
\item
  if \(\gamma^{\overline{}}\,\)is\(\,\gamma\,\) with reverse orientation
  then:
  \[\Large\int_{\gamma}{f(z)\,dz}\,=\,-\int_{\gamma^{-}}{f(z)\,dz}\]
\item
  One has the inequality :
  \[\left|\int_{\gamma}{f(z)\,dz}\right|\,\le\,sup{\underset{z\in\gamma}{}\left|f(z)\right|\times length(\gamma)} \]
  if a continuous function \(f\) has a primitive \(F\) in \(\Omega\,\)
  and \(\gamma\) is a curve in \(\Omega\) that begins at \(w_{1}\) and
  ends at \(w_{2}\), then:
  \[\Large \int_{\gamma}{f(z)\,dz}\,=\,F(w_{2})-F(w_{1})\]
\end{enumerate}

if \(\gamma\) is a closed curve in \(\Omega\) and \(f\) is continuous
and has a primitive in \(\Omega\),then:
\[\Large\int_{\gamma}{f(z)\,dz}=0 \]

Since \(\mathcal{C}\) is a \textbf{unit circle} parameterized by
\(\large\,\mathcal{z}(t)\,=\,\mathcal{e}^{it},\,\,0\le t\le 2\pi\)
\[\Huge \implies\, \int_{C}{f(z)\,dz}\,=\,\int_{0}^{2\pi}{\frac{ie^{it}}{e^{it}}\,dt}\,=\,2\pi i\,\ne 0\]


    % Add a bibliography block to the postdoc
    
    
    
\end{document}
