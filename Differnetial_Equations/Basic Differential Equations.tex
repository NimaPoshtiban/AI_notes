\documentclass{article}
\author{Nima Poshtiba}
\date{\today}

\title{A reminder on basic FODEs and SODEs in Real domain}

\begin{document}
\maketitle
\tableofcontents
\section{First Order Differential Equations (FODEs)}
\subsection{Linear FODEs}
\paragraph{Every linear FODE has the following form:}
\[
	P(x)+\frac{dy}{dx}=g(x) \qquad \textbf{P(x) can be empty}
\]
\paragraph{Here are the steps to solve these kinds of equations}
\begin{enumerate}
	\item Change the form to the standard form \[	P(x)+\frac{dy}{dx}=g(x) \qquad \textbf{P(x) can be empty}\]\vspace{2em}
	\item Calculate the Integral Coefficient $\mu(x)$\, where: \[\mu(x)=e^{\int{P(x)dx}}\].\\ \vspace{2em}
	Now multiply the equation by $\mu(x)\,$
	which will be in the following form: \[\mu(x)P(x)+\mu(x)\frac{dy}{dx}=g(x)\mu(x)\].\\
	Then change the equation to the following form: \[(\mu(x)y(x))^{'}=g(x)\mu(x)\].\\ \vspace{2em}
	\item Apply integration to the both side of the equation: \[\int{(\mu(x)y(x))^{'}dx}=\int{g(x)\mu(x)dx}\].\\
	Then result will be: \[\mu(x)y(x)+c=\int{g(x)\mu(x)dx}\].\\
	The final form is: \[y(t) = \frac{\int{\mu(x)g(x)dx}+c}{\mu(x)} \].\\ \vspace{2em}
\end{enumerate}
\subsection{Non-Linear FODEs}
\subsubsection{Separable FODEs}
\paragraph{Each Separable FODEs has the following form:}
\[N(y)\frac{dy}{dx}=M(x)\]
\paragraph{Solving Separable FODEs}
\subparagraph{Normalize the equation to the standard form:}
\[N(y)\frac{dy}{dx}=M(x)\]
\subparagraph{Final result: integrate both sides and solve for y}
\[\int{N(y)dy}=\int{M(x)dx}\]
\subsubsection{Exact-Equations}
\paragraph{The standard form of an Exact-Equation is}
\[M(x,y)+N(x,y)\frac{dy}{dx}=0\]
\paragraph{Solving the equation}
\subparagraph{Change the form to the following form:
\[\Psi_{x}=M(x,y)\quad,\Psi_{y}=N(x,y)\]
Which will change the form to the following:
\[\Psi_{x}+\Psi_{y}\frac{dy}{dx}=0\]
}
\subparagraph{Now Check if $\Psi(x,y)$ does exists by checking whether this equation holds true:
	\[N_{x}=M_{y}\]
}
\subparagraph{assuming the mentioned equation holds true; This will be the implicit answer of the equation:
	\[\Psi(x,y)=c\]
}
\subparagraph{if any initial condition has been provided (e.g. y(1)=0)\, for solving for a explicit answer we use the following formula:
	\[\Psi(x,y)=\int{M\,dx}\quad or \quad \Psi(x,y)=\int{N\,dy}\]
Choose the easier one
}
\subparagraph{depending on the chosen integral we will have:
	\[\Psi(x,y)=M+h(y)\]
	or
	\[\Psi(x,y)=N+h(x)\]
}
\subparagraph{begin to find the value of h(x) or h(y) from $\Psi$.if solved using M:
	\[\Psi_{y}=N\]
	and if solved using N:
	\[\Psi_{x}=M\]
	Extract $h^{'}$ function from the mentioned $\Psi$ function.
}
\subparagraph{To find $h$ function integrate it and we will get a constant:
	\[h=\int{h^{'}} +c \]
}
\subparagraph{Finally substitute the result into equation.}



\subsubsection{Bernoulli Differential Equations}
\paragraph{The form of the equation:
	\[y^{'}+P(x)y=q(x)y^{n}\]
}
\paragraph{Solving First Order Bernoulli Equation:}
\begin{enumerate}
	\item Divide the equation by $y^{n}$:
	\[\Rightarrow y^{-n}y^{'}+P(x)y^{1-n}=q(x)\].\\ \vspace{2em}
	\item Use substitution:
	\[u=y^{n-1}\quad,u^{'}=(1-n)y^{-n}y^{'}\].\\ \vspace{2em}
	\item Plug the substitution
	\[\frac{1}{n-1}u^{'}+P(x)u=q(x)\]. \\ \vspace{2em}
	\item Now the equation has become \textbf{a Linear Equation or a separable Equation depending on the substitution method}.\\ \vspace{2em}
	\item Finally after solving for $\mathbf{u}$, begin solving for $y$.
\end{enumerate}

\pagebreak
\section{Supplementary formulas}
\begin{enumerate}
	\item fractional derivative: \[D^{-a}f(x)=\frac{1}{\Gamma(a)}\int_{0}^{x}{(x-t)^{a-1}f(t)dt}\].\\ \vspace{2em}
	\item complex fractional derivative:
	\[D^{a}f(z)=\frac{\Gamma(a+1)}{2\pi\,i}\oint_{C}{\frac{f(t)}{(t-z)^{a+1}}dt}\]
\end{enumerate}
\end{document}