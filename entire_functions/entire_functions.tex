\documentclass[11pt]{article}

    \usepackage[breakable]{tcolorbox}
    \usepackage{parskip} % Stop auto-indenting (to mimic markdown behaviour)
    

    % Basic figure setup, for now with no caption control since it's done
    % automatically by Pandoc (which extracts ![](path) syntax from Markdown).
    \usepackage{graphicx}
    % Keep aspect ratio if custom image width or height is specified
    \setkeys{Gin}{keepaspectratio}
    % Maintain compatibility with old templates. Remove in nbconvert 6.0
    \let\Oldincludegraphics\includegraphics
    % Ensure that by default, figures have no caption (until we provide a
    % proper Figure object with a Caption API and a way to capture that
    % in the conversion process - todo).
    \usepackage{caption}
    \DeclareCaptionFormat{nocaption}{}
    \captionsetup{format=nocaption,aboveskip=0pt,belowskip=0pt}

    \usepackage{float}
    \floatplacement{figure}{H} % forces figures to be placed at the correct location
    \usepackage{xcolor} % Allow colors to be defined
    \usepackage{enumerate} % Needed for markdown enumerations to work
    \usepackage{geometry} % Used to adjust the document margins
    \usepackage{amsmath} % Equations
    \usepackage{amssymb} % Equations
    \usepackage{textcomp} % defines textquotesingle
    % Hack from http://tex.stackexchange.com/a/47451/13684:
    \AtBeginDocument{%
        \def\PYZsq{\textquotesingle}% Upright quotes in Pygmentized code
    }
    \usepackage{upquote} % Upright quotes for verbatim code
    \usepackage{eurosym} % defines \euro

    \usepackage{iftex}
    \ifPDFTeX
        \usepackage[T1]{fontenc}
        \IfFileExists{alphabeta.sty}{
              \usepackage{alphabeta}
          }{
              \usepackage[mathletters]{ucs}
              \usepackage[utf8x]{inputenc}
          }
    \else
        \usepackage{fontspec}
        \usepackage{unicode-math}
    \fi

    \usepackage{fancyvrb} % verbatim replacement that allows latex
    \usepackage{grffile} % extends the file name processing of package graphics
                         % to support a larger range
    \makeatletter % fix for old versions of grffile with XeLaTeX
    \@ifpackagelater{grffile}{2019/11/01}
    {
      % Do nothing on new versions
    }
    {
      \def\Gread@@xetex#1{%
        \IfFileExists{"\Gin@base".bb}%
        {\Gread@eps{\Gin@base.bb}}%
        {\Gread@@xetex@aux#1}%
      }
    }
    \makeatother
    \usepackage[Export]{adjustbox} % Used to constrain images to a maximum size
    \adjustboxset{max size={0.9\linewidth}{0.9\paperheight}}

    % The hyperref package gives us a pdf with properly built
    % internal navigation ('pdf bookmarks' for the table of contents,
    % internal cross-reference links, web links for URLs, etc.)
    \usepackage{hyperref}
    % The default LaTeX title has an obnoxious amount of whitespace. By default,
    % titling removes some of it. It also provides customization options.
    \usepackage{titling}
    \usepackage{longtable} % longtable support required by pandoc >1.10
    \usepackage{booktabs}  % table support for pandoc > 1.12.2
    \usepackage{array}     % table support for pandoc >= 2.11.3
    \usepackage{calc}      % table minipage width calculation for pandoc >= 2.11.1
    \usepackage[inline]{enumitem} % IRkernel/repr support (it uses the enumerate* environment)
    \usepackage[normalem]{ulem} % ulem is needed to support strikethroughs (\sout)
                                % normalem makes italics be italics, not underlines
    \usepackage{soul}      % strikethrough (\st) support for pandoc >= 3.0.0
    \usepackage{mathrsfs}
    

    
    % Colors for the hyperref package
    \definecolor{urlcolor}{rgb}{0,.145,.698}
    \definecolor{linkcolor}{rgb}{.71,0.21,0.01}
    \definecolor{citecolor}{rgb}{.12,.54,.11}

    % ANSI colors
    \definecolor{ansi-black}{HTML}{3E424D}
    \definecolor{ansi-black-intense}{HTML}{282C36}
    \definecolor{ansi-red}{HTML}{E75C58}
    \definecolor{ansi-red-intense}{HTML}{B22B31}
    \definecolor{ansi-green}{HTML}{00A250}
    \definecolor{ansi-green-intense}{HTML}{007427}
    \definecolor{ansi-yellow}{HTML}{DDB62B}
    \definecolor{ansi-yellow-intense}{HTML}{B27D12}
    \definecolor{ansi-blue}{HTML}{208FFB}
    \definecolor{ansi-blue-intense}{HTML}{0065CA}
    \definecolor{ansi-magenta}{HTML}{D160C4}
    \definecolor{ansi-magenta-intense}{HTML}{A03196}
    \definecolor{ansi-cyan}{HTML}{60C6C8}
    \definecolor{ansi-cyan-intense}{HTML}{258F8F}
    \definecolor{ansi-white}{HTML}{C5C1B4}
    \definecolor{ansi-white-intense}{HTML}{A1A6B2}
    \definecolor{ansi-default-inverse-fg}{HTML}{FFFFFF}
    \definecolor{ansi-default-inverse-bg}{HTML}{000000}

    % common color for the border for error outputs.
    \definecolor{outerrorbackground}{HTML}{FFDFDF}

    % commands and environments needed by pandoc snippets
    % extracted from the output of `pandoc -s`
    \providecommand{\tightlist}{%
      \setlength{\itemsep}{0pt}\setlength{\parskip}{0pt}}
    \DefineVerbatimEnvironment{Highlighting}{Verbatim}{commandchars=\\\{\}}
    % Add ',fontsize=\small' for more characters per line
    \newenvironment{Shaded}{}{}
    \newcommand{\KeywordTok}[1]{\textcolor[rgb]{0.00,0.44,0.13}{\textbf{{#1}}}}
    \newcommand{\DataTypeTok}[1]{\textcolor[rgb]{0.56,0.13,0.00}{{#1}}}
    \newcommand{\DecValTok}[1]{\textcolor[rgb]{0.25,0.63,0.44}{{#1}}}
    \newcommand{\BaseNTok}[1]{\textcolor[rgb]{0.25,0.63,0.44}{{#1}}}
    \newcommand{\FloatTok}[1]{\textcolor[rgb]{0.25,0.63,0.44}{{#1}}}
    \newcommand{\CharTok}[1]{\textcolor[rgb]{0.25,0.44,0.63}{{#1}}}
    \newcommand{\StringTok}[1]{\textcolor[rgb]{0.25,0.44,0.63}{{#1}}}
    \newcommand{\CommentTok}[1]{\textcolor[rgb]{0.38,0.63,0.69}{\textit{{#1}}}}
    \newcommand{\OtherTok}[1]{\textcolor[rgb]{0.00,0.44,0.13}{{#1}}}
    \newcommand{\AlertTok}[1]{\textcolor[rgb]{1.00,0.00,0.00}{\textbf{{#1}}}}
    \newcommand{\FunctionTok}[1]{\textcolor[rgb]{0.02,0.16,0.49}{{#1}}}
    \newcommand{\RegionMarkerTok}[1]{{#1}}
    \newcommand{\ErrorTok}[1]{\textcolor[rgb]{1.00,0.00,0.00}{\textbf{{#1}}}}
    \newcommand{\NormalTok}[1]{{#1}}

    % Additional commands for more recent versions of Pandoc
    \newcommand{\ConstantTok}[1]{\textcolor[rgb]{0.53,0.00,0.00}{{#1}}}
    \newcommand{\SpecialCharTok}[1]{\textcolor[rgb]{0.25,0.44,0.63}{{#1}}}
    \newcommand{\VerbatimStringTok}[1]{\textcolor[rgb]{0.25,0.44,0.63}{{#1}}}
    \newcommand{\SpecialStringTok}[1]{\textcolor[rgb]{0.73,0.40,0.53}{{#1}}}
    \newcommand{\ImportTok}[1]{{#1}}
    \newcommand{\DocumentationTok}[1]{\textcolor[rgb]{0.73,0.13,0.13}{\textit{{#1}}}}
    \newcommand{\AnnotationTok}[1]{\textcolor[rgb]{0.38,0.63,0.69}{\textbf{\textit{{#1}}}}}
    \newcommand{\CommentVarTok}[1]{\textcolor[rgb]{0.38,0.63,0.69}{\textbf{\textit{{#1}}}}}
    \newcommand{\VariableTok}[1]{\textcolor[rgb]{0.10,0.09,0.49}{{#1}}}
    \newcommand{\ControlFlowTok}[1]{\textcolor[rgb]{0.00,0.44,0.13}{\textbf{{#1}}}}
    \newcommand{\OperatorTok}[1]{\textcolor[rgb]{0.40,0.40,0.40}{{#1}}}
    \newcommand{\BuiltInTok}[1]{{#1}}
    \newcommand{\ExtensionTok}[1]{{#1}}
    \newcommand{\PreprocessorTok}[1]{\textcolor[rgb]{0.74,0.48,0.00}{{#1}}}
    \newcommand{\AttributeTok}[1]{\textcolor[rgb]{0.49,0.56,0.16}{{#1}}}
    \newcommand{\InformationTok}[1]{\textcolor[rgb]{0.38,0.63,0.69}{\textbf{\textit{{#1}}}}}
    \newcommand{\WarningTok}[1]{\textcolor[rgb]{0.38,0.63,0.69}{\textbf{\textit{{#1}}}}}


    % Define a nice break command that doesn't care if a line doesn't already
    % exist.
    \def\br{\hspace*{\fill} \\* }
    % Math Jax compatibility definitions
    \def\gt{>}
    \def\lt{<}
    \let\Oldtex\TeX
    \let\Oldlatex\LaTeX
    \renewcommand{\TeX}{\textrm{\Oldtex}}
    \renewcommand{\LaTeX}{\textrm{\Oldlatex}}
    % Document parameters
    % Document title
    \title{entire\_functions}
    
    
    
    
    
    
    
% Pygments definitions
\makeatletter
\def\PY@reset{\let\PY@it=\relax \let\PY@bf=\relax%
    \let\PY@ul=\relax \let\PY@tc=\relax%
    \let\PY@bc=\relax \let\PY@ff=\relax}
\def\PY@tok#1{\csname PY@tok@#1\endcsname}
\def\PY@toks#1+{\ifx\relax#1\empty\else%
    \PY@tok{#1}\expandafter\PY@toks\fi}
\def\PY@do#1{\PY@bc{\PY@tc{\PY@ul{%
    \PY@it{\PY@bf{\PY@ff{#1}}}}}}}
\def\PY#1#2{\PY@reset\PY@toks#1+\relax+\PY@do{#2}}

\@namedef{PY@tok@w}{\def\PY@tc##1{\textcolor[rgb]{0.73,0.73,0.73}{##1}}}
\@namedef{PY@tok@c}{\let\PY@it=\textit\def\PY@tc##1{\textcolor[rgb]{0.24,0.48,0.48}{##1}}}
\@namedef{PY@tok@cp}{\def\PY@tc##1{\textcolor[rgb]{0.61,0.40,0.00}{##1}}}
\@namedef{PY@tok@k}{\let\PY@bf=\textbf\def\PY@tc##1{\textcolor[rgb]{0.00,0.50,0.00}{##1}}}
\@namedef{PY@tok@kp}{\def\PY@tc##1{\textcolor[rgb]{0.00,0.50,0.00}{##1}}}
\@namedef{PY@tok@kt}{\def\PY@tc##1{\textcolor[rgb]{0.69,0.00,0.25}{##1}}}
\@namedef{PY@tok@o}{\def\PY@tc##1{\textcolor[rgb]{0.40,0.40,0.40}{##1}}}
\@namedef{PY@tok@ow}{\let\PY@bf=\textbf\def\PY@tc##1{\textcolor[rgb]{0.67,0.13,1.00}{##1}}}
\@namedef{PY@tok@nb}{\def\PY@tc##1{\textcolor[rgb]{0.00,0.50,0.00}{##1}}}
\@namedef{PY@tok@nf}{\def\PY@tc##1{\textcolor[rgb]{0.00,0.00,1.00}{##1}}}
\@namedef{PY@tok@nc}{\let\PY@bf=\textbf\def\PY@tc##1{\textcolor[rgb]{0.00,0.00,1.00}{##1}}}
\@namedef{PY@tok@nn}{\let\PY@bf=\textbf\def\PY@tc##1{\textcolor[rgb]{0.00,0.00,1.00}{##1}}}
\@namedef{PY@tok@ne}{\let\PY@bf=\textbf\def\PY@tc##1{\textcolor[rgb]{0.80,0.25,0.22}{##1}}}
\@namedef{PY@tok@nv}{\def\PY@tc##1{\textcolor[rgb]{0.10,0.09,0.49}{##1}}}
\@namedef{PY@tok@no}{\def\PY@tc##1{\textcolor[rgb]{0.53,0.00,0.00}{##1}}}
\@namedef{PY@tok@nl}{\def\PY@tc##1{\textcolor[rgb]{0.46,0.46,0.00}{##1}}}
\@namedef{PY@tok@ni}{\let\PY@bf=\textbf\def\PY@tc##1{\textcolor[rgb]{0.44,0.44,0.44}{##1}}}
\@namedef{PY@tok@na}{\def\PY@tc##1{\textcolor[rgb]{0.41,0.47,0.13}{##1}}}
\@namedef{PY@tok@nt}{\let\PY@bf=\textbf\def\PY@tc##1{\textcolor[rgb]{0.00,0.50,0.00}{##1}}}
\@namedef{PY@tok@nd}{\def\PY@tc##1{\textcolor[rgb]{0.67,0.13,1.00}{##1}}}
\@namedef{PY@tok@s}{\def\PY@tc##1{\textcolor[rgb]{0.73,0.13,0.13}{##1}}}
\@namedef{PY@tok@sd}{\let\PY@it=\textit\def\PY@tc##1{\textcolor[rgb]{0.73,0.13,0.13}{##1}}}
\@namedef{PY@tok@si}{\let\PY@bf=\textbf\def\PY@tc##1{\textcolor[rgb]{0.64,0.35,0.47}{##1}}}
\@namedef{PY@tok@se}{\let\PY@bf=\textbf\def\PY@tc##1{\textcolor[rgb]{0.67,0.36,0.12}{##1}}}
\@namedef{PY@tok@sr}{\def\PY@tc##1{\textcolor[rgb]{0.64,0.35,0.47}{##1}}}
\@namedef{PY@tok@ss}{\def\PY@tc##1{\textcolor[rgb]{0.10,0.09,0.49}{##1}}}
\@namedef{PY@tok@sx}{\def\PY@tc##1{\textcolor[rgb]{0.00,0.50,0.00}{##1}}}
\@namedef{PY@tok@m}{\def\PY@tc##1{\textcolor[rgb]{0.40,0.40,0.40}{##1}}}
\@namedef{PY@tok@gh}{\let\PY@bf=\textbf\def\PY@tc##1{\textcolor[rgb]{0.00,0.00,0.50}{##1}}}
\@namedef{PY@tok@gu}{\let\PY@bf=\textbf\def\PY@tc##1{\textcolor[rgb]{0.50,0.00,0.50}{##1}}}
\@namedef{PY@tok@gd}{\def\PY@tc##1{\textcolor[rgb]{0.63,0.00,0.00}{##1}}}
\@namedef{PY@tok@gi}{\def\PY@tc##1{\textcolor[rgb]{0.00,0.52,0.00}{##1}}}
\@namedef{PY@tok@gr}{\def\PY@tc##1{\textcolor[rgb]{0.89,0.00,0.00}{##1}}}
\@namedef{PY@tok@ge}{\let\PY@it=\textit}
\@namedef{PY@tok@gs}{\let\PY@bf=\textbf}
\@namedef{PY@tok@ges}{\let\PY@bf=\textbf\let\PY@it=\textit}
\@namedef{PY@tok@gp}{\let\PY@bf=\textbf\def\PY@tc##1{\textcolor[rgb]{0.00,0.00,0.50}{##1}}}
\@namedef{PY@tok@go}{\def\PY@tc##1{\textcolor[rgb]{0.44,0.44,0.44}{##1}}}
\@namedef{PY@tok@gt}{\def\PY@tc##1{\textcolor[rgb]{0.00,0.27,0.87}{##1}}}
\@namedef{PY@tok@err}{\def\PY@bc##1{{\setlength{\fboxsep}{\string -\fboxrule}\fcolorbox[rgb]{1.00,0.00,0.00}{1,1,1}{\strut ##1}}}}
\@namedef{PY@tok@kc}{\let\PY@bf=\textbf\def\PY@tc##1{\textcolor[rgb]{0.00,0.50,0.00}{##1}}}
\@namedef{PY@tok@kd}{\let\PY@bf=\textbf\def\PY@tc##1{\textcolor[rgb]{0.00,0.50,0.00}{##1}}}
\@namedef{PY@tok@kn}{\let\PY@bf=\textbf\def\PY@tc##1{\textcolor[rgb]{0.00,0.50,0.00}{##1}}}
\@namedef{PY@tok@kr}{\let\PY@bf=\textbf\def\PY@tc##1{\textcolor[rgb]{0.00,0.50,0.00}{##1}}}
\@namedef{PY@tok@bp}{\def\PY@tc##1{\textcolor[rgb]{0.00,0.50,0.00}{##1}}}
\@namedef{PY@tok@fm}{\def\PY@tc##1{\textcolor[rgb]{0.00,0.00,1.00}{##1}}}
\@namedef{PY@tok@vc}{\def\PY@tc##1{\textcolor[rgb]{0.10,0.09,0.49}{##1}}}
\@namedef{PY@tok@vg}{\def\PY@tc##1{\textcolor[rgb]{0.10,0.09,0.49}{##1}}}
\@namedef{PY@tok@vi}{\def\PY@tc##1{\textcolor[rgb]{0.10,0.09,0.49}{##1}}}
\@namedef{PY@tok@vm}{\def\PY@tc##1{\textcolor[rgb]{0.10,0.09,0.49}{##1}}}
\@namedef{PY@tok@sa}{\def\PY@tc##1{\textcolor[rgb]{0.73,0.13,0.13}{##1}}}
\@namedef{PY@tok@sb}{\def\PY@tc##1{\textcolor[rgb]{0.73,0.13,0.13}{##1}}}
\@namedef{PY@tok@sc}{\def\PY@tc##1{\textcolor[rgb]{0.73,0.13,0.13}{##1}}}
\@namedef{PY@tok@dl}{\def\PY@tc##1{\textcolor[rgb]{0.73,0.13,0.13}{##1}}}
\@namedef{PY@tok@s2}{\def\PY@tc##1{\textcolor[rgb]{0.73,0.13,0.13}{##1}}}
\@namedef{PY@tok@sh}{\def\PY@tc##1{\textcolor[rgb]{0.73,0.13,0.13}{##1}}}
\@namedef{PY@tok@s1}{\def\PY@tc##1{\textcolor[rgb]{0.73,0.13,0.13}{##1}}}
\@namedef{PY@tok@mb}{\def\PY@tc##1{\textcolor[rgb]{0.40,0.40,0.40}{##1}}}
\@namedef{PY@tok@mf}{\def\PY@tc##1{\textcolor[rgb]{0.40,0.40,0.40}{##1}}}
\@namedef{PY@tok@mh}{\def\PY@tc##1{\textcolor[rgb]{0.40,0.40,0.40}{##1}}}
\@namedef{PY@tok@mi}{\def\PY@tc##1{\textcolor[rgb]{0.40,0.40,0.40}{##1}}}
\@namedef{PY@tok@il}{\def\PY@tc##1{\textcolor[rgb]{0.40,0.40,0.40}{##1}}}
\@namedef{PY@tok@mo}{\def\PY@tc##1{\textcolor[rgb]{0.40,0.40,0.40}{##1}}}
\@namedef{PY@tok@ch}{\let\PY@it=\textit\def\PY@tc##1{\textcolor[rgb]{0.24,0.48,0.48}{##1}}}
\@namedef{PY@tok@cm}{\let\PY@it=\textit\def\PY@tc##1{\textcolor[rgb]{0.24,0.48,0.48}{##1}}}
\@namedef{PY@tok@cpf}{\let\PY@it=\textit\def\PY@tc##1{\textcolor[rgb]{0.24,0.48,0.48}{##1}}}
\@namedef{PY@tok@c1}{\let\PY@it=\textit\def\PY@tc##1{\textcolor[rgb]{0.24,0.48,0.48}{##1}}}
\@namedef{PY@tok@cs}{\let\PY@it=\textit\def\PY@tc##1{\textcolor[rgb]{0.24,0.48,0.48}{##1}}}

\def\PYZbs{\char`\\}
\def\PYZus{\char`\_}
\def\PYZob{\char`\{}
\def\PYZcb{\char`\}}
\def\PYZca{\char`\^}
\def\PYZam{\char`\&}
\def\PYZlt{\char`\<}
\def\PYZgt{\char`\>}
\def\PYZsh{\char`\#}
\def\PYZpc{\char`\%}
\def\PYZdl{\char`\$}
\def\PYZhy{\char`\-}
\def\PYZsq{\char`\'}
\def\PYZdq{\char`\"}
\def\PYZti{\char`\~}
% for compatibility with earlier versions
\def\PYZat{@}
\def\PYZlb{[}
\def\PYZrb{]}
\makeatother


    % For linebreaks inside Verbatim environment from package fancyvrb.
    \makeatletter
        \newbox\Wrappedcontinuationbox
        \newbox\Wrappedvisiblespacebox
        \newcommand*\Wrappedvisiblespace {\textcolor{red}{\textvisiblespace}}
        \newcommand*\Wrappedcontinuationsymbol {\textcolor{red}{\llap{\tiny$\m@th\hookrightarrow$}}}
        \newcommand*\Wrappedcontinuationindent {3ex }
        \newcommand*\Wrappedafterbreak {\kern\Wrappedcontinuationindent\copy\Wrappedcontinuationbox}
        % Take advantage of the already applied Pygments mark-up to insert
        % potential linebreaks for TeX processing.
        %        {, <, #, %, $, ' and ": go to next line.
        %        _, }, ^, &, >, - and ~: stay at end of broken line.
        % Use of \textquotesingle for straight quote.
        \newcommand*\Wrappedbreaksatspecials {%
            \def\PYGZus{\discretionary{\char`\_}{\Wrappedafterbreak}{\char`\_}}%
            \def\PYGZob{\discretionary{}{\Wrappedafterbreak\char`\{}{\char`\{}}%
            \def\PYGZcb{\discretionary{\char`\}}{\Wrappedafterbreak}{\char`\}}}%
            \def\PYGZca{\discretionary{\char`\^}{\Wrappedafterbreak}{\char`\^}}%
            \def\PYGZam{\discretionary{\char`\&}{\Wrappedafterbreak}{\char`\&}}%
            \def\PYGZlt{\discretionary{}{\Wrappedafterbreak\char`\<}{\char`\<}}%
            \def\PYGZgt{\discretionary{\char`\>}{\Wrappedafterbreak}{\char`\>}}%
            \def\PYGZsh{\discretionary{}{\Wrappedafterbreak\char`\#}{\char`\#}}%
            \def\PYGZpc{\discretionary{}{\Wrappedafterbreak\char`\%}{\char`\%}}%
            \def\PYGZdl{\discretionary{}{\Wrappedafterbreak\char`\$}{\char`\$}}%
            \def\PYGZhy{\discretionary{\char`\-}{\Wrappedafterbreak}{\char`\-}}%
            \def\PYGZsq{\discretionary{}{\Wrappedafterbreak\textquotesingle}{\textquotesingle}}%
            \def\PYGZdq{\discretionary{}{\Wrappedafterbreak\char`\"}{\char`\"}}%
            \def\PYGZti{\discretionary{\char`\~}{\Wrappedafterbreak}{\char`\~}}%
        }
        % Some characters . , ; ? ! / are not pygmentized.
        % This macro makes them "active" and they will insert potential linebreaks
        \newcommand*\Wrappedbreaksatpunct {%
            \lccode`\~`\.\lowercase{\def~}{\discretionary{\hbox{\char`\.}}{\Wrappedafterbreak}{\hbox{\char`\.}}}%
            \lccode`\~`\,\lowercase{\def~}{\discretionary{\hbox{\char`\,}}{\Wrappedafterbreak}{\hbox{\char`\,}}}%
            \lccode`\~`\;\lowercase{\def~}{\discretionary{\hbox{\char`\;}}{\Wrappedafterbreak}{\hbox{\char`\;}}}%
            \lccode`\~`\:\lowercase{\def~}{\discretionary{\hbox{\char`\:}}{\Wrappedafterbreak}{\hbox{\char`\:}}}%
            \lccode`\~`\?\lowercase{\def~}{\discretionary{\hbox{\char`\?}}{\Wrappedafterbreak}{\hbox{\char`\?}}}%
            \lccode`\~`\!\lowercase{\def~}{\discretionary{\hbox{\char`\!}}{\Wrappedafterbreak}{\hbox{\char`\!}}}%
            \lccode`\~`\/\lowercase{\def~}{\discretionary{\hbox{\char`\/}}{\Wrappedafterbreak}{\hbox{\char`\/}}}%
            \catcode`\.\active
            \catcode`\,\active
            \catcode`\;\active
            \catcode`\:\active
            \catcode`\?\active
            \catcode`\!\active
            \catcode`\/\active
            \lccode`\~`\~
        }
    \makeatother

    \let\OriginalVerbatim=\Verbatim
    \makeatletter
    \renewcommand{\Verbatim}[1][1]{%
        %\parskip\z@skip
        \sbox\Wrappedcontinuationbox {\Wrappedcontinuationsymbol}%
        \sbox\Wrappedvisiblespacebox {\FV@SetupFont\Wrappedvisiblespace}%
        \def\FancyVerbFormatLine ##1{\hsize\linewidth
            \vtop{\raggedright\hyphenpenalty\z@\exhyphenpenalty\z@
                \doublehyphendemerits\z@\finalhyphendemerits\z@
                \strut ##1\strut}%
        }%
        % If the linebreak is at a space, the latter will be displayed as visible
        % space at end of first line, and a continuation symbol starts next line.
        % Stretch/shrink are however usually zero for typewriter font.
        \def\FV@Space {%
            \nobreak\hskip\z@ plus\fontdimen3\font minus\fontdimen4\font
            \discretionary{\copy\Wrappedvisiblespacebox}{\Wrappedafterbreak}
            {\kern\fontdimen2\font}%
        }%

        % Allow breaks at special characters using \PYG... macros.
        \Wrappedbreaksatspecials
        % Breaks at punctuation characters . , ; ? ! and / need catcode=\active
        \OriginalVerbatim[#1,codes*=\Wrappedbreaksatpunct]%
    }
    \makeatother

    % Exact colors from NB
    \definecolor{incolor}{HTML}{303F9F}
    \definecolor{outcolor}{HTML}{D84315}
    \definecolor{cellborder}{HTML}{CFCFCF}
    \definecolor{cellbackground}{HTML}{F7F7F7}

    % prompt
    \makeatletter
    \newcommand{\boxspacing}{\kern\kvtcb@left@rule\kern\kvtcb@boxsep}
    \makeatother
    \newcommand{\prompt}[4]{
        {\ttfamily\llap{{\color{#2}[#3]:\hspace{3pt}#4}}\vspace{-\baselineskip}}
    }
    

    
    % Prevent overflowing lines due to hard-to-break entities
    \sloppy
    % Setup hyperref package
    \hypersetup{
      breaklinks=true,  % so long urls are correctly broken across lines
      colorlinks=true,
      urlcolor=urlcolor,
      linkcolor=linkcolor,
      citecolor=citecolor,
      }
    % Slightly bigger margins than the latex defaults
    
    \geometry{verbose,tmargin=1in,bmargin=1in,lmargin=1in,rmargin=1in}
    
    

\begin{document}
    
    \maketitle
    
    

    
    \begin{tcolorbox}[breakable, size=fbox, boxrule=1pt, pad at break*=1mm,colback=cellbackground, colframe=cellborder]
\prompt{In}{incolor}{1}{\boxspacing}
\begin{Verbatim}[commandchars=\\\{\}]
\PY{k}{import}\PY{+w}{ }\PY{n}{Pkg}
\end{Verbatim}
\end{tcolorbox}

    \section{Entire Functions}\label{entire-functions}

\textbf{\(D_{R},C_{R}\) are open disc and circle of radius \(R\)
centered at origin}

    \subsection{Jensen's Formula}\label{jensens-formula}

Let \(\Omega\) be an open set that contains the closure of a disc
\(D_{R}\) and suppose \(f\) is holomorphic in
\(\Omega\,\,\),\(f(0)\ne0\), and \(f\) vanishes nowhere on the circle
\(C_{R}\). If \(z_{1},z{2},\cdots,z_{n}\) denote the zeros of \(f\)
inside the disc, then

\[\huge\log|f(0)|=\sum_{k=1}^{N}\log(\frac{|z_{k}|}{R})\,+ \frac{1}{2\pi}\int_{0}^{2\pi}\log|f(R\,e^{i\theta})| d\theta\]

If \(f\) is a holomorphic function on the closure of a disk \(D_{R}\),
we denote \(n(r)\) the number of zeros of \(f\) inside the disc
\(D_{r}\), with \(0<r<R\) , \(n(r)\) is a non-decreasing function of
\(r\). We claim that if \(f(0)\ne0\) and \(f\) does not vanish on the
circle \(C_{R}\),then

\[\huge\int_{0}^{R}n(r)\frac{dr}{r}\,=\,\frac{1}{2\pi}\int_{0}^{2\pi}\log|f(R\,e^{i\theta})|d\theta\,\,-\log|f(0)|\]

    \subsection{Functions of finite order}\label{functions-of-finite-order}

Let \(f\) be an entire function. If there exists a positive number
\(\mathcal{p}\) and constants \(A,B>0\) such that

\[\large|f(z)| \le Ae^{B|z|^{\mathcal{p}}}\quad\forall\,\,z\in\mathbb{C}\]

then we say that f has an order of growth \(\le\mathcal{p}\). We define
the order of growth of \(f\) as \(\mathcal{p}_{f}=inf_{\mathcal{p}}\),
where infimum is overall \(p>0\) such that \(f\) has an order of growth
\(\le p\). For example \(\large e^{z^{2}}\) has the order of growth
\(2\).

If \(f\) is an entire function that has an order of growth \(\le p\),
then

\begin{itemize}
\tightlist
\item
  \(\large n(r)\le \mathcal{C}_{r}p\quad\exists\mathcal{C}>0\) and all
  sufficiently large \(r\)
\item
  If \(z_{1},z_{2},\cdots\)denote the zeros of \(f\),with \(z_{k}\ne0\),
  then \(\large\forall s>p\,\) we have
\end{itemize}

\[\huge\sum_{k=1}^{\infty}\frac{1}{|z_{k}|^{s}}<\infty\]

\subsubsection{Examples}\label{examples}

\begin{enumerate}
\def\labelenumi{\arabic{enumi})}
\item
  \(\large f(z)=\dfrac{e^{i\pi z}-e^{-i\pi z}}{2i} \implies |f(z)| \le e^{\pi|z|}\)
  and \(f\) has an order of growth \(\le1\). By taking
  \(z=ix,\,x\in\mathbb{R}\) the order of growth of \(f\) is \(1\).
  However, \(f\) vanishes to order of \(\,1\,\) at
  \(\,z=n\quad\forall n\in\mathbb{Z}\) and
  \(\large\sum_{n\ne0}\frac{1}{|n|^{s}} < \infty\) when \(s>1\)
\item
  \(\large f(z)=\cos z^{1/2}\),which we define by
  \(\large cos z^{1/2}=\sum_{n=0}^{\infty}(-1)^{n}\dfrac{z^{n}}{(2n)!}\).
  Then \(f\) is entire, and it is easy to see that
  \(\large|f(z)|\le e^{|z|^{1/2}}\) and the order of growth of \(f\) is
  \(\dfrac{1}{2}\). Moreover \(f(z)\) vanishes when
  \(z_{n}=((n+\frac{1}{2})\pi)^{2}\), while
  \(\sum_{n}1/|z_{n}|^{s} < \infty\) exactly when
  \(\large s>\frac{1}{2}\)
\end{enumerate}

    \begin{tcolorbox}[breakable, size=fbox, boxrule=1pt, pad at break*=1mm,colback=cellbackground, colframe=cellborder]
\prompt{In}{incolor}{2}{\boxspacing}
\begin{Verbatim}[commandchars=\\\{\}]
\PY{k}{function}\PY{+w}{ }\PY{n}{fz}\PY{p}{(}\PY{n}{z}\PY{p}{)}\PY{o}{::}\PY{k+kt}{Complex}
\PY{+w}{    }\PY{n}{e}\PY{+w}{ }\PY{o}{=}\PY{+w}{ }\PY{n}{exp}\PY{p}{(}\PY{l+m+mi}{1}\PY{p}{)}
\PY{+w}{    }\PY{k}{return}\PY{+w}{ }\PY{p}{(}\PY{n}{e}\PY{o}{\PYZca{}}\PY{p}{(}\PY{l+m+mi}{1}\PY{n+nb}{im}\PY{o}{*}\PY{n+nb}{π}\PY{o}{*}\PY{n}{z}\PY{p}{)}\PY{o}{\PYZhy{}}\PY{n}{e}\PY{o}{\PYZca{}}\PY{p}{(}\PY{o}{\PYZhy{}}\PY{l+m+mi}{1}\PY{n+nb}{im}\PY{o}{*}\PY{n+nb}{π}\PY{o}{*}\PY{n}{z}\PY{p}{)}\PY{p}{)}\PY{+w}{ }\PY{o}{/}\PY{+w}{ }\PY{l+m+mi}{2}\PY{n+nb}{im}
\PY{k}{end}
\end{Verbatim}
\end{tcolorbox}

            \begin{tcolorbox}[breakable, size=fbox, boxrule=.5pt, pad at break*=1mm, opacityfill=0]
\prompt{Out}{outcolor}{2}{\boxspacing}
\begin{Verbatim}[commandchars=\\\{\}]
fz (generic function with 1 method)
\end{Verbatim}
\end{tcolorbox}
        
    \begin{tcolorbox}[breakable, size=fbox, boxrule=1pt, pad at break*=1mm,colback=cellbackground, colframe=cellborder]
\prompt{In}{incolor}{3}{\boxspacing}
\begin{Verbatim}[commandchars=\\\{\}]
\PY{k}{function}\PY{+w}{ }\PY{n}{cosinez}\PY{p}{(}\PY{n}{z}\PY{p}{)}\PY{o}{::}\PY{k+kt}{Complex}
\PY{+w}{    }\PY{n}{z}\PY{+w}{ }\PY{o}{=}\PY{+w}{ }\PY{n}{z}\PY{o}{\PYZca{}}\PY{l+m+mf}{0.5}
\PY{+w}{    }\PY{n}{sum}\PY{+w}{ }\PY{o}{=}\PY{+w}{ }\PY{k+kt}{Complex}\PY{p}{(}\PY{l+m+mi}{0}\PY{p}{,}\PY{l+m+mi}{0}\PY{p}{)}
\PY{+w}{    }\PY{k}{for}\PY{+w}{ }\PY{n}{i}\PY{+w}{ }\PY{o}{=}\PY{+w}{ }\PY{l+m+mi}{0}\PY{o}{:}\PY{l+m+mi}{10}
\PY{+w}{        }\PY{n}{sum}\PY{+w}{ }\PY{o}{+=}\PY{+w}{ }\PY{p}{(}\PY{p}{(}\PY{o}{\PYZhy{}}\PY{l+m+mi}{1}\PY{p}{)}\PY{o}{\PYZca{}}\PY{n}{i}\PY{p}{)}\PY{o}{*}\PY{p}{(}\PY{n}{z}\PY{o}{\PYZca{}}\PY{p}{(}\PY{n}{i}\PY{p}{)}\PY{o}{/}\PY{n}{factorial}\PY{p}{(}\PY{l+m+mi}{2}\PY{o}{*}\PY{n}{i}\PY{p}{)}\PY{p}{)}
\PY{+w}{    }\PY{k}{end}
\PY{+w}{    }\PY{k}{return}\PY{+w}{ }\PY{n}{sum}
\PY{k}{end}
\end{Verbatim}
\end{tcolorbox}

            \begin{tcolorbox}[breakable, size=fbox, boxrule=.5pt, pad at break*=1mm, opacityfill=0]
\prompt{Out}{outcolor}{3}{\boxspacing}
\begin{Verbatim}[commandchars=\\\{\}]
cosinez (generic function with 1 method)
\end{Verbatim}
\end{tcolorbox}
        
    \begin{tcolorbox}[breakable, size=fbox, boxrule=1pt, pad at break*=1mm,colback=cellbackground, colframe=cellborder]
\prompt{In}{incolor}{4}{\boxspacing}
\begin{Verbatim}[commandchars=\\\{\}]
\PY{k}{using}\PY{+w}{ }\PY{n}{GLMakie}
\PY{k}{using}\PY{+w}{ }\PY{n}{LaTeXStrings}
\PY{n}{r}\PY{+w}{ }\PY{o}{=}\PY{+w}{ }\PY{n}{range}\PY{p}{(}\PY{l+m+mi}{0}\PY{p}{,}\PY{l+m+mi}{10}\PY{p}{,}\PY{n}{length}\PY{o}{=}\PY{l+m+mi}{10000}\PY{p}{)}
\PY{n}{real\PYZus{}fz}\PY{+w}{ }\PY{o}{=}\PY{+w}{ }\PY{p}{[}\PY{n}{real}\PY{o}{.}\PY{p}{(}\PY{n}{fz}\PY{p}{(}\PY{n}{z}\PY{p}{)}\PY{p}{)}\PY{+w}{ }\PY{k}{for}\PY{+w}{ }\PY{n}{z}\PY{+w}{ }\PY{k}{in}\PY{+w}{ }\PY{n}{r}\PY{p}{]}
\PY{n}{imag\PYZus{}fz}\PY{+w}{ }\PY{o}{=}\PY{+w}{ }\PY{p}{[}\PY{n}{imag}\PY{o}{.}\PY{p}{(}\PY{n}{fz}\PY{p}{(}\PY{n}{z}\PY{p}{)}\PY{p}{)}\PY{+w}{ }\PY{k}{for}\PY{+w}{ }\PY{n}{z}\PY{+w}{ }\PY{k}{in}\PY{+w}{ }\PY{n}{r}\PY{o}{*}\PY{l+m+mi}{1}\PY{n+nb}{im}\PY{p}{]}
\PY{n}{rad}\PY{+w}{ }\PY{o}{=}\PY{+w}{ }\PY{p}{[}\PY{n}{sqrt}\PY{p}{(}\PY{n}{x}\PY{o}{\PYZca{}}\PY{l+m+mi}{2}\PY{+w}{ }\PY{o}{+}\PY{+w}{ }\PY{n}{y}\PY{o}{\PYZca{}}\PY{l+m+mi}{2}\PY{p}{)}\PY{+w}{ }\PY{k}{for}\PY{+w}{ }\PY{p}{(}\PY{n}{x}\PY{p}{,}\PY{+w}{ }\PY{n}{y}\PY{p}{)}\PY{+w}{ }\PY{k}{in}\PY{+w}{ }\PY{n}{zip}\PY{p}{(}\PY{n}{real\PYZus{}fz}\PY{p}{,}\PY{+w}{ }\PY{n}{imag\PYZus{}fz}\PY{p}{)}\PY{p}{]}
\PY{n}{θ}\PY{+w}{ }\PY{o}{=}\PY{+w}{ }\PY{p}{[}\PY{n}{atan}\PY{p}{(}\PY{n}{y}\PY{p}{,}\PY{n}{x}\PY{p}{)}\PY{+w}{ }\PY{k}{for}\PY{+w}{ }\PY{p}{(}\PY{n}{x}\PY{p}{,}\PY{n}{y}\PY{p}{)}\PY{+w}{ }\PY{k}{in}\PY{+w}{ }\PY{n}{zip}\PY{p}{(}\PY{n}{real\PYZus{}fz}\PY{p}{,}\PY{n}{imag\PYZus{}fz}\PY{p}{)}\PY{p}{]}
\PY{n}{f}\PY{+w}{ }\PY{o}{=}\PY{+w}{ }\PY{n}{Figure}\PY{p}{(}\PY{n}{size}\PY{+w}{ }\PY{o}{=}\PY{+w}{ }\PY{p}{(}\PY{l+m+mi}{800}\PY{p}{,}\PY{+w}{ }\PY{l+m+mi}{400}\PY{p}{)}\PY{p}{)}

\PY{n}{real\PYZus{}cz}\PY{+w}{ }\PY{o}{=}\PY{+w}{ }\PY{p}{[}\PY{n}{real}\PY{o}{.}\PY{p}{(}\PY{n}{cosinez}\PY{p}{(}\PY{n}{z}\PY{p}{)}\PY{p}{)}\PY{+w}{ }\PY{k}{for}\PY{+w}{ }\PY{n}{z}\PY{+w}{ }\PY{k}{in}\PY{+w}{ }\PY{n}{r}\PY{p}{]}
\PY{n}{imag\PYZus{}cz}\PY{+w}{ }\PY{o}{=}\PY{+w}{ }\PY{p}{[}\PY{n}{imag}\PY{o}{.}\PY{p}{(}\PY{n}{cosinez}\PY{p}{(}\PY{n}{z}\PY{p}{)}\PY{p}{)}\PY{+w}{ }\PY{k}{for}\PY{+w}{ }\PY{n}{z}\PY{+w}{ }\PY{k}{in}\PY{+w}{ }\PY{n}{r}\PY{o}{*}\PY{l+m+mi}{1}\PY{n+nb}{im}\PY{p}{]}
\PY{n}{radcz}\PY{+w}{ }\PY{o}{=}\PY{+w}{ }\PY{p}{[}\PY{n}{sqrt}\PY{p}{(}\PY{n}{x}\PY{o}{\PYZca{}}\PY{l+m+mi}{2}\PY{+w}{ }\PY{o}{+}\PY{+w}{ }\PY{n}{y}\PY{o}{\PYZca{}}\PY{l+m+mi}{2}\PY{p}{)}\PY{+w}{ }\PY{k}{for}\PY{+w}{ }\PY{p}{(}\PY{n}{x}\PY{p}{,}\PY{+w}{ }\PY{n}{y}\PY{p}{)}\PY{+w}{ }\PY{k}{in}\PY{+w}{ }\PY{n}{zip}\PY{p}{(}\PY{n}{real\PYZus{}cz}\PY{p}{,}\PY{+w}{ }\PY{n}{imag\PYZus{}cz}\PY{p}{)}\PY{p}{]}
\PY{n}{θcz}\PY{+w}{ }\PY{o}{=}\PY{+w}{ }\PY{p}{[}\PY{n}{atan}\PY{p}{(}\PY{n}{y}\PY{p}{,}\PY{n}{x}\PY{p}{)}\PY{+w}{ }\PY{k}{for}\PY{+w}{ }\PY{p}{(}\PY{n}{x}\PY{p}{,}\PY{n}{y}\PY{p}{)}\PY{+w}{ }\PY{k}{in}\PY{+w}{ }\PY{n}{zip}\PY{p}{(}\PY{n}{real\PYZus{}cz}\PY{p}{,}\PY{n}{imag\PYZus{}cz}\PY{p}{)}\PY{p}{]}


\PY{n}{ax}\PY{+w}{ }\PY{o}{=}\PY{+w}{ }\PY{n}{PolarAxis}\PY{p}{(}\PY{n}{f}\PY{p}{[}\PY{l+m+mi}{1}\PY{p}{,}\PY{+w}{ }\PY{l+m+mi}{1}\PY{p}{]}\PY{p}{,}\PY{n}{title}\PY{+w}{ }\PY{o}{=}\PY{+w}{ }\PY{l+s+sa}{L}\PY{l+s}{\PYZdq{}\PYZdq{}\PYZdq{}}\PY{l+s}{f(z)=}\PY{l+s+se}{\PYZbs{}f}\PY{l+s}{rac\PYZob{}e\PYZca{}\PYZob{}i}\PY{l+s}{\PYZbs{}}\PY{l+s}{pi z\PYZcb{}\PYZhy{}e\PYZca{}\PYZob{}\PYZhy{}i}\PY{l+s}{\PYZbs{}}\PY{l+s}{pi z\PYZcb{}\PYZcb{}\PYZob{}2i\PYZcb{}}\PY{l+s}{\PYZdq{}\PYZdq{}\PYZdq{}}\PY{p}{,}\PY{n}{rticklabelsvisible}\PY{o}{=}\PY{n+nb}{true}\PY{p}{)}
\PY{n}{lineobject}\PY{+w}{ }\PY{o}{=}\PY{+w}{ }\PY{n}{scatter!}\PY{p}{(}\PY{n}{ax}\PY{p}{,}\PY{+w}{ }\PY{n}{rad}\PY{p}{,}\PY{+w}{ }\PY{n}{θ}\PY{p}{,}\PY{+w}{ }\PY{n}{color}\PY{+w}{ }\PY{o}{=}\PY{+w}{ }\PY{l+s+ss}{:orange}\PY{p}{)}

\PY{n}{ax}\PY{+w}{ }\PY{o}{=}\PY{+w}{ }\PY{n}{PolarAxis}\PY{p}{(}\PY{n}{f}\PY{p}{[}\PY{l+m+mi}{1}\PY{p}{,}\PY{+w}{ }\PY{l+m+mi}{2}\PY{p}{]}\PY{p}{,}\PY{n}{title}\PY{+w}{ }\PY{o}{=}\PY{+w}{ }\PY{l+s+sa}{L}\PY{l+s}{\PYZdq{}\PYZdq{}\PYZdq{}}\PY{l+s}{cos z\PYZca{}\PYZob{}1/2\PYZcb{}=}\PY{l+s}{\PYZbs{}}\PY{l+s}{sum\PYZus{}\PYZob{}n=0\PYZcb{}\PYZca{}\PYZob{}}\PY{l+s}{\PYZbs{}}\PY{l+s}{infty\PYZcb{}(\PYZhy{}1)\PYZca{}\PYZob{}n\PYZcb{}}\PY{l+s+se}{\PYZbs{}f}\PY{l+s}{rac\PYZob{}z\PYZca{}\PYZob{}n\PYZcb{}\PYZcb{}\PYZob{}(2n)!\PYZcb{}}\PY{l+s}{\PYZdq{}\PYZdq{}\PYZdq{}}\PY{p}{,}\PY{n}{rticklabelsvisible}\PY{o}{=}\PY{n+nb}{true}\PY{p}{)}
\PY{n}{lineobject}\PY{+w}{ }\PY{o}{=}\PY{+w}{ }\PY{n}{scatter!}\PY{p}{(}\PY{n}{ax}\PY{p}{,}\PY{+w}{ }\PY{n}{radcz}\PY{p}{,}\PY{+w}{ }\PY{n}{θcz}\PY{p}{,}\PY{+w}{ }\PY{n}{color}\PY{+w}{ }\PY{o}{=}\PY{+w}{ }\PY{l+s+ss}{:red}\PY{p}{)}
\PY{+w}{    }
\PY{n}{f}
\end{Verbatim}
\end{tcolorbox}
 
            
\prompt{Out}{outcolor}{4}{}
    
    \begin{center}
    \adjustimage{max size={0.9\linewidth}{0.9\paperheight}}{output_6_0.png}
    \end{center}
    { \hspace*{\fill} \\}
    

    \subsection{Infinite Products}\label{infinite-products}

Given a sequence \(\left\{a_{n}\right\}_{n=1}^{\infty}\) of complex
numbers, we say that the product
\(\large \prod_{n=1}^{\infty}(1+a_{n})\) converges of the limit
\[\huge \lim_{N\to\infty}\prod_{n=1}^{\infty}(1+a_{n})\] of the partial
products \textbf{exists}.

Suppose \(\left\{F_{n}\right\}\) is a sequence of holomorphic functions
on the open set \(\Omega\). If \(\exists c_{n} > 0\) such that
\(\sum c_{n} < \infty\) and
\(|F_{n}(z)-1| \le c_{n}\,\forall z\in\Omega\) then

\begin{itemize}
\tightlist
\item
  The product \(\prod_{n=1}^{\infty}F_{n}(z)\) converges uniformly in
  \(\Omega\) to a holomorphic function \(F(z)\)
\item
  If \(F_{n}(z)\) does not vanish \(\forall n\) then
\end{itemize}

\[\huge \dfrac{{F^{\prime}}(z)}{F(z)}=\sum_{n=1}^{\infty}\dfrac{{F^{\prime}}_{n}(z)}{F_{n}(z)}\]

    \subsection{Weierstrass Infinite
Products}\label{weierstrass-infinite-products}

Given any sequence \(\left\{a_{n}\right\}\) of complex numbers with
\(|a_{n}|\to\infty\) as \(n\to\infty\), there exists an entire function
\(f\) that vanishes \(\forall z=a_{n}\) and nowhere else. Any other such
entire function is of the form \(\large f(z)e^{g(z)}\), where \(g\) is
entire.

For each integer \(k\ge0\) we define \textbf{canonical factors} by

\[\Large E_{0}(z)=1-z\quad\text{and}\quad E_{k}(z)=(1-z)e^{z+\frac{z^{2}}{2}+\cdots+\frac{z^{k}}{k}},\quad\text{for}\quad k\ge1\]

The integer \(k\) is called the \textbf{degree} of canonical factors.

If \(\large|z|\le\frac{1}{2}\), then
\(\large|1-E_{k}(z)| \le c|z|^{k+1}\quad \exists c>0\)

Suppose that we are given a zero of order m at origin, and that
\(a_{1},a_{2},\cdots\) are all non-zero. Then we define
\textbf{Weierstrass Product} by

\[\Huge f(z)=e^{g(z)}\prod_{n=1}^{\infty}E_{n}(\frac{z}{a_{n}})\]

where \(g(z)\) is an entire function

    \subsection{Hadamard's factorization
theorem}\label{hadamards-factorization-theorem}

It is a refinement of Weierstrass's factorization theorem

\[\Huge f(z)=e^{g(z)}z^{m}\prod_{n=1}^{\infty}E_{n}(\frac{z}{a_{n}})\]

    \begin{tcolorbox}[breakable, size=fbox, boxrule=1pt, pad at break*=1mm,colback=cellbackground, colframe=cellborder]
\prompt{In}{incolor}{ }{\boxspacing}
\begin{Verbatim}[commandchars=\\\{\}]

\end{Verbatim}
\end{tcolorbox}

    \begin{tcolorbox}[breakable, size=fbox, boxrule=1pt, pad at break*=1mm,colback=cellbackground, colframe=cellborder]
\prompt{In}{incolor}{ }{\boxspacing}
\begin{Verbatim}[commandchars=\\\{\}]

\end{Verbatim}
\end{tcolorbox}


    % Add a bibliography block to the postdoc
    
    
    
\end{document}
