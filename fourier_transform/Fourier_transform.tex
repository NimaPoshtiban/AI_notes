\documentclass[11pt]{article}

    \usepackage[breakable]{tcolorbox}
    \usepackage{parskip} % Stop auto-indenting (to mimic markdown behaviour)
    

    % Basic figure setup, for now with no caption control since it's done
    % automatically by Pandoc (which extracts ![](path) syntax from Markdown).
    \usepackage{graphicx}
    % Keep aspect ratio if custom image width or height is specified
    \setkeys{Gin}{keepaspectratio}
    % Maintain compatibility with old templates. Remove in nbconvert 6.0
    \let\Oldincludegraphics\includegraphics
    % Ensure that by default, figures have no caption (until we provide a
    % proper Figure object with a Caption API and a way to capture that
    % in the conversion process - todo).
    \usepackage{caption}
    \DeclareCaptionFormat{nocaption}{}
    \captionsetup{format=nocaption,aboveskip=0pt,belowskip=0pt}

    \usepackage{float}
    \floatplacement{figure}{H} % forces figures to be placed at the correct location
    \usepackage{xcolor} % Allow colors to be defined
    \usepackage{enumerate} % Needed for markdown enumerations to work
    \usepackage{geometry} % Used to adjust the document margins
    \usepackage{amsmath} % Equations
    \usepackage{amssymb} % Equations
    \usepackage{textcomp} % defines textquotesingle
    % Hack from http://tex.stackexchange.com/a/47451/13684:
    \AtBeginDocument{%
        \def\PYZsq{\textquotesingle}% Upright quotes in Pygmentized code
    }
    \usepackage{upquote} % Upright quotes for verbatim code
    \usepackage{eurosym} % defines \euro

    \usepackage{iftex}
    \ifPDFTeX
        \usepackage[T1]{fontenc}
        \IfFileExists{alphabeta.sty}{
              \usepackage{alphabeta}
          }{
              \usepackage[mathletters]{ucs}
              \usepackage[utf8x]{inputenc}
          }
    \else
        \usepackage{fontspec}
        \usepackage{unicode-math}
    \fi

    \usepackage{fancyvrb} % verbatim replacement that allows latex
    \usepackage{grffile} % extends the file name processing of package graphics
                         % to support a larger range
    \makeatletter % fix for old versions of grffile with XeLaTeX
    \@ifpackagelater{grffile}{2019/11/01}
    {
      % Do nothing on new versions
    }
    {
      \def\Gread@@xetex#1{%
        \IfFileExists{"\Gin@base".bb}%
        {\Gread@eps{\Gin@base.bb}}%
        {\Gread@@xetex@aux#1}%
      }
    }
    \makeatother
    \usepackage[Export]{adjustbox} % Used to constrain images to a maximum size
    \adjustboxset{max size={0.9\linewidth}{0.9\paperheight}}

    % The hyperref package gives us a pdf with properly built
    % internal navigation ('pdf bookmarks' for the table of contents,
    % internal cross-reference links, web links for URLs, etc.)
    \usepackage{hyperref}
    % The default LaTeX title has an obnoxious amount of whitespace. By default,
    % titling removes some of it. It also provides customization options.
    \usepackage{titling}
    \usepackage{longtable} % longtable support required by pandoc >1.10
    \usepackage{booktabs}  % table support for pandoc > 1.12.2
    \usepackage{array}     % table support for pandoc >= 2.11.3
    \usepackage{calc}      % table minipage width calculation for pandoc >= 2.11.1
    \usepackage[inline]{enumitem} % IRkernel/repr support (it uses the enumerate* environment)
    \usepackage[normalem]{ulem} % ulem is needed to support strikethroughs (\sout)
                                % normalem makes italics be italics, not underlines
    \usepackage{soul}      % strikethrough (\st) support for pandoc >= 3.0.0
    \usepackage{mathrsfs}
    

    
    % Colors for the hyperref package
    \definecolor{urlcolor}{rgb}{0,.145,.698}
    \definecolor{linkcolor}{rgb}{.71,0.21,0.01}
    \definecolor{citecolor}{rgb}{.12,.54,.11}

    % ANSI colors
    \definecolor{ansi-black}{HTML}{3E424D}
    \definecolor{ansi-black-intense}{HTML}{282C36}
    \definecolor{ansi-red}{HTML}{E75C58}
    \definecolor{ansi-red-intense}{HTML}{B22B31}
    \definecolor{ansi-green}{HTML}{00A250}
    \definecolor{ansi-green-intense}{HTML}{007427}
    \definecolor{ansi-yellow}{HTML}{DDB62B}
    \definecolor{ansi-yellow-intense}{HTML}{B27D12}
    \definecolor{ansi-blue}{HTML}{208FFB}
    \definecolor{ansi-blue-intense}{HTML}{0065CA}
    \definecolor{ansi-magenta}{HTML}{D160C4}
    \definecolor{ansi-magenta-intense}{HTML}{A03196}
    \definecolor{ansi-cyan}{HTML}{60C6C8}
    \definecolor{ansi-cyan-intense}{HTML}{258F8F}
    \definecolor{ansi-white}{HTML}{C5C1B4}
    \definecolor{ansi-white-intense}{HTML}{A1A6B2}
    \definecolor{ansi-default-inverse-fg}{HTML}{FFFFFF}
    \definecolor{ansi-default-inverse-bg}{HTML}{000000}

    % common color for the border for error outputs.
    \definecolor{outerrorbackground}{HTML}{FFDFDF}

    % commands and environments needed by pandoc snippets
    % extracted from the output of `pandoc -s`
    \providecommand{\tightlist}{%
      \setlength{\itemsep}{0pt}\setlength{\parskip}{0pt}}
    \DefineVerbatimEnvironment{Highlighting}{Verbatim}{commandchars=\\\{\}}
    % Add ',fontsize=\small' for more characters per line
    \newenvironment{Shaded}{}{}
    \newcommand{\KeywordTok}[1]{\textcolor[rgb]{0.00,0.44,0.13}{\textbf{{#1}}}}
    \newcommand{\DataTypeTok}[1]{\textcolor[rgb]{0.56,0.13,0.00}{{#1}}}
    \newcommand{\DecValTok}[1]{\textcolor[rgb]{0.25,0.63,0.44}{{#1}}}
    \newcommand{\BaseNTok}[1]{\textcolor[rgb]{0.25,0.63,0.44}{{#1}}}
    \newcommand{\FloatTok}[1]{\textcolor[rgb]{0.25,0.63,0.44}{{#1}}}
    \newcommand{\CharTok}[1]{\textcolor[rgb]{0.25,0.44,0.63}{{#1}}}
    \newcommand{\StringTok}[1]{\textcolor[rgb]{0.25,0.44,0.63}{{#1}}}
    \newcommand{\CommentTok}[1]{\textcolor[rgb]{0.38,0.63,0.69}{\textit{{#1}}}}
    \newcommand{\OtherTok}[1]{\textcolor[rgb]{0.00,0.44,0.13}{{#1}}}
    \newcommand{\AlertTok}[1]{\textcolor[rgb]{1.00,0.00,0.00}{\textbf{{#1}}}}
    \newcommand{\FunctionTok}[1]{\textcolor[rgb]{0.02,0.16,0.49}{{#1}}}
    \newcommand{\RegionMarkerTok}[1]{{#1}}
    \newcommand{\ErrorTok}[1]{\textcolor[rgb]{1.00,0.00,0.00}{\textbf{{#1}}}}
    \newcommand{\NormalTok}[1]{{#1}}

    % Additional commands for more recent versions of Pandoc
    \newcommand{\ConstantTok}[1]{\textcolor[rgb]{0.53,0.00,0.00}{{#1}}}
    \newcommand{\SpecialCharTok}[1]{\textcolor[rgb]{0.25,0.44,0.63}{{#1}}}
    \newcommand{\VerbatimStringTok}[1]{\textcolor[rgb]{0.25,0.44,0.63}{{#1}}}
    \newcommand{\SpecialStringTok}[1]{\textcolor[rgb]{0.73,0.40,0.53}{{#1}}}
    \newcommand{\ImportTok}[1]{{#1}}
    \newcommand{\DocumentationTok}[1]{\textcolor[rgb]{0.73,0.13,0.13}{\textit{{#1}}}}
    \newcommand{\AnnotationTok}[1]{\textcolor[rgb]{0.38,0.63,0.69}{\textbf{\textit{{#1}}}}}
    \newcommand{\CommentVarTok}[1]{\textcolor[rgb]{0.38,0.63,0.69}{\textbf{\textit{{#1}}}}}
    \newcommand{\VariableTok}[1]{\textcolor[rgb]{0.10,0.09,0.49}{{#1}}}
    \newcommand{\ControlFlowTok}[1]{\textcolor[rgb]{0.00,0.44,0.13}{\textbf{{#1}}}}
    \newcommand{\OperatorTok}[1]{\textcolor[rgb]{0.40,0.40,0.40}{{#1}}}
    \newcommand{\BuiltInTok}[1]{{#1}}
    \newcommand{\ExtensionTok}[1]{{#1}}
    \newcommand{\PreprocessorTok}[1]{\textcolor[rgb]{0.74,0.48,0.00}{{#1}}}
    \newcommand{\AttributeTok}[1]{\textcolor[rgb]{0.49,0.56,0.16}{{#1}}}
    \newcommand{\InformationTok}[1]{\textcolor[rgb]{0.38,0.63,0.69}{\textbf{\textit{{#1}}}}}
    \newcommand{\WarningTok}[1]{\textcolor[rgb]{0.38,0.63,0.69}{\textbf{\textit{{#1}}}}}


    % Define a nice break command that doesn't care if a line doesn't already
    % exist.
    \def\br{\hspace*{\fill} \\* }
    % Math Jax compatibility definitions
    \def\gt{>}
    \def\lt{<}
    \let\Oldtex\TeX
    \let\Oldlatex\LaTeX
    \renewcommand{\TeX}{\textrm{\Oldtex}}
    \renewcommand{\LaTeX}{\textrm{\Oldlatex}}
    % Document parameters
    % Document title
    \title{Fourier\_transform}
    
    
    
    
    
    
    
% Pygments definitions
\makeatletter
\def\PY@reset{\let\PY@it=\relax \let\PY@bf=\relax%
    \let\PY@ul=\relax \let\PY@tc=\relax%
    \let\PY@bc=\relax \let\PY@ff=\relax}
\def\PY@tok#1{\csname PY@tok@#1\endcsname}
\def\PY@toks#1+{\ifx\relax#1\empty\else%
    \PY@tok{#1}\expandafter\PY@toks\fi}
\def\PY@do#1{\PY@bc{\PY@tc{\PY@ul{%
    \PY@it{\PY@bf{\PY@ff{#1}}}}}}}
\def\PY#1#2{\PY@reset\PY@toks#1+\relax+\PY@do{#2}}

\@namedef{PY@tok@w}{\def\PY@tc##1{\textcolor[rgb]{0.73,0.73,0.73}{##1}}}
\@namedef{PY@tok@c}{\let\PY@it=\textit\def\PY@tc##1{\textcolor[rgb]{0.24,0.48,0.48}{##1}}}
\@namedef{PY@tok@cp}{\def\PY@tc##1{\textcolor[rgb]{0.61,0.40,0.00}{##1}}}
\@namedef{PY@tok@k}{\let\PY@bf=\textbf\def\PY@tc##1{\textcolor[rgb]{0.00,0.50,0.00}{##1}}}
\@namedef{PY@tok@kp}{\def\PY@tc##1{\textcolor[rgb]{0.00,0.50,0.00}{##1}}}
\@namedef{PY@tok@kt}{\def\PY@tc##1{\textcolor[rgb]{0.69,0.00,0.25}{##1}}}
\@namedef{PY@tok@o}{\def\PY@tc##1{\textcolor[rgb]{0.40,0.40,0.40}{##1}}}
\@namedef{PY@tok@ow}{\let\PY@bf=\textbf\def\PY@tc##1{\textcolor[rgb]{0.67,0.13,1.00}{##1}}}
\@namedef{PY@tok@nb}{\def\PY@tc##1{\textcolor[rgb]{0.00,0.50,0.00}{##1}}}
\@namedef{PY@tok@nf}{\def\PY@tc##1{\textcolor[rgb]{0.00,0.00,1.00}{##1}}}
\@namedef{PY@tok@nc}{\let\PY@bf=\textbf\def\PY@tc##1{\textcolor[rgb]{0.00,0.00,1.00}{##1}}}
\@namedef{PY@tok@nn}{\let\PY@bf=\textbf\def\PY@tc##1{\textcolor[rgb]{0.00,0.00,1.00}{##1}}}
\@namedef{PY@tok@ne}{\let\PY@bf=\textbf\def\PY@tc##1{\textcolor[rgb]{0.80,0.25,0.22}{##1}}}
\@namedef{PY@tok@nv}{\def\PY@tc##1{\textcolor[rgb]{0.10,0.09,0.49}{##1}}}
\@namedef{PY@tok@no}{\def\PY@tc##1{\textcolor[rgb]{0.53,0.00,0.00}{##1}}}
\@namedef{PY@tok@nl}{\def\PY@tc##1{\textcolor[rgb]{0.46,0.46,0.00}{##1}}}
\@namedef{PY@tok@ni}{\let\PY@bf=\textbf\def\PY@tc##1{\textcolor[rgb]{0.44,0.44,0.44}{##1}}}
\@namedef{PY@tok@na}{\def\PY@tc##1{\textcolor[rgb]{0.41,0.47,0.13}{##1}}}
\@namedef{PY@tok@nt}{\let\PY@bf=\textbf\def\PY@tc##1{\textcolor[rgb]{0.00,0.50,0.00}{##1}}}
\@namedef{PY@tok@nd}{\def\PY@tc##1{\textcolor[rgb]{0.67,0.13,1.00}{##1}}}
\@namedef{PY@tok@s}{\def\PY@tc##1{\textcolor[rgb]{0.73,0.13,0.13}{##1}}}
\@namedef{PY@tok@sd}{\let\PY@it=\textit\def\PY@tc##1{\textcolor[rgb]{0.73,0.13,0.13}{##1}}}
\@namedef{PY@tok@si}{\let\PY@bf=\textbf\def\PY@tc##1{\textcolor[rgb]{0.64,0.35,0.47}{##1}}}
\@namedef{PY@tok@se}{\let\PY@bf=\textbf\def\PY@tc##1{\textcolor[rgb]{0.67,0.36,0.12}{##1}}}
\@namedef{PY@tok@sr}{\def\PY@tc##1{\textcolor[rgb]{0.64,0.35,0.47}{##1}}}
\@namedef{PY@tok@ss}{\def\PY@tc##1{\textcolor[rgb]{0.10,0.09,0.49}{##1}}}
\@namedef{PY@tok@sx}{\def\PY@tc##1{\textcolor[rgb]{0.00,0.50,0.00}{##1}}}
\@namedef{PY@tok@m}{\def\PY@tc##1{\textcolor[rgb]{0.40,0.40,0.40}{##1}}}
\@namedef{PY@tok@gh}{\let\PY@bf=\textbf\def\PY@tc##1{\textcolor[rgb]{0.00,0.00,0.50}{##1}}}
\@namedef{PY@tok@gu}{\let\PY@bf=\textbf\def\PY@tc##1{\textcolor[rgb]{0.50,0.00,0.50}{##1}}}
\@namedef{PY@tok@gd}{\def\PY@tc##1{\textcolor[rgb]{0.63,0.00,0.00}{##1}}}
\@namedef{PY@tok@gi}{\def\PY@tc##1{\textcolor[rgb]{0.00,0.52,0.00}{##1}}}
\@namedef{PY@tok@gr}{\def\PY@tc##1{\textcolor[rgb]{0.89,0.00,0.00}{##1}}}
\@namedef{PY@tok@ge}{\let\PY@it=\textit}
\@namedef{PY@tok@gs}{\let\PY@bf=\textbf}
\@namedef{PY@tok@ges}{\let\PY@bf=\textbf\let\PY@it=\textit}
\@namedef{PY@tok@gp}{\let\PY@bf=\textbf\def\PY@tc##1{\textcolor[rgb]{0.00,0.00,0.50}{##1}}}
\@namedef{PY@tok@go}{\def\PY@tc##1{\textcolor[rgb]{0.44,0.44,0.44}{##1}}}
\@namedef{PY@tok@gt}{\def\PY@tc##1{\textcolor[rgb]{0.00,0.27,0.87}{##1}}}
\@namedef{PY@tok@err}{\def\PY@bc##1{{\setlength{\fboxsep}{\string -\fboxrule}\fcolorbox[rgb]{1.00,0.00,0.00}{1,1,1}{\strut ##1}}}}
\@namedef{PY@tok@kc}{\let\PY@bf=\textbf\def\PY@tc##1{\textcolor[rgb]{0.00,0.50,0.00}{##1}}}
\@namedef{PY@tok@kd}{\let\PY@bf=\textbf\def\PY@tc##1{\textcolor[rgb]{0.00,0.50,0.00}{##1}}}
\@namedef{PY@tok@kn}{\let\PY@bf=\textbf\def\PY@tc##1{\textcolor[rgb]{0.00,0.50,0.00}{##1}}}
\@namedef{PY@tok@kr}{\let\PY@bf=\textbf\def\PY@tc##1{\textcolor[rgb]{0.00,0.50,0.00}{##1}}}
\@namedef{PY@tok@bp}{\def\PY@tc##1{\textcolor[rgb]{0.00,0.50,0.00}{##1}}}
\@namedef{PY@tok@fm}{\def\PY@tc##1{\textcolor[rgb]{0.00,0.00,1.00}{##1}}}
\@namedef{PY@tok@vc}{\def\PY@tc##1{\textcolor[rgb]{0.10,0.09,0.49}{##1}}}
\@namedef{PY@tok@vg}{\def\PY@tc##1{\textcolor[rgb]{0.10,0.09,0.49}{##1}}}
\@namedef{PY@tok@vi}{\def\PY@tc##1{\textcolor[rgb]{0.10,0.09,0.49}{##1}}}
\@namedef{PY@tok@vm}{\def\PY@tc##1{\textcolor[rgb]{0.10,0.09,0.49}{##1}}}
\@namedef{PY@tok@sa}{\def\PY@tc##1{\textcolor[rgb]{0.73,0.13,0.13}{##1}}}
\@namedef{PY@tok@sb}{\def\PY@tc##1{\textcolor[rgb]{0.73,0.13,0.13}{##1}}}
\@namedef{PY@tok@sc}{\def\PY@tc##1{\textcolor[rgb]{0.73,0.13,0.13}{##1}}}
\@namedef{PY@tok@dl}{\def\PY@tc##1{\textcolor[rgb]{0.73,0.13,0.13}{##1}}}
\@namedef{PY@tok@s2}{\def\PY@tc##1{\textcolor[rgb]{0.73,0.13,0.13}{##1}}}
\@namedef{PY@tok@sh}{\def\PY@tc##1{\textcolor[rgb]{0.73,0.13,0.13}{##1}}}
\@namedef{PY@tok@s1}{\def\PY@tc##1{\textcolor[rgb]{0.73,0.13,0.13}{##1}}}
\@namedef{PY@tok@mb}{\def\PY@tc##1{\textcolor[rgb]{0.40,0.40,0.40}{##1}}}
\@namedef{PY@tok@mf}{\def\PY@tc##1{\textcolor[rgb]{0.40,0.40,0.40}{##1}}}
\@namedef{PY@tok@mh}{\def\PY@tc##1{\textcolor[rgb]{0.40,0.40,0.40}{##1}}}
\@namedef{PY@tok@mi}{\def\PY@tc##1{\textcolor[rgb]{0.40,0.40,0.40}{##1}}}
\@namedef{PY@tok@il}{\def\PY@tc##1{\textcolor[rgb]{0.40,0.40,0.40}{##1}}}
\@namedef{PY@tok@mo}{\def\PY@tc##1{\textcolor[rgb]{0.40,0.40,0.40}{##1}}}
\@namedef{PY@tok@ch}{\let\PY@it=\textit\def\PY@tc##1{\textcolor[rgb]{0.24,0.48,0.48}{##1}}}
\@namedef{PY@tok@cm}{\let\PY@it=\textit\def\PY@tc##1{\textcolor[rgb]{0.24,0.48,0.48}{##1}}}
\@namedef{PY@tok@cpf}{\let\PY@it=\textit\def\PY@tc##1{\textcolor[rgb]{0.24,0.48,0.48}{##1}}}
\@namedef{PY@tok@c1}{\let\PY@it=\textit\def\PY@tc##1{\textcolor[rgb]{0.24,0.48,0.48}{##1}}}
\@namedef{PY@tok@cs}{\let\PY@it=\textit\def\PY@tc##1{\textcolor[rgb]{0.24,0.48,0.48}{##1}}}

\def\PYZbs{\char`\\}
\def\PYZus{\char`\_}
\def\PYZob{\char`\{}
\def\PYZcb{\char`\}}
\def\PYZca{\char`\^}
\def\PYZam{\char`\&}
\def\PYZlt{\char`\<}
\def\PYZgt{\char`\>}
\def\PYZsh{\char`\#}
\def\PYZpc{\char`\%}
\def\PYZdl{\char`\$}
\def\PYZhy{\char`\-}
\def\PYZsq{\char`\'}
\def\PYZdq{\char`\"}
\def\PYZti{\char`\~}
% for compatibility with earlier versions
\def\PYZat{@}
\def\PYZlb{[}
\def\PYZrb{]}
\makeatother


    % For linebreaks inside Verbatim environment from package fancyvrb.
    \makeatletter
        \newbox\Wrappedcontinuationbox
        \newbox\Wrappedvisiblespacebox
        \newcommand*\Wrappedvisiblespace {\textcolor{red}{\textvisiblespace}}
        \newcommand*\Wrappedcontinuationsymbol {\textcolor{red}{\llap{\tiny$\m@th\hookrightarrow$}}}
        \newcommand*\Wrappedcontinuationindent {3ex }
        \newcommand*\Wrappedafterbreak {\kern\Wrappedcontinuationindent\copy\Wrappedcontinuationbox}
        % Take advantage of the already applied Pygments mark-up to insert
        % potential linebreaks for TeX processing.
        %        {, <, #, %, $, ' and ": go to next line.
        %        _, }, ^, &, >, - and ~: stay at end of broken line.
        % Use of \textquotesingle for straight quote.
        \newcommand*\Wrappedbreaksatspecials {%
            \def\PYGZus{\discretionary{\char`\_}{\Wrappedafterbreak}{\char`\_}}%
            \def\PYGZob{\discretionary{}{\Wrappedafterbreak\char`\{}{\char`\{}}%
            \def\PYGZcb{\discretionary{\char`\}}{\Wrappedafterbreak}{\char`\}}}%
            \def\PYGZca{\discretionary{\char`\^}{\Wrappedafterbreak}{\char`\^}}%
            \def\PYGZam{\discretionary{\char`\&}{\Wrappedafterbreak}{\char`\&}}%
            \def\PYGZlt{\discretionary{}{\Wrappedafterbreak\char`\<}{\char`\<}}%
            \def\PYGZgt{\discretionary{\char`\>}{\Wrappedafterbreak}{\char`\>}}%
            \def\PYGZsh{\discretionary{}{\Wrappedafterbreak\char`\#}{\char`\#}}%
            \def\PYGZpc{\discretionary{}{\Wrappedafterbreak\char`\%}{\char`\%}}%
            \def\PYGZdl{\discretionary{}{\Wrappedafterbreak\char`\$}{\char`\$}}%
            \def\PYGZhy{\discretionary{\char`\-}{\Wrappedafterbreak}{\char`\-}}%
            \def\PYGZsq{\discretionary{}{\Wrappedafterbreak\textquotesingle}{\textquotesingle}}%
            \def\PYGZdq{\discretionary{}{\Wrappedafterbreak\char`\"}{\char`\"}}%
            \def\PYGZti{\discretionary{\char`\~}{\Wrappedafterbreak}{\char`\~}}%
        }
        % Some characters . , ; ? ! / are not pygmentized.
        % This macro makes them "active" and they will insert potential linebreaks
        \newcommand*\Wrappedbreaksatpunct {%
            \lccode`\~`\.\lowercase{\def~}{\discretionary{\hbox{\char`\.}}{\Wrappedafterbreak}{\hbox{\char`\.}}}%
            \lccode`\~`\,\lowercase{\def~}{\discretionary{\hbox{\char`\,}}{\Wrappedafterbreak}{\hbox{\char`\,}}}%
            \lccode`\~`\;\lowercase{\def~}{\discretionary{\hbox{\char`\;}}{\Wrappedafterbreak}{\hbox{\char`\;}}}%
            \lccode`\~`\:\lowercase{\def~}{\discretionary{\hbox{\char`\:}}{\Wrappedafterbreak}{\hbox{\char`\:}}}%
            \lccode`\~`\?\lowercase{\def~}{\discretionary{\hbox{\char`\?}}{\Wrappedafterbreak}{\hbox{\char`\?}}}%
            \lccode`\~`\!\lowercase{\def~}{\discretionary{\hbox{\char`\!}}{\Wrappedafterbreak}{\hbox{\char`\!}}}%
            \lccode`\~`\/\lowercase{\def~}{\discretionary{\hbox{\char`\/}}{\Wrappedafterbreak}{\hbox{\char`\/}}}%
            \catcode`\.\active
            \catcode`\,\active
            \catcode`\;\active
            \catcode`\:\active
            \catcode`\?\active
            \catcode`\!\active
            \catcode`\/\active
            \lccode`\~`\~
        }
    \makeatother

    \let\OriginalVerbatim=\Verbatim
    \makeatletter
    \renewcommand{\Verbatim}[1][1]{%
        %\parskip\z@skip
        \sbox\Wrappedcontinuationbox {\Wrappedcontinuationsymbol}%
        \sbox\Wrappedvisiblespacebox {\FV@SetupFont\Wrappedvisiblespace}%
        \def\FancyVerbFormatLine ##1{\hsize\linewidth
            \vtop{\raggedright\hyphenpenalty\z@\exhyphenpenalty\z@
                \doublehyphendemerits\z@\finalhyphendemerits\z@
                \strut ##1\strut}%
        }%
        % If the linebreak is at a space, the latter will be displayed as visible
        % space at end of first line, and a continuation symbol starts next line.
        % Stretch/shrink are however usually zero for typewriter font.
        \def\FV@Space {%
            \nobreak\hskip\z@ plus\fontdimen3\font minus\fontdimen4\font
            \discretionary{\copy\Wrappedvisiblespacebox}{\Wrappedafterbreak}
            {\kern\fontdimen2\font}%
        }%

        % Allow breaks at special characters using \PYG... macros.
        \Wrappedbreaksatspecials
        % Breaks at punctuation characters . , ; ? ! and / need catcode=\active
        \OriginalVerbatim[#1,codes*=\Wrappedbreaksatpunct]%
    }
    \makeatother

    % Exact colors from NB
    \definecolor{incolor}{HTML}{303F9F}
    \definecolor{outcolor}{HTML}{D84315}
    \definecolor{cellborder}{HTML}{CFCFCF}
    \definecolor{cellbackground}{HTML}{F7F7F7}

    % prompt
    \makeatletter
    \newcommand{\boxspacing}{\kern\kvtcb@left@rule\kern\kvtcb@boxsep}
    \makeatother
    \newcommand{\prompt}[4]{
        {\ttfamily\llap{{\color{#2}[#3]:\hspace{3pt}#4}}\vspace{-\baselineskip}}
    }
    

    
    % Prevent overflowing lines due to hard-to-break entities
    \sloppy
    % Setup hyperref package
    \hypersetup{
      breaklinks=true,  % so long urls are correctly broken across lines
      colorlinks=true,
      urlcolor=urlcolor,
      linkcolor=linkcolor,
      citecolor=citecolor,
      }
    % Slightly bigger margins than the latex defaults
    
    \geometry{verbose,tmargin=1in,bmargin=1in,lmargin=1in,rmargin=1in}
    
    

\begin{document}
    
    \maketitle
    
    

    
    \begin{tcolorbox}[breakable, size=fbox, boxrule=1pt, pad at break*=1mm,colback=cellbackground, colframe=cellborder]
\prompt{In}{incolor}{88}{\boxspacing}
\begin{Verbatim}[commandchars=\\\{\}]
\PY{k}{import}\PY{+w}{ }\PY{n}{Pkg}
\PY{k}{import}\PY{+w}{ }\PY{n}{GLMakie}
\PY{k}{import}\PY{+w}{ }\PY{n}{FFTW}
\PY{k}{using}\PY{+w}{ }\PY{n}{LaTeXStrings}
\end{Verbatim}
\end{tcolorbox}

    \section{\texorpdfstring{The Fourier Transform
\(\mathscr{F}\)}{The Fourier Transform \textbackslash mathscr\{F\}}}\label{the-fourier-transform-mathscrf}

    \subsection{Reminders}\label{reminders}

if \(f\) is a function on \(\mathbb{R}\) that satisfies appropriate
regularity and decay conditions, then its Fourier transform
\(\mathscr{F}\) is defined by:

\[\large\hat{f}(\xi)=\int_{-\infty}^{+\infty}f(x)e^{-2\pi ix\xi}\,dx\quad \xi\in\mathbb{R}\]

and its counter part, the Fourier inverse formula, holds:

\[\large f(x)=\int_{-\infty}^{+\infty}\hat{f}(\xi)e^{2\pi ix\xi}\,d\xi\quad x\in\mathbb{R}\]

If there be a holomorphic extension of \(f\) to \(\mathbb{C}\), in a
bounded interval \([-M,M]\) that satisfies the growth condition:

\[\large|f(z)|\le Ae^{2\pi M|z|}\quad \exists A>0\] \#\#\#\# Moderate
decrease

function \(f\) is of \textbf{moderate decrease} if \(f\) is continous
and \(\exists A >0\) so that
\(|f(x)| \le \dfrac{A}{1+x^{2}}\quad \forall x\in\mathbb{R}\). A more
restrictive condition is that \(f\in\mathbf{S}\) where \(\mathbf{S}\) is
the \textbf{Shwart sphere} of testing functions

    \subsection{\texorpdfstring{The class
\(\Im\)}{The class \textbackslash Im}}\label{the-class-im}

For each a\textgreater0 we denote by \(\Im_{a}\) the class of all
functions f that satisfies the following two conditions:

\begin{enumerate}
\def\labelenumi{\arabic{enumi})}
\tightlist
\item
  The function \(f\) is holomorphic in the horizontal strip
  \[S_{a}=\left\{z\in\mathbb{C}\,:\,|Im(z)| < a\right\}\]
\item
  There exists a constant \(A>0\) such that
  \[|f(x+iy)|\le\dfrac{A}{1+x^{2}}\quad \forall x\in\mathbb{R},\quad |y|<a\]
\end{enumerate}

\(\Im_{a}\) consists of those holomorphic functions on
\(\mathbf{S}_{a}\) that are of \textbf{moderate decay} on each
horizontal line \(Im(z)=y\),uniformely in \(-a<y<a\)

    \begin{tcolorbox}[breakable, size=fbox, boxrule=1pt, pad at break*=1mm,colback=cellbackground, colframe=cellborder]
\prompt{In}{incolor}{89}{\boxspacing}
\begin{Verbatim}[commandchars=\\\{\}]
\PY{k}{begin}
\PY{+w}{    }\PY{n}{fig}\PY{+w}{ }\PY{o}{=}\PY{+w}{ }\PY{n}{GLMakie}\PY{o}{.}\PY{n}{Figure}\PY{p}{(}\PY{n}{size}\PY{+w}{ }\PY{o}{=}\PY{+w}{ }\PY{p}{(}\PY{l+m+mi}{600}\PY{p}{,}\PY{+w}{ }\PY{l+m+mi}{600}\PY{p}{)}\PY{p}{)}
\PY{+w}{    }\PY{n}{ax}\PY{+w}{ }\PY{o}{=}\PY{+w}{ }\PY{n}{Axis}\PY{p}{(}\PY{n}{fig}\PY{p}{[}\PY{l+m+mi}{1}\PY{p}{,}\PY{+w}{ }\PY{l+m+mi}{1}\PY{p}{]}\PY{p}{,}\PY{+w}{ }\PY{n}{limits}\PY{+w}{ }\PY{o}{=}\PY{+w}{ }\PY{p}{(}\PY{o}{\PYZhy{}}\PY{l+m+mi}{2}\PY{p}{,}\PY{+w}{ }\PY{l+m+mi}{2}\PY{p}{,}\PY{+w}{ }\PY{o}{\PYZhy{}}\PY{l+m+mi}{2}\PY{p}{,}\PY{+w}{ }\PY{l+m+mi}{2}\PY{p}{)}\PY{p}{,}\PY{+w}{ }\PY{n}{title}\PY{+w}{ }\PY{o}{=}\PY{+w}{ }\PY{l+s}{\PYZdq{}}\PY{l+s}{Horizontal Strip in Complex Plane}\PY{l+s}{\PYZdq{}}\PY{p}{)}
\PY{n}{a}\PY{+w}{ }\PY{o}{=}\PY{+w}{ }\PY{o}{\PYZhy{}}\PY{l+m+mi}{1}
\PY{n}{b}\PY{+w}{ }\PY{o}{=}\PY{+w}{ }\PY{l+m+mi}{1}
\PY{+w}{    }
\PY{+w}{    }\PY{n}{polygon\PYZus{}vertices}\PY{+w}{ }\PY{o}{=}\PY{+w}{ }\PY{n}{GLMakie}\PY{o}{.}\PY{n}{Point2f0}\PY{o}{.}\PY{p}{(}\PY{p}{[}
\PY{+w}{        }\PY{p}{(}\PY{o}{\PYZhy{}}\PY{l+m+mi}{2}\PY{p}{,}\PY{+w}{ }\PY{n}{a}\PY{p}{)}\PY{p}{,}\PY{+w}{ }\PY{p}{(}\PY{l+m+mi}{2}\PY{p}{,}\PY{+w}{ }\PY{n}{a}\PY{p}{)}\PY{p}{,}\PY{+w}{ }\PY{p}{(}\PY{l+m+mi}{2}\PY{p}{,}\PY{+w}{ }\PY{n}{b}\PY{p}{)}\PY{p}{,}\PY{+w}{ }\PY{p}{(}\PY{o}{\PYZhy{}}\PY{l+m+mi}{2}\PY{p}{,}\PY{+w}{ }\PY{n}{b}\PY{p}{)}
\PY{+w}{    }\PY{p}{]}\PY{p}{)}

\PY{+w}{    }\PY{c}{\PYZsh{} Draw the horizontal strip as a polygon}
\PY{+w}{    }\PY{n}{GLMakie}\PY{o}{.}\PY{n}{poly!}\PY{p}{(}\PY{n}{ax}\PY{p}{,}\PY{+w}{ }\PY{n}{polygon\PYZus{}vertices}\PY{p}{,}\PY{+w}{ }\PY{n}{color}\PY{+w}{ }\PY{o}{=}\PY{+w}{ }\PY{p}{(}\PY{l+s+ss}{:grey}\PY{p}{,}\PY{+w}{ }\PY{l+m+mf}{0.5}\PY{p}{)}\PY{p}{)}

\PY{+w}{    }\PY{c}{\PYZsh{} Draw the boundaries}
\PY{+w}{    }\PY{n}{GLMakie}\PY{o}{.}\PY{n}{lines!}\PY{p}{(}\PY{n}{ax}\PY{p}{,}\PY{+w}{ }\PY{p}{[}\PY{o}{\PYZhy{}}\PY{l+m+mi}{2}\PY{p}{,}\PY{+w}{ }\PY{l+m+mi}{2}\PY{p}{]}\PY{p}{,}\PY{+w}{ }\PY{p}{[}\PY{n}{a}\PY{p}{,}\PY{+w}{ }\PY{n}{a}\PY{p}{]}\PY{p}{,}\PY{+w}{ }\PY{n}{color}\PY{+w}{ }\PY{o}{=}\PY{+w}{ }\PY{l+s+ss}{:red}\PY{p}{)}
\PY{+w}{    }\PY{n}{GLMakie}\PY{o}{.}\PY{n}{lines!}\PY{p}{(}\PY{n}{ax}\PY{p}{,}\PY{+w}{ }\PY{p}{[}\PY{o}{\PYZhy{}}\PY{l+m+mi}{2}\PY{p}{,}\PY{+w}{ }\PY{l+m+mi}{2}\PY{p}{]}\PY{p}{,}\PY{+w}{ }\PY{p}{[}\PY{n}{b}\PY{p}{,}\PY{+w}{ }\PY{n}{b}\PY{p}{]}\PY{p}{,}\PY{+w}{ }\PY{n}{color}\PY{+w}{ }\PY{o}{=}\PY{+w}{ }\PY{l+s+ss}{:red}\PY{p}{)}

\PY{+w}{    }\PY{n}{fig}
\PY{k}{end}
\end{Verbatim}
\end{tcolorbox}
 
            
\prompt{Out}{outcolor}{89}{}
    
    \begin{center}
    \adjustimage{max size={0.9\linewidth}{0.9\paperheight}}{output_4_0.png}
    \end{center}
    { \hspace*{\fill} \\}
    

    \textbf{If \(f\in\Im_{a}\), then for every \(n\), the \(n^{th}\)
derivative of \(f\) belongs to \(\Im_{b}\,\,\forall\,\,0<b<a\)}

Examples: \[\large f(z)=e^{-\pi z^{2}}\quad \forall a\in \Im_{a}\]

    \begin{tcolorbox}[breakable, size=fbox, boxrule=1pt, pad at break*=1mm,colback=cellbackground, colframe=cellborder]
\prompt{In}{incolor}{90}{\boxspacing}
\begin{Verbatim}[commandchars=\\\{\}]
\PY{k}{begin}
\PY{k}{using}\PY{+w}{ }\PY{n}{GLMakie}
\PY{k}{using}\PY{+w}{ }\PY{n}{FFTW}
\PY{k}{function}\PY{+w}{ }\PY{n}{f}\PY{p}{(}\PY{n}{z}\PY{p}{)}
\PY{+w}{    }\PY{k}{return}\PY{+w}{ }\PY{n}{exp}\PY{p}{(}\PY{o}{\PYZhy{}}\PY{n+nb}{π}\PY{+w}{ }\PY{o}{*}\PY{+w}{ }\PY{n}{z}\PY{o}{\PYZca{}}\PY{l+m+mi}{2}\PY{p}{)}
\PY{k}{end}

\PY{c}{\PYZsh{} Define the range for z and the sample points}
\PY{n}{z}\PY{+w}{ }\PY{o}{=}\PY{+w}{ }\PY{k+kt}{LinRange}\PY{p}{(}\PY{o}{\PYZhy{}}\PY{l+m+mi}{10}\PY{p}{,}\PY{+w}{ }\PY{l+m+mi}{10}\PY{p}{,}\PY{+w}{ }\PY{l+m+mi}{1024}\PY{p}{)}
\PY{n}{f\PYZus{}values}\PY{+w}{ }\PY{o}{=}\PY{+w}{ }\PY{n}{f}\PY{o}{.}\PY{p}{(}\PY{n}{z}\PY{p}{)}

\PY{c}{\PYZsh{} Compute the Fourier transform using FFT}
\PY{n}{F\PYZus{}transform}\PY{+w}{ }\PY{o}{=}\PY{+w}{ }\PY{n}{fftshift}\PY{p}{(}\PY{n}{fft}\PY{p}{(}\PY{n}{f\PYZus{}values}\PY{p}{)}\PY{p}{)}

\PY{c}{\PYZsh{} Frequency range for the Fourier transform}
\PY{n}{dz}\PY{+w}{ }\PY{o}{=}\PY{+w}{ }\PY{n}{z}\PY{p}{[}\PY{l+m+mi}{2}\PY{p}{]}\PY{+w}{ }\PY{o}{\PYZhy{}}\PY{+w}{ }\PY{n}{z}\PY{p}{[}\PY{l+m+mi}{1}\PY{p}{]}
\PY{n}{N}\PY{+w}{ }\PY{o}{=}\PY{+w}{ }\PY{n}{length}\PY{p}{(}\PY{n}{z}\PY{p}{)}
\PY{n}{freq}\PY{+w}{ }\PY{o}{=}\PY{+w}{ }\PY{n}{fftshift}\PY{p}{(}\PY{n}{fftfreq}\PY{p}{(}\PY{n}{N}\PY{p}{,}\PY{+w}{ }\PY{n}{dz}\PY{p}{)}\PY{p}{)}

\PY{c}{\PYZsh{} Convert data to Point2f type for plotting}
\PY{n}{original\PYZus{}function\PYZus{}points}\PY{+w}{ }\PY{o}{=}\PY{+w}{ }\PY{n}{Point2f}\PY{o}{.}\PY{p}{(}\PY{n}{collect}\PY{p}{(}\PY{n}{z}\PY{p}{)}\PY{p}{,}\PY{+w}{ }\PY{n}{collect}\PY{p}{(}\PY{n}{f\PYZus{}values}\PY{p}{)}\PY{p}{)}
\PY{n}{fourier\PYZus{}transform\PYZus{}points}\PY{+w}{ }\PY{o}{=}\PY{+w}{ }\PY{n}{Point2f}\PY{o}{.}\PY{p}{(}\PY{n}{collect}\PY{p}{(}\PY{n}{freq}\PY{p}{)}\PY{p}{,}\PY{+w}{ }\PY{n}{collect}\PY{p}{(}\PY{n}{abs}\PY{o}{.}\PY{p}{(}\PY{n}{F\PYZus{}transform}\PY{p}{)}\PY{p}{)}\PY{p}{)}

\PY{c}{\PYZsh{} Plot the original function and its Fourier transform using GLMakie}
\PY{n}{fig}\PY{+w}{ }\PY{o}{=}\PY{+w}{ }\PY{n}{Figure}\PY{p}{(}\PY{n}{size}\PY{+w}{ }\PY{o}{=}\PY{+w}{ }\PY{p}{(}\PY{l+m+mi}{1200}\PY{p}{,}\PY{+w}{ }\PY{l+m+mi}{600}\PY{p}{)}\PY{p}{)}

\PY{n}{ax1}\PY{+w}{ }\PY{o}{=}\PY{+w}{ }\PY{n}{Axis}\PY{p}{(}\PY{n}{fig}\PY{p}{[}\PY{l+m+mi}{1}\PY{p}{,}\PY{+w}{ }\PY{l+m+mi}{1}\PY{p}{]}\PY{p}{,}\PY{+w}{ }\PY{n}{title}\PY{+w}{ }\PY{o}{=}\PY{+w}{ }\PY{l+s}{\PYZdq{}}\PY{l+s}{Original Function}\PY{l+s}{\PYZdq{}}\PY{p}{,}\PY{+w}{ }\PY{n}{xlabel}\PY{+w}{ }\PY{o}{=}\PY{+w}{ }\PY{l+s}{\PYZdq{}}\PY{l+s}{z}\PY{l+s}{\PYZdq{}}\PY{p}{,}\PY{+w}{ }\PY{n}{ylabel}\PY{+w}{ }\PY{o}{=}\PY{+w}{ }\PY{l+s}{\PYZdq{}}\PY{l+s}{f(z)}\PY{l+s}{\PYZdq{}}\PY{p}{)}
\PY{n}{lines!}\PY{p}{(}\PY{n}{ax1}\PY{p}{,}\PY{+w}{ }\PY{n}{collect}\PY{p}{(}\PY{n}{z}\PY{p}{)}\PY{p}{,}\PY{+w}{ }\PY{n}{collect}\PY{p}{(}\PY{n}{f\PYZus{}values}\PY{p}{)}\PY{p}{,}\PY{+w}{ }\PY{n}{color}\PY{+w}{ }\PY{o}{=}\PY{+w}{ }\PY{l+s+ss}{:blue}\PY{p}{)}

\PY{n}{ax2}\PY{+w}{ }\PY{o}{=}\PY{+w}{ }\PY{n}{Axis}\PY{p}{(}\PY{n}{fig}\PY{p}{[}\PY{l+m+mi}{1}\PY{p}{,}\PY{+w}{ }\PY{l+m+mi}{2}\PY{p}{]}\PY{p}{,}\PY{+w}{ }\PY{n}{title}\PY{+w}{ }\PY{o}{=}\PY{+w}{ }\PY{l+s}{\PYZdq{}}\PY{l+s}{Fourier Transform}\PY{l+s}{\PYZdq{}}\PY{p}{,}\PY{+w}{ }\PY{n}{xlabel}\PY{+w}{ }\PY{o}{=}\PY{+w}{ }\PY{l+s}{\PYZdq{}}\PY{l+s}{Frequency (ξ)}\PY{l+s}{\PYZdq{}}\PY{p}{,}\PY{+w}{ }\PY{n}{ylabel}\PY{+w}{ }\PY{o}{=}\PY{+w}{ }\PY{l+s}{\PYZdq{}}\PY{l+s}{|F(ξ)|}\PY{l+s}{\PYZdq{}}\PY{p}{)}
\PY{n}{lines!}\PY{p}{(}\PY{n}{ax2}\PY{p}{,}\PY{+w}{ }\PY{n}{collect}\PY{p}{(}\PY{n}{freq}\PY{p}{)}\PY{p}{,}\PY{+w}{ }\PY{n}{collect}\PY{p}{(}\PY{n}{abs}\PY{o}{.}\PY{p}{(}\PY{n}{F\PYZus{}transform}\PY{p}{)}\PY{p}{)}\PY{p}{,}\PY{+w}{ }\PY{n}{color}\PY{+w}{ }\PY{o}{=}\PY{+w}{ }\PY{l+s+ss}{:red}\PY{p}{)}

\PY{n}{fig}\PY{p}{[}\PY{l+m+mi}{1}\PY{p}{,}\PY{+w}{ }\PY{l+m+mi}{1}\PY{p}{]}\PY{+w}{ }\PY{o}{=}\PY{+w}{ }\PY{n}{ax1}
\PY{n}{fig}\PY{p}{[}\PY{l+m+mi}{1}\PY{p}{,}\PY{+w}{ }\PY{l+m+mi}{2}\PY{p}{]}\PY{+w}{ }\PY{o}{=}\PY{+w}{ }\PY{n}{ax2}
\PY{n}{fig}
\PY{k}{end}
\end{Verbatim}
\end{tcolorbox}
 
            
\prompt{Out}{outcolor}{90}{}
    
    \begin{center}
    \adjustimage{max size={0.9\linewidth}{0.9\paperheight}}{output_6_0.png}
    \end{center}
    { \hspace*{\fill} \\}
    

    \[\large f(z)=\frac{1}{\pi}\dfrac{c}{c^{2}+z^{2}}\quad\text{which has simple poles at}\quad z=\pm ci\in \Im_{a}\quad\forall\,\, 0<a<c\]

    \begin{tcolorbox}[breakable, size=fbox, boxrule=1pt, pad at break*=1mm,colback=cellbackground, colframe=cellborder]
\prompt{In}{incolor}{91}{\boxspacing}
\begin{Verbatim}[commandchars=\\\{\}]
\PY{k}{begin}
\PY{k}{using}\PY{+w}{ }\PY{n}{GLMakie}
\PY{k}{using}\PY{+w}{ }\PY{n}{FFTW}

\PY{c}{\PYZsh{} Define the function}
\PY{n}{c}\PY{+w}{ }\PY{o}{=}\PY{+w}{ }\PY{l+m+mf}{1.0}\PY{+w}{  }\PY{c}{\PYZsh{} You can change this value as needed}
\PY{k}{function}\PY{+w}{ }\PY{n}{f}\PY{p}{(}\PY{n}{z}\PY{p}{)}
\PY{+w}{    }\PY{k}{return}\PY{+w}{ }\PY{p}{(}\PY{l+m+mi}{1}\PY{+w}{ }\PY{o}{/}\PY{+w}{ }\PY{n+nb}{π}\PY{p}{)}\PY{+w}{ }\PY{o}{*}\PY{+w}{ }\PY{p}{(}\PY{n}{c}\PY{+w}{ }\PY{o}{/}\PY{+w}{ }\PY{p}{(}\PY{n}{c}\PY{o}{\PYZca{}}\PY{l+m+mi}{2}\PY{+w}{ }\PY{o}{+}\PY{+w}{ }\PY{n}{z}\PY{o}{\PYZca{}}\PY{l+m+mi}{2}\PY{p}{)}\PY{p}{)}
\PY{k}{end}

\PY{c}{\PYZsh{} Define the range for z and the sample points}
\PY{n}{z}\PY{+w}{ }\PY{o}{=}\PY{+w}{ }\PY{k+kt}{LinRange}\PY{p}{(}\PY{o}{\PYZhy{}}\PY{l+m+mi}{10}\PY{p}{,}\PY{+w}{ }\PY{l+m+mi}{10}\PY{p}{,}\PY{+w}{ }\PY{l+m+mi}{1024}\PY{p}{)}
\PY{n}{f\PYZus{}values}\PY{+w}{ }\PY{o}{=}\PY{+w}{ }\PY{n}{f}\PY{o}{.}\PY{p}{(}\PY{n}{z}\PY{p}{)}

\PY{c}{\PYZsh{} Compute the Fourier transform using FFT}
\PY{n}{F\PYZus{}transform}\PY{+w}{ }\PY{o}{=}\PY{+w}{ }\PY{n}{fftshift}\PY{p}{(}\PY{n}{fft}\PY{p}{(}\PY{n}{f\PYZus{}values}\PY{p}{)}\PY{p}{)}

\PY{c}{\PYZsh{} Frequency range for the Fourier transform}
\PY{n}{dz}\PY{+w}{ }\PY{o}{=}\PY{+w}{ }\PY{n}{z}\PY{p}{[}\PY{l+m+mi}{2}\PY{p}{]}\PY{+w}{ }\PY{o}{\PYZhy{}}\PY{+w}{ }\PY{n}{z}\PY{p}{[}\PY{l+m+mi}{1}\PY{p}{]}
\PY{n}{N}\PY{+w}{ }\PY{o}{=}\PY{+w}{ }\PY{n}{length}\PY{p}{(}\PY{n}{z}\PY{p}{)}
\PY{n}{freq}\PY{+w}{ }\PY{o}{=}\PY{+w}{ }\PY{n}{fftshift}\PY{p}{(}\PY{n}{fftfreq}\PY{p}{(}\PY{n}{N}\PY{p}{,}\PY{+w}{ }\PY{n}{dz}\PY{p}{)}\PY{p}{)}

\PY{c}{\PYZsh{} Plot the original function and its Fourier transform using GLMakie}
\PY{n}{fig}\PY{+w}{ }\PY{o}{=}\PY{+w}{ }\PY{n}{Figure}\PY{p}{(}\PY{n}{size}\PY{+w}{ }\PY{o}{=}\PY{+w}{ }\PY{p}{(}\PY{l+m+mi}{1200}\PY{p}{,}\PY{+w}{ }\PY{l+m+mi}{600}\PY{p}{)}\PY{p}{)}

\PY{n}{ax1}\PY{+w}{ }\PY{o}{=}\PY{+w}{ }\PY{n}{Axis}\PY{p}{(}\PY{n}{fig}\PY{p}{[}\PY{l+m+mi}{1}\PY{p}{,}\PY{+w}{ }\PY{l+m+mi}{1}\PY{p}{]}\PY{p}{,}\PY{+w}{ }\PY{n}{title}\PY{+w}{ }\PY{o}{=}\PY{+w}{ }\PY{l+s}{\PYZdq{}}\PY{l+s}{Original Function}\PY{l+s}{\PYZdq{}}\PY{p}{,}\PY{+w}{ }\PY{n}{xlabel}\PY{+w}{ }\PY{o}{=}\PY{+w}{ }\PY{l+s}{\PYZdq{}}\PY{l+s}{z}\PY{l+s}{\PYZdq{}}\PY{p}{,}\PY{+w}{ }\PY{n}{ylabel}\PY{+w}{ }\PY{o}{=}\PY{+w}{ }\PY{l+s}{\PYZdq{}}\PY{l+s}{f(z)}\PY{l+s}{\PYZdq{}}\PY{p}{)}
\PY{n}{lines!}\PY{p}{(}\PY{n}{ax1}\PY{p}{,}\PY{+w}{ }\PY{n}{z}\PY{p}{,}\PY{+w}{ }\PY{n}{f\PYZus{}values}\PY{p}{,}\PY{+w}{ }\PY{n}{color}\PY{+w}{ }\PY{o}{=}\PY{+w}{ }\PY{l+s+ss}{:blue}\PY{p}{)}

\PY{n}{ax2}\PY{+w}{ }\PY{o}{=}\PY{+w}{ }\PY{n}{Axis}\PY{p}{(}\PY{n}{fig}\PY{p}{[}\PY{l+m+mi}{1}\PY{p}{,}\PY{+w}{ }\PY{l+m+mi}{2}\PY{p}{]}\PY{p}{,}\PY{+w}{ }\PY{n}{title}\PY{+w}{ }\PY{o}{=}\PY{+w}{ }\PY{l+s}{\PYZdq{}}\PY{l+s}{Fourier Transform}\PY{l+s}{\PYZdq{}}\PY{p}{,}\PY{+w}{ }\PY{n}{xlabel}\PY{+w}{ }\PY{o}{=}\PY{+w}{ }\PY{l+s}{\PYZdq{}}\PY{l+s}{Frequency (ξ)}\PY{l+s}{\PYZdq{}}\PY{p}{,}\PY{+w}{ }\PY{n}{ylabel}\PY{+w}{ }\PY{o}{=}\PY{+w}{ }\PY{l+s}{\PYZdq{}}\PY{l+s}{|F(ξ)|}\PY{l+s}{\PYZdq{}}\PY{p}{)}
\PY{n}{lines!}\PY{p}{(}\PY{n}{ax2}\PY{p}{,}\PY{+w}{ }\PY{n}{freq}\PY{p}{,}\PY{+w}{ }\PY{n}{abs}\PY{o}{.}\PY{p}{(}\PY{n}{F\PYZus{}transform}\PY{p}{)}\PY{p}{,}\PY{+w}{ }\PY{n}{color}\PY{+w}{ }\PY{o}{=}\PY{+w}{ }\PY{l+s+ss}{:red}\PY{p}{)}

\PY{n}{fig}\PY{p}{[}\PY{l+m+mi}{1}\PY{p}{,}\PY{+w}{ }\PY{l+m+mi}{1}\PY{p}{]}\PY{+w}{ }\PY{o}{=}\PY{+w}{ }\PY{n}{ax1}
\PY{n}{fig}\PY{p}{[}\PY{l+m+mi}{1}\PY{p}{,}\PY{+w}{ }\PY{l+m+mi}{2}\PY{p}{]}\PY{+w}{ }\PY{o}{=}\PY{+w}{ }\PY{n}{ax2}

\PY{+w}{    }\PY{n}{fig}
\PY{k}{end}
\end{Verbatim}
\end{tcolorbox}
 
            
\prompt{Out}{outcolor}{91}{}
    
    \begin{center}
    \adjustimage{max size={0.9\linewidth}{0.9\paperheight}}{output_8_0.png}
    \end{center}
    { \hspace*{\fill} \\}
    

    \subsection{\texorpdfstring{Action of the Fourier transform on
\(\Im\)}{Action of the Fourier transform on \textbackslash Im}}\label{action-of-the-fourier-transform-on-im}

If \(f\) belongs to the class \(\Im_{a}\,\,\exists\, a>0\) then

\$\$\textbar{}\hat{f}(\xi)\textbar{}\le Be\^{}\{-2\pi b\textbar{}\xi\textbar\}\quad \forall 0\le b
\textless a

\(\implies\) whenever \(f\in\Im\) then \(\hat{f}\) has rapid decay at
\(\infty\)

If \(f\in\Im\) then the Fourier inversion holds, namely
\[\Large f(x)=\int_{-\infty}^{+\infty} \hat{f}(\xi)e^{2\pi i x\xi}\,d\xi\quad\forall x\in\mathbb{R}\]

\subsubsection{Poisson Summation
Formula}\label{poisson-summation-formula}

If \(f\in\Im\) then :
\(\sum_{n\in\mathbb{Z}}f(n)=\sum_{n\in\mathbb{Z}}\hat{f}(n)\)

\[\Huge\implies \sum_{n=-\infty}^{\infty}e^{-\pi t(n+a)^{2}}\,=\,\sum_{n=-\infty}^{\infty}t^{-1/2}e^{\pi n^{2}/t}e^{2\pi ina}\]

\textbf{for any fixed \(\large t>0\) and \(\large a\in\mathbb{R}\)}

\textbf{Special Case}: \(a=0\) is the transformation law for a version
of \textbf{``theta function''}:

if we define \(\vartheta\) for \(t>0\) by the series
\(\large\vartheta(t)=\sum_{n=-\infty}^{\infty} e^{-\pi n^{2}t}\) then

\[\huge \vartheta(t)=t^{-1/2}\vartheta(1/t)\quad t>0\]

    \subsection{Poley-Wiener Theorem}\label{poley-wiener-theorem}

\begin{itemize}
\item
  \textbf{If \(\large\hat{f}(\xi)=O(e^{-2\pi a|\xi|})\quad\exists a>0\)
  and \(f\) vanishes in a non empty open interval, then \(f=0\)}
\item
  \textbf{If \(f\) is continuous and of moderate decrease on
  \(\mathbb{R}\). Then \(f\) has an extension to the complex plane that
  is entire with \(\large|f(z)|\le Ae^{2\pi M |z|}\quad\exists A>0\), if
  and only if \(\hat{f}\) is supported in the interval \([-M, M]\)}
\end{itemize}

\textbf{Suppose \(F\) is a holomorphic function in the sector}

\[\Large\mathcal{S}=\left\{z:\frac{-\pi}{4}<\arg z < \frac{\pi}{4}\right\}\]

that is continuous on the closure of \(\mathcal{S}\). Assume
\(|F(z)|\le 1\) on the boundary of the sector, and that there are
constants \(C,c>0\) such that
\(|F(z)|\le Ce^{c|z|}\quad \forall z\in\mathcal{S}\) Then
\[\huge|F(z)|\le 1 \quad \forall\,\,z\in\mathcal{S}\]

Note: \(\arg(z)\) denotes the angle which \(z\) makes with the positive
real axis.

\textbf{Suppose} \(f\) and \(\hat{f}\) have \emph{moderate decrease}.
Then \(\Large\hat{f}(\xi)=0\quad\forall\,\,\xi<0\) if and only if \(f\)
can be extended to a continuous and bounded function in the close upper
half-plane \(\left\{z=x+iy\,:\,y>0\right\}\) with \(f\) holomorphic in
the interior.

    \begin{tcolorbox}[breakable, size=fbox, boxrule=1pt, pad at break*=1mm,colback=cellbackground, colframe=cellborder]
\prompt{In}{incolor}{92}{\boxspacing}
\begin{Verbatim}[commandchars=\\\{\}]
\PY{k}{begin}
\PY{+w}{    }\PY{n}{fig}\PY{+w}{ }\PY{o}{=}\PY{+w}{ }\PY{n}{GLMakie}\PY{o}{.}\PY{n}{Figure}\PY{p}{(}\PY{n}{size}\PY{+w}{ }\PY{o}{=}\PY{+w}{ }\PY{p}{(}\PY{l+m+mi}{600}\PY{p}{,}\PY{+w}{ }\PY{l+m+mi}{600}\PY{p}{)}\PY{p}{)}
\PY{+w}{    }\PY{n}{ax}\PY{+w}{ }\PY{o}{=}\PY{+w}{ }\PY{n}{GLMakie}\PY{o}{.}\PY{n}{Axis}\PY{p}{(}\PY{n}{fig}\PY{p}{[}\PY{l+m+mi}{1}\PY{p}{,}\PY{+w}{ }\PY{l+m+mi}{1}\PY{p}{]}\PY{p}{,}\PY{+w}{ }\PY{n}{limits}\PY{+w}{ }\PY{o}{=}\PY{+w}{ }\PY{p}{(}\PY{o}{\PYZhy{}}\PY{l+m+mi}{2}\PY{p}{,}\PY{+w}{ }\PY{l+m+mi}{2}\PY{p}{,}\PY{+w}{ }\PY{o}{\PYZhy{}}\PY{l+m+mi}{2}\PY{p}{,}\PY{+w}{ }\PY{l+m+mi}{2}\PY{p}{)}\PY{p}{,}\PY{+w}{ }\PY{n}{title}\PY{+w}{ }\PY{o}{=}\PY{+w}{ }\PY{l+s}{\PYZdq{}}\PY{l+s}{Sector of Complex Plane}\PY{l+s}{\PYZdq{}}\PY{p}{,}\PY{n}{xticksvisible}\PY{o}{=}\PY{n+nb}{false}\PY{p}{,}\PY{n}{yticksvisible}\PY{o}{=}\PY{n+nb}{false}\PY{p}{,}\PY{n}{xlabel}\PY{o}{=}\PY{l+s}{\PYZdq{}}\PY{l+s}{Real axis}\PY{l+s}{\PYZdq{}}\PY{p}{,}\PY{n}{ylabel}\PY{o}{=}\PY{l+s}{\PYZdq{}}\PY{l+s}{Im Axis}\PY{l+s}{\PYZdq{}}\PY{p}{)}

\PY{+w}{    }\PY{n}{θ}\PY{+w}{ }\PY{o}{=}\PY{+w}{ }\PY{n}{range}\PY{p}{(}\PY{o}{\PYZhy{}}\PY{n+nb}{π}\PY{o}{/}\PY{l+m+mi}{4}\PY{p}{,}\PY{+w}{ }\PY{n+nb}{π}\PY{o}{/}\PY{l+m+mi}{4}\PY{p}{;}\PY{+w}{ }\PY{n}{length}\PY{o}{=}\PY{l+m+mi}{10000}\PY{p}{)}
\PY{+w}{    }\PY{n}{x}\PY{+w}{ }\PY{o}{=}\PY{+w}{ }\PY{n}{cos}\PY{o}{.}\PY{p}{(}\PY{n}{θ}\PY{p}{)}
\PY{+w}{    }\PY{n}{y}\PY{+w}{ }\PY{o}{=}\PY{+w}{ }\PY{n}{sin}\PY{o}{.}\PY{p}{(}\PY{n}{θ}\PY{p}{)}

\PY{+w}{    }\PY{n}{GLMakie}\PY{o}{.}\PY{n}{lines!}\PY{p}{(}\PY{n}{ax}\PY{p}{,}\PY{+w}{ }\PY{p}{[}\PY{l+m+mi}{0}\PY{p}{,}\PY{+w}{ }\PY{l+m+mi}{0}\PY{p}{]}\PY{p}{,}\PY{+w}{ }\PY{p}{[}\PY{l+m+mi}{0}\PY{p}{,}\PY{+w}{ }\PY{l+m+mi}{2}\PY{p}{]}\PY{p}{)}
\PY{+w}{    }\PY{n}{GLMakie}\PY{o}{.}\PY{n}{lines!}\PY{p}{(}\PY{n}{ax}\PY{p}{,}\PY{+w}{ }\PY{p}{[}\PY{l+m+mi}{0}\PY{p}{,}\PY{+w}{ }\PY{l+m+mi}{2}\PY{p}{]}\PY{p}{,}\PY{+w}{ }\PY{p}{[}\PY{l+m+mi}{0}\PY{p}{,}\PY{+w}{ }\PY{l+m+mi}{0}\PY{p}{]}\PY{p}{)}
\PY{+w}{    }\PY{n}{GLMakie}\PY{o}{.}\PY{n}{lines!}\PY{p}{(}\PY{n}{ax}\PY{p}{,}\PY{+w}{ }\PY{p}{[}\PY{l+m+mi}{0}\PY{p}{,}\PY{+w}{ }\PY{l+m+mi}{2}\PY{p}{]}\PY{p}{,}\PY{+w}{ }\PY{p}{[}\PY{l+m+mi}{0}\PY{p}{,}\PY{+w}{ }\PY{l+m+mi}{2}\PY{p}{]}\PY{p}{)}
\PY{+w}{  }
\PY{+w}{    }\PY{n}{GLMakie}\PY{o}{.}\PY{n}{scatter!}\PY{p}{(}\PY{n}{ax}\PY{p}{,}\PY{+w}{ }\PY{p}{[}\PY{n}{x}\PY{p}{;}\PY{+w}{ }\PY{o}{\PYZhy{}}\PY{n}{x}\PY{p}{]}\PY{p}{,}\PY{+w}{ }\PY{p}{[}\PY{n}{y}\PY{p}{;}\PY{+w}{ }\PY{o}{\PYZhy{}}\PY{n}{y}\PY{p}{]}\PY{p}{,}\PY{+w}{ }\PY{n}{color}\PY{+w}{ }\PY{o}{=}\PY{+w}{ }\PY{l+s+ss}{:blue}\PY{p}{)}
\PY{+w}{    }\PY{n}{GLMakie}\PY{o}{.}\PY{n}{poly!}\PY{p}{(}\PY{n}{ax}\PY{p}{,}\PY{+w}{ }\PY{n}{GLMakie}\PY{o}{.}\PY{n}{Point2f0}\PY{o}{.}\PY{p}{(}\PY{p}{[}\PY{p}{(}\PY{l+m+mi}{0}\PY{p}{,}\PY{+w}{ }\PY{l+m+mi}{0}\PY{p}{)}\PY{p}{,}\PY{+w}{ }\PY{p}{(}\PY{l+m+mi}{2}\PY{p}{,}\PY{+w}{ }\PY{n+nb}{π}\PY{o}{/}\PY{l+m+mi}{4}\PY{p}{)}\PY{p}{,}\PY{+w}{ }\PY{p}{(}\PY{l+m+mi}{2}\PY{p}{,}\PY{+w}{ }\PY{o}{\PYZhy{}}\PY{n+nb}{π}\PY{o}{/}\PY{l+m+mi}{4}\PY{p}{)}\PY{p}{]}\PY{p}{)}\PY{p}{,}\PY{+w}{ }\PY{n}{color}\PY{+w}{ }\PY{o}{=}\PY{+w}{ }\PY{p}{(}\PY{l+s+ss}{:lightblue}\PY{p}{,}\PY{+w}{ }\PY{l+m+mf}{0.5}\PY{p}{)}\PY{p}{)}
\PY{+w}{    }
\PY{+w}{    }\PY{n}{fig}
\PY{k}{end}
\end{Verbatim}
\end{tcolorbox}
 
            
\prompt{Out}{outcolor}{92}{}
    
    \begin{center}
    \adjustimage{max size={0.9\linewidth}{0.9\paperheight}}{output_11_0.png}
    \end{center}
    { \hspace*{\fill} \\}
    

    \begin{tcolorbox}[breakable, size=fbox, boxrule=1pt, pad at break*=1mm,colback=cellbackground, colframe=cellborder]
\prompt{In}{incolor}{ }{\boxspacing}
\begin{Verbatim}[commandchars=\\\{\}]

\end{Verbatim}
\end{tcolorbox}

    \begin{tcolorbox}[breakable, size=fbox, boxrule=1pt, pad at break*=1mm,colback=cellbackground, colframe=cellborder]
\prompt{In}{incolor}{ }{\boxspacing}
\begin{Verbatim}[commandchars=\\\{\}]

\end{Verbatim}
\end{tcolorbox}


    % Add a bibliography block to the postdoc
    
    
    
\end{document}
