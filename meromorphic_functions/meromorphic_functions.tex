\documentclass[11pt]{article}

    \usepackage[breakable]{tcolorbox}
    \usepackage{parskip} % Stop auto-indenting (to mimic markdown behaviour)
    

    % Basic figure setup, for now with no caption control since it's done
    % automatically by Pandoc (which extracts ![](path) syntax from Markdown).
    \usepackage{graphicx}
    % Keep aspect ratio if custom image width or height is specified
    \setkeys{Gin}{keepaspectratio}
    % Maintain compatibility with old templates. Remove in nbconvert 6.0
    \let\Oldincludegraphics\includegraphics
    % Ensure that by default, figures have no caption (until we provide a
    % proper Figure object with a Caption API and a way to capture that
    % in the conversion process - todo).
    \usepackage{caption}
    \DeclareCaptionFormat{nocaption}{}
    \captionsetup{format=nocaption,aboveskip=0pt,belowskip=0pt}

    \usepackage{float}
    \floatplacement{figure}{H} % forces figures to be placed at the correct location
    \usepackage{xcolor} % Allow colors to be defined
    \usepackage{enumerate} % Needed for markdown enumerations to work
    \usepackage{geometry} % Used to adjust the document margins
    \usepackage{amsmath} % Equations
    \usepackage{amssymb} % Equations
    \usepackage{textcomp} % defines textquotesingle
    % Hack from http://tex.stackexchange.com/a/47451/13684:
    \AtBeginDocument{%
        \def\PYZsq{\textquotesingle}% Upright quotes in Pygmentized code
    }
    \usepackage{upquote} % Upright quotes for verbatim code
    \usepackage{eurosym} % defines \euro

    \usepackage{iftex}
    \ifPDFTeX
        \usepackage[T1]{fontenc}
        \IfFileExists{alphabeta.sty}{
              \usepackage{alphabeta}
          }{
              \usepackage[mathletters]{ucs}
              \usepackage[utf8x]{inputenc}
          }
    \else
        \usepackage{fontspec}
        \usepackage{unicode-math}
    \fi

    \usepackage{fancyvrb} % verbatim replacement that allows latex
    \usepackage{grffile} % extends the file name processing of package graphics
                         % to support a larger range
    \makeatletter % fix for old versions of grffile with XeLaTeX
    \@ifpackagelater{grffile}{2019/11/01}
    {
      % Do nothing on new versions
    }
    {
      \def\Gread@@xetex#1{%
        \IfFileExists{"\Gin@base".bb}%
        {\Gread@eps{\Gin@base.bb}}%
        {\Gread@@xetex@aux#1}%
      }
    }
    \makeatother
    \usepackage[Export]{adjustbox} % Used to constrain images to a maximum size
    \adjustboxset{max size={0.9\linewidth}{0.9\paperheight}}

    % The hyperref package gives us a pdf with properly built
    % internal navigation ('pdf bookmarks' for the table of contents,
    % internal cross-reference links, web links for URLs, etc.)
    \usepackage{hyperref}
    % The default LaTeX title has an obnoxious amount of whitespace. By default,
    % titling removes some of it. It also provides customization options.
    \usepackage{titling}
    \usepackage{longtable} % longtable support required by pandoc >1.10
    \usepackage{booktabs}  % table support for pandoc > 1.12.2
    \usepackage{array}     % table support for pandoc >= 2.11.3
    \usepackage{calc}      % table minipage width calculation for pandoc >= 2.11.1
    \usepackage[inline]{enumitem} % IRkernel/repr support (it uses the enumerate* environment)
    \usepackage[normalem]{ulem} % ulem is needed to support strikethroughs (\sout)
                                % normalem makes italics be italics, not underlines
    \usepackage{soul}      % strikethrough (\st) support for pandoc >= 3.0.0
    \usepackage{mathrsfs}
    

    
    % Colors for the hyperref package
    \definecolor{urlcolor}{rgb}{0,.145,.698}
    \definecolor{linkcolor}{rgb}{.71,0.21,0.01}
    \definecolor{citecolor}{rgb}{.12,.54,.11}

    % ANSI colors
    \definecolor{ansi-black}{HTML}{3E424D}
    \definecolor{ansi-black-intense}{HTML}{282C36}
    \definecolor{ansi-red}{HTML}{E75C58}
    \definecolor{ansi-red-intense}{HTML}{B22B31}
    \definecolor{ansi-green}{HTML}{00A250}
    \definecolor{ansi-green-intense}{HTML}{007427}
    \definecolor{ansi-yellow}{HTML}{DDB62B}
    \definecolor{ansi-yellow-intense}{HTML}{B27D12}
    \definecolor{ansi-blue}{HTML}{208FFB}
    \definecolor{ansi-blue-intense}{HTML}{0065CA}
    \definecolor{ansi-magenta}{HTML}{D160C4}
    \definecolor{ansi-magenta-intense}{HTML}{A03196}
    \definecolor{ansi-cyan}{HTML}{60C6C8}
    \definecolor{ansi-cyan-intense}{HTML}{258F8F}
    \definecolor{ansi-white}{HTML}{C5C1B4}
    \definecolor{ansi-white-intense}{HTML}{A1A6B2}
    \definecolor{ansi-default-inverse-fg}{HTML}{FFFFFF}
    \definecolor{ansi-default-inverse-bg}{HTML}{000000}

    % common color for the border for error outputs.
    \definecolor{outerrorbackground}{HTML}{FFDFDF}

    % commands and environments needed by pandoc snippets
    % extracted from the output of `pandoc -s`
    \providecommand{\tightlist}{%
      \setlength{\itemsep}{0pt}\setlength{\parskip}{0pt}}
    \DefineVerbatimEnvironment{Highlighting}{Verbatim}{commandchars=\\\{\}}
    % Add ',fontsize=\small' for more characters per line
    \newenvironment{Shaded}{}{}
    \newcommand{\KeywordTok}[1]{\textcolor[rgb]{0.00,0.44,0.13}{\textbf{{#1}}}}
    \newcommand{\DataTypeTok}[1]{\textcolor[rgb]{0.56,0.13,0.00}{{#1}}}
    \newcommand{\DecValTok}[1]{\textcolor[rgb]{0.25,0.63,0.44}{{#1}}}
    \newcommand{\BaseNTok}[1]{\textcolor[rgb]{0.25,0.63,0.44}{{#1}}}
    \newcommand{\FloatTok}[1]{\textcolor[rgb]{0.25,0.63,0.44}{{#1}}}
    \newcommand{\CharTok}[1]{\textcolor[rgb]{0.25,0.44,0.63}{{#1}}}
    \newcommand{\StringTok}[1]{\textcolor[rgb]{0.25,0.44,0.63}{{#1}}}
    \newcommand{\CommentTok}[1]{\textcolor[rgb]{0.38,0.63,0.69}{\textit{{#1}}}}
    \newcommand{\OtherTok}[1]{\textcolor[rgb]{0.00,0.44,0.13}{{#1}}}
    \newcommand{\AlertTok}[1]{\textcolor[rgb]{1.00,0.00,0.00}{\textbf{{#1}}}}
    \newcommand{\FunctionTok}[1]{\textcolor[rgb]{0.02,0.16,0.49}{{#1}}}
    \newcommand{\RegionMarkerTok}[1]{{#1}}
    \newcommand{\ErrorTok}[1]{\textcolor[rgb]{1.00,0.00,0.00}{\textbf{{#1}}}}
    \newcommand{\NormalTok}[1]{{#1}}

    % Additional commands for more recent versions of Pandoc
    \newcommand{\ConstantTok}[1]{\textcolor[rgb]{0.53,0.00,0.00}{{#1}}}
    \newcommand{\SpecialCharTok}[1]{\textcolor[rgb]{0.25,0.44,0.63}{{#1}}}
    \newcommand{\VerbatimStringTok}[1]{\textcolor[rgb]{0.25,0.44,0.63}{{#1}}}
    \newcommand{\SpecialStringTok}[1]{\textcolor[rgb]{0.73,0.40,0.53}{{#1}}}
    \newcommand{\ImportTok}[1]{{#1}}
    \newcommand{\DocumentationTok}[1]{\textcolor[rgb]{0.73,0.13,0.13}{\textit{{#1}}}}
    \newcommand{\AnnotationTok}[1]{\textcolor[rgb]{0.38,0.63,0.69}{\textbf{\textit{{#1}}}}}
    \newcommand{\CommentVarTok}[1]{\textcolor[rgb]{0.38,0.63,0.69}{\textbf{\textit{{#1}}}}}
    \newcommand{\VariableTok}[1]{\textcolor[rgb]{0.10,0.09,0.49}{{#1}}}
    \newcommand{\ControlFlowTok}[1]{\textcolor[rgb]{0.00,0.44,0.13}{\textbf{{#1}}}}
    \newcommand{\OperatorTok}[1]{\textcolor[rgb]{0.40,0.40,0.40}{{#1}}}
    \newcommand{\BuiltInTok}[1]{{#1}}
    \newcommand{\ExtensionTok}[1]{{#1}}
    \newcommand{\PreprocessorTok}[1]{\textcolor[rgb]{0.74,0.48,0.00}{{#1}}}
    \newcommand{\AttributeTok}[1]{\textcolor[rgb]{0.49,0.56,0.16}{{#1}}}
    \newcommand{\InformationTok}[1]{\textcolor[rgb]{0.38,0.63,0.69}{\textbf{\textit{{#1}}}}}
    \newcommand{\WarningTok}[1]{\textcolor[rgb]{0.38,0.63,0.69}{\textbf{\textit{{#1}}}}}


    % Define a nice break command that doesn't care if a line doesn't already
    % exist.
    \def\br{\hspace*{\fill} \\* }
    % Math Jax compatibility definitions
    \def\gt{>}
    \def\lt{<}
    \let\Oldtex\TeX
    \let\Oldlatex\LaTeX
    \renewcommand{\TeX}{\textrm{\Oldtex}}
    \renewcommand{\LaTeX}{\textrm{\Oldlatex}}
    % Document parameters
    % Document title
    \title{meromorphic\_functions}
    
    
    
    
    
    
    
% Pygments definitions
\makeatletter
\def\PY@reset{\let\PY@it=\relax \let\PY@bf=\relax%
    \let\PY@ul=\relax \let\PY@tc=\relax%
    \let\PY@bc=\relax \let\PY@ff=\relax}
\def\PY@tok#1{\csname PY@tok@#1\endcsname}
\def\PY@toks#1+{\ifx\relax#1\empty\else%
    \PY@tok{#1}\expandafter\PY@toks\fi}
\def\PY@do#1{\PY@bc{\PY@tc{\PY@ul{%
    \PY@it{\PY@bf{\PY@ff{#1}}}}}}}
\def\PY#1#2{\PY@reset\PY@toks#1+\relax+\PY@do{#2}}

\@namedef{PY@tok@w}{\def\PY@tc##1{\textcolor[rgb]{0.73,0.73,0.73}{##1}}}
\@namedef{PY@tok@c}{\let\PY@it=\textit\def\PY@tc##1{\textcolor[rgb]{0.24,0.48,0.48}{##1}}}
\@namedef{PY@tok@cp}{\def\PY@tc##1{\textcolor[rgb]{0.61,0.40,0.00}{##1}}}
\@namedef{PY@tok@k}{\let\PY@bf=\textbf\def\PY@tc##1{\textcolor[rgb]{0.00,0.50,0.00}{##1}}}
\@namedef{PY@tok@kp}{\def\PY@tc##1{\textcolor[rgb]{0.00,0.50,0.00}{##1}}}
\@namedef{PY@tok@kt}{\def\PY@tc##1{\textcolor[rgb]{0.69,0.00,0.25}{##1}}}
\@namedef{PY@tok@o}{\def\PY@tc##1{\textcolor[rgb]{0.40,0.40,0.40}{##1}}}
\@namedef{PY@tok@ow}{\let\PY@bf=\textbf\def\PY@tc##1{\textcolor[rgb]{0.67,0.13,1.00}{##1}}}
\@namedef{PY@tok@nb}{\def\PY@tc##1{\textcolor[rgb]{0.00,0.50,0.00}{##1}}}
\@namedef{PY@tok@nf}{\def\PY@tc##1{\textcolor[rgb]{0.00,0.00,1.00}{##1}}}
\@namedef{PY@tok@nc}{\let\PY@bf=\textbf\def\PY@tc##1{\textcolor[rgb]{0.00,0.00,1.00}{##1}}}
\@namedef{PY@tok@nn}{\let\PY@bf=\textbf\def\PY@tc##1{\textcolor[rgb]{0.00,0.00,1.00}{##1}}}
\@namedef{PY@tok@ne}{\let\PY@bf=\textbf\def\PY@tc##1{\textcolor[rgb]{0.80,0.25,0.22}{##1}}}
\@namedef{PY@tok@nv}{\def\PY@tc##1{\textcolor[rgb]{0.10,0.09,0.49}{##1}}}
\@namedef{PY@tok@no}{\def\PY@tc##1{\textcolor[rgb]{0.53,0.00,0.00}{##1}}}
\@namedef{PY@tok@nl}{\def\PY@tc##1{\textcolor[rgb]{0.46,0.46,0.00}{##1}}}
\@namedef{PY@tok@ni}{\let\PY@bf=\textbf\def\PY@tc##1{\textcolor[rgb]{0.44,0.44,0.44}{##1}}}
\@namedef{PY@tok@na}{\def\PY@tc##1{\textcolor[rgb]{0.41,0.47,0.13}{##1}}}
\@namedef{PY@tok@nt}{\let\PY@bf=\textbf\def\PY@tc##1{\textcolor[rgb]{0.00,0.50,0.00}{##1}}}
\@namedef{PY@tok@nd}{\def\PY@tc##1{\textcolor[rgb]{0.67,0.13,1.00}{##1}}}
\@namedef{PY@tok@s}{\def\PY@tc##1{\textcolor[rgb]{0.73,0.13,0.13}{##1}}}
\@namedef{PY@tok@sd}{\let\PY@it=\textit\def\PY@tc##1{\textcolor[rgb]{0.73,0.13,0.13}{##1}}}
\@namedef{PY@tok@si}{\let\PY@bf=\textbf\def\PY@tc##1{\textcolor[rgb]{0.64,0.35,0.47}{##1}}}
\@namedef{PY@tok@se}{\let\PY@bf=\textbf\def\PY@tc##1{\textcolor[rgb]{0.67,0.36,0.12}{##1}}}
\@namedef{PY@tok@sr}{\def\PY@tc##1{\textcolor[rgb]{0.64,0.35,0.47}{##1}}}
\@namedef{PY@tok@ss}{\def\PY@tc##1{\textcolor[rgb]{0.10,0.09,0.49}{##1}}}
\@namedef{PY@tok@sx}{\def\PY@tc##1{\textcolor[rgb]{0.00,0.50,0.00}{##1}}}
\@namedef{PY@tok@m}{\def\PY@tc##1{\textcolor[rgb]{0.40,0.40,0.40}{##1}}}
\@namedef{PY@tok@gh}{\let\PY@bf=\textbf\def\PY@tc##1{\textcolor[rgb]{0.00,0.00,0.50}{##1}}}
\@namedef{PY@tok@gu}{\let\PY@bf=\textbf\def\PY@tc##1{\textcolor[rgb]{0.50,0.00,0.50}{##1}}}
\@namedef{PY@tok@gd}{\def\PY@tc##1{\textcolor[rgb]{0.63,0.00,0.00}{##1}}}
\@namedef{PY@tok@gi}{\def\PY@tc##1{\textcolor[rgb]{0.00,0.52,0.00}{##1}}}
\@namedef{PY@tok@gr}{\def\PY@tc##1{\textcolor[rgb]{0.89,0.00,0.00}{##1}}}
\@namedef{PY@tok@ge}{\let\PY@it=\textit}
\@namedef{PY@tok@gs}{\let\PY@bf=\textbf}
\@namedef{PY@tok@ges}{\let\PY@bf=\textbf\let\PY@it=\textit}
\@namedef{PY@tok@gp}{\let\PY@bf=\textbf\def\PY@tc##1{\textcolor[rgb]{0.00,0.00,0.50}{##1}}}
\@namedef{PY@tok@go}{\def\PY@tc##1{\textcolor[rgb]{0.44,0.44,0.44}{##1}}}
\@namedef{PY@tok@gt}{\def\PY@tc##1{\textcolor[rgb]{0.00,0.27,0.87}{##1}}}
\@namedef{PY@tok@err}{\def\PY@bc##1{{\setlength{\fboxsep}{\string -\fboxrule}\fcolorbox[rgb]{1.00,0.00,0.00}{1,1,1}{\strut ##1}}}}
\@namedef{PY@tok@kc}{\let\PY@bf=\textbf\def\PY@tc##1{\textcolor[rgb]{0.00,0.50,0.00}{##1}}}
\@namedef{PY@tok@kd}{\let\PY@bf=\textbf\def\PY@tc##1{\textcolor[rgb]{0.00,0.50,0.00}{##1}}}
\@namedef{PY@tok@kn}{\let\PY@bf=\textbf\def\PY@tc##1{\textcolor[rgb]{0.00,0.50,0.00}{##1}}}
\@namedef{PY@tok@kr}{\let\PY@bf=\textbf\def\PY@tc##1{\textcolor[rgb]{0.00,0.50,0.00}{##1}}}
\@namedef{PY@tok@bp}{\def\PY@tc##1{\textcolor[rgb]{0.00,0.50,0.00}{##1}}}
\@namedef{PY@tok@fm}{\def\PY@tc##1{\textcolor[rgb]{0.00,0.00,1.00}{##1}}}
\@namedef{PY@tok@vc}{\def\PY@tc##1{\textcolor[rgb]{0.10,0.09,0.49}{##1}}}
\@namedef{PY@tok@vg}{\def\PY@tc##1{\textcolor[rgb]{0.10,0.09,0.49}{##1}}}
\@namedef{PY@tok@vi}{\def\PY@tc##1{\textcolor[rgb]{0.10,0.09,0.49}{##1}}}
\@namedef{PY@tok@vm}{\def\PY@tc##1{\textcolor[rgb]{0.10,0.09,0.49}{##1}}}
\@namedef{PY@tok@sa}{\def\PY@tc##1{\textcolor[rgb]{0.73,0.13,0.13}{##1}}}
\@namedef{PY@tok@sb}{\def\PY@tc##1{\textcolor[rgb]{0.73,0.13,0.13}{##1}}}
\@namedef{PY@tok@sc}{\def\PY@tc##1{\textcolor[rgb]{0.73,0.13,0.13}{##1}}}
\@namedef{PY@tok@dl}{\def\PY@tc##1{\textcolor[rgb]{0.73,0.13,0.13}{##1}}}
\@namedef{PY@tok@s2}{\def\PY@tc##1{\textcolor[rgb]{0.73,0.13,0.13}{##1}}}
\@namedef{PY@tok@sh}{\def\PY@tc##1{\textcolor[rgb]{0.73,0.13,0.13}{##1}}}
\@namedef{PY@tok@s1}{\def\PY@tc##1{\textcolor[rgb]{0.73,0.13,0.13}{##1}}}
\@namedef{PY@tok@mb}{\def\PY@tc##1{\textcolor[rgb]{0.40,0.40,0.40}{##1}}}
\@namedef{PY@tok@mf}{\def\PY@tc##1{\textcolor[rgb]{0.40,0.40,0.40}{##1}}}
\@namedef{PY@tok@mh}{\def\PY@tc##1{\textcolor[rgb]{0.40,0.40,0.40}{##1}}}
\@namedef{PY@tok@mi}{\def\PY@tc##1{\textcolor[rgb]{0.40,0.40,0.40}{##1}}}
\@namedef{PY@tok@il}{\def\PY@tc##1{\textcolor[rgb]{0.40,0.40,0.40}{##1}}}
\@namedef{PY@tok@mo}{\def\PY@tc##1{\textcolor[rgb]{0.40,0.40,0.40}{##1}}}
\@namedef{PY@tok@ch}{\let\PY@it=\textit\def\PY@tc##1{\textcolor[rgb]{0.24,0.48,0.48}{##1}}}
\@namedef{PY@tok@cm}{\let\PY@it=\textit\def\PY@tc##1{\textcolor[rgb]{0.24,0.48,0.48}{##1}}}
\@namedef{PY@tok@cpf}{\let\PY@it=\textit\def\PY@tc##1{\textcolor[rgb]{0.24,0.48,0.48}{##1}}}
\@namedef{PY@tok@c1}{\let\PY@it=\textit\def\PY@tc##1{\textcolor[rgb]{0.24,0.48,0.48}{##1}}}
\@namedef{PY@tok@cs}{\let\PY@it=\textit\def\PY@tc##1{\textcolor[rgb]{0.24,0.48,0.48}{##1}}}

\def\PYZbs{\char`\\}
\def\PYZus{\char`\_}
\def\PYZob{\char`\{}
\def\PYZcb{\char`\}}
\def\PYZca{\char`\^}
\def\PYZam{\char`\&}
\def\PYZlt{\char`\<}
\def\PYZgt{\char`\>}
\def\PYZsh{\char`\#}
\def\PYZpc{\char`\%}
\def\PYZdl{\char`\$}
\def\PYZhy{\char`\-}
\def\PYZsq{\char`\'}
\def\PYZdq{\char`\"}
\def\PYZti{\char`\~}
% for compatibility with earlier versions
\def\PYZat{@}
\def\PYZlb{[}
\def\PYZrb{]}
\makeatother


    % For linebreaks inside Verbatim environment from package fancyvrb.
    \makeatletter
        \newbox\Wrappedcontinuationbox
        \newbox\Wrappedvisiblespacebox
        \newcommand*\Wrappedvisiblespace {\textcolor{red}{\textvisiblespace}}
        \newcommand*\Wrappedcontinuationsymbol {\textcolor{red}{\llap{\tiny$\m@th\hookrightarrow$}}}
        \newcommand*\Wrappedcontinuationindent {3ex }
        \newcommand*\Wrappedafterbreak {\kern\Wrappedcontinuationindent\copy\Wrappedcontinuationbox}
        % Take advantage of the already applied Pygments mark-up to insert
        % potential linebreaks for TeX processing.
        %        {, <, #, %, $, ' and ": go to next line.
        %        _, }, ^, &, >, - and ~: stay at end of broken line.
        % Use of \textquotesingle for straight quote.
        \newcommand*\Wrappedbreaksatspecials {%
            \def\PYGZus{\discretionary{\char`\_}{\Wrappedafterbreak}{\char`\_}}%
            \def\PYGZob{\discretionary{}{\Wrappedafterbreak\char`\{}{\char`\{}}%
            \def\PYGZcb{\discretionary{\char`\}}{\Wrappedafterbreak}{\char`\}}}%
            \def\PYGZca{\discretionary{\char`\^}{\Wrappedafterbreak}{\char`\^}}%
            \def\PYGZam{\discretionary{\char`\&}{\Wrappedafterbreak}{\char`\&}}%
            \def\PYGZlt{\discretionary{}{\Wrappedafterbreak\char`\<}{\char`\<}}%
            \def\PYGZgt{\discretionary{\char`\>}{\Wrappedafterbreak}{\char`\>}}%
            \def\PYGZsh{\discretionary{}{\Wrappedafterbreak\char`\#}{\char`\#}}%
            \def\PYGZpc{\discretionary{}{\Wrappedafterbreak\char`\%}{\char`\%}}%
            \def\PYGZdl{\discretionary{}{\Wrappedafterbreak\char`\$}{\char`\$}}%
            \def\PYGZhy{\discretionary{\char`\-}{\Wrappedafterbreak}{\char`\-}}%
            \def\PYGZsq{\discretionary{}{\Wrappedafterbreak\textquotesingle}{\textquotesingle}}%
            \def\PYGZdq{\discretionary{}{\Wrappedafterbreak\char`\"}{\char`\"}}%
            \def\PYGZti{\discretionary{\char`\~}{\Wrappedafterbreak}{\char`\~}}%
        }
        % Some characters . , ; ? ! / are not pygmentized.
        % This macro makes them "active" and they will insert potential linebreaks
        \newcommand*\Wrappedbreaksatpunct {%
            \lccode`\~`\.\lowercase{\def~}{\discretionary{\hbox{\char`\.}}{\Wrappedafterbreak}{\hbox{\char`\.}}}%
            \lccode`\~`\,\lowercase{\def~}{\discretionary{\hbox{\char`\,}}{\Wrappedafterbreak}{\hbox{\char`\,}}}%
            \lccode`\~`\;\lowercase{\def~}{\discretionary{\hbox{\char`\;}}{\Wrappedafterbreak}{\hbox{\char`\;}}}%
            \lccode`\~`\:\lowercase{\def~}{\discretionary{\hbox{\char`\:}}{\Wrappedafterbreak}{\hbox{\char`\:}}}%
            \lccode`\~`\?\lowercase{\def~}{\discretionary{\hbox{\char`\?}}{\Wrappedafterbreak}{\hbox{\char`\?}}}%
            \lccode`\~`\!\lowercase{\def~}{\discretionary{\hbox{\char`\!}}{\Wrappedafterbreak}{\hbox{\char`\!}}}%
            \lccode`\~`\/\lowercase{\def~}{\discretionary{\hbox{\char`\/}}{\Wrappedafterbreak}{\hbox{\char`\/}}}%
            \catcode`\.\active
            \catcode`\,\active
            \catcode`\;\active
            \catcode`\:\active
            \catcode`\?\active
            \catcode`\!\active
            \catcode`\/\active
            \lccode`\~`\~
        }
    \makeatother

    \let\OriginalVerbatim=\Verbatim
    \makeatletter
    \renewcommand{\Verbatim}[1][1]{%
        %\parskip\z@skip
        \sbox\Wrappedcontinuationbox {\Wrappedcontinuationsymbol}%
        \sbox\Wrappedvisiblespacebox {\FV@SetupFont\Wrappedvisiblespace}%
        \def\FancyVerbFormatLine ##1{\hsize\linewidth
            \vtop{\raggedright\hyphenpenalty\z@\exhyphenpenalty\z@
                \doublehyphendemerits\z@\finalhyphendemerits\z@
                \strut ##1\strut}%
        }%
        % If the linebreak is at a space, the latter will be displayed as visible
        % space at end of first line, and a continuation symbol starts next line.
        % Stretch/shrink are however usually zero for typewriter font.
        \def\FV@Space {%
            \nobreak\hskip\z@ plus\fontdimen3\font minus\fontdimen4\font
            \discretionary{\copy\Wrappedvisiblespacebox}{\Wrappedafterbreak}
            {\kern\fontdimen2\font}%
        }%

        % Allow breaks at special characters using \PYG... macros.
        \Wrappedbreaksatspecials
        % Breaks at punctuation characters . , ; ? ! and / need catcode=\active
        \OriginalVerbatim[#1,codes*=\Wrappedbreaksatpunct]%
    }
    \makeatother

    % Exact colors from NB
    \definecolor{incolor}{HTML}{303F9F}
    \definecolor{outcolor}{HTML}{D84315}
    \definecolor{cellborder}{HTML}{CFCFCF}
    \definecolor{cellbackground}{HTML}{F7F7F7}

    % prompt
    \makeatletter
    \newcommand{\boxspacing}{\kern\kvtcb@left@rule\kern\kvtcb@boxsep}
    \makeatother
    \newcommand{\prompt}[4]{
        {\ttfamily\llap{{\color{#2}[#3]:\hspace{3pt}#4}}\vspace{-\baselineskip}}
    }
    

    
    % Prevent overflowing lines due to hard-to-break entities
    \sloppy
    % Setup hyperref package
    \hypersetup{
      breaklinks=true,  % so long urls are correctly broken across lines
      colorlinks=true,
      urlcolor=urlcolor,
      linkcolor=linkcolor,
      citecolor=citecolor,
      }
    % Slightly bigger margins than the latex defaults
    
    \geometry{verbose,tmargin=1in,bmargin=1in,lmargin=1in,rmargin=1in}
    
    

\begin{document}
    
    \maketitle
    
    

    
    \begin{tcolorbox}[breakable, size=fbox, boxrule=1pt, pad at break*=1mm,colback=cellbackground, colframe=cellborder]
\prompt{In}{incolor}{1}{\boxspacing}
\begin{Verbatim}[commandchars=\\\{\}]
\PY{c}{\PYZsh{}\PYZsh{}\PYZsh{} A Pluto.jl notebook \PYZsh{}\PYZsh{}\PYZsh{}}
\PY{c}{\PYZsh{} v0.20.3}
\PY{k}{using}\PY{+w}{ }\PY{n}{Markdown}
\PY{k}{using}\PY{+w}{ }\PY{n}{InteractiveUtils}
\end{Verbatim}
\end{tcolorbox}

    \section{Meromorphic Functions and the
Logarithm}\label{meromorphic-functions-and-the-logarithm}

    \begin{tcolorbox}[breakable, size=fbox, boxrule=1pt, pad at break*=1mm,colback=cellbackground, colframe=cellborder]
\prompt{In}{incolor}{ }{\boxspacing}
\begin{Verbatim}[commandchars=\\\{\}]
\PY{c}{\PYZsh{} ╠═╡ show\PYZus{}logs = false}
\PY{k}{begin}
\PY{+w}{	}
\PY{+w}{	}\PY{k}{import}\PY{+w}{ }\PY{n}{Pkg}
\PY{+w}{	}\PY{n}{Pkg}\PY{o}{.}\PY{n}{activate}\PY{p}{(}\PY{p}{)}
\PY{+w}{    }\PY{k}{import}\PY{+w}{ }\PY{n}{GLMakie}
\PY{+w}{	}\PY{k}{import}\PY{+w}{ }\PY{n}{Images}
\PY{+w}{	}\PY{k}{import}\PY{+w}{ }\PY{n}{PlutoUI}
\PY{+w}{	}\PY{k}{using}\PY{+w}{ }\PY{n}{LaTeXStrings}
\PY{+w}{	}
\PY{k}{end}
\end{Verbatim}
\end{tcolorbox}

    \subsection{Zeros and Poles}\label{zeros-and-poles}

Meromorphic functions are determined by their zeros and poles.

\begin{enumerate}
\def\labelenumi{\arabic{enumi})}
\item
  Removeable singularities
\item
  Poles
\item
  Essential singularities
\end{enumerate}

A point singularity of a function is a complex number \(z_{0}\) such
that \(f\) is defined in a neighborhood of \(z_{0}\) but not at the
point \(z_{0}\) itself is called \textbf{Isolated singularity}

\textbf{note}: consider \(\Omega\) a connected open set

    \subsubsection{order of n}\label{order-of-n}

Suppose \(f\) is holomorphic on \(\Omega\), has a zero point at
\(z_{0} \in \Omega\) and does not vanished identically in \(\Omega\),
then there exists a neighborhood \(U \subset \Omega\) of \(z_{0}\), a
non-vanishing holomorphic function \(g\) on \(U\), and a unique positive
integer \(n\) such that

\[f(z)=(z-z_{0})^{n}g(z)\quad\forall z\in U\]

we say \(f\) has a \textbf{zero of order n} (or multiplicity) at
\(z_{0}\), if zero is order of \(1\) then it is called \textbf{simple}

\paragraph{Principal parts and
Residue}\label{principal-parts-and-residue}

let's define a deleted neighborhood of \(z_{0}\) to be an open disk
centered at \(z_{0}\) , minus the point \(z_{0}\), that is, the set:

\[\left\{\,z: 0< \left|z-z_{0}\right| < r \right\}\,\,\text{for some r}\quad \exists r > 0\]

then the function \(f\) defined in a deleted neighborhood of \(z_{0}\)
has a \textbf{pole} \(z_{0}\), if the function \(\dfrac{1}{f}\) defined
to be \(0\) at \(z_{0}\), is holomorphic in a full neighborhood of
\(z_{0}\)

\[\large\implies \dfrac{1}{f(z)}\,=(z-z_{0})^{n}g(z)\]

If \(f\) has a pole at \(z_{0} \in \Omega\), then in a neighborhood of
that point there exists a non-vanishing holomorphic function \(h\) and a
unique \textbf{positive integer} such that:

\(\large f(z)=(z-z_{0})^{-n}h(z),\) from previous info

\(\large \implies h(z)=\dfrac{1}{g(z)}\)

The integer \(n\) is called the order (or multiplicity) of the pole and
describes the rate at which the function grows near \(z_{0}\). if the
pole is the order of \(1\) then it is called \textbf{simple}

If \(f\) has a pole of order \(n\) at \(z_{0}\) then:

\[\large f(z)=\underline{\frac{a_{-n}}{(z-z_{0})^{n}}+\frac{a_{n-1}}{(z-z_{0})^{n-1}}+\cdots + \frac{a_{-1}}{(z-z_{0})}} + G(z)\]

The \textbf{sum} (the underlined part of the function) is called the
\textbf{principal part} of \(f\) at the pole \(z_{0}\) and the
coefficient \(a_{-1}\) is the \textbf{residue} of \(f\) at that pole
\(\implies res_{z_{0}}f\,=a_{-1}\)

If \(P(z)\) is denotes the principal part above and \(\mathcal{C}\) is
any circle centered at \(z_{0}\)

\[\implies \large \dfrac{1}{2\pi i}\oint_{\mathcal{C}}{P(z)dz}\,=\,a_{-1}\]

when \(f\) has a simple pole at \(z_{0}\)
\(\large \implies res_{z_{0}}=\lim_{z\to z_{0}}{(z-z_{0})}f(z)\)

\subsubsection{Conclusion}\label{conclusion}

If \(f\) has a pole of order \(n\) at \(z_{0}\) then

\[\Large res_{z_{0}}f\,=\,\lim_{z\to z_{0}} \dfrac{1}{(n-1)!}(\frac{d}{dz})^{n-1}(z-z_{0})^{n}f(z)\]

    \subsection{Residue Formula}\label{residue-formula}

If \(f\) is holomorphic in an open set containing a toy contour
\(\gamma\) and its interior points except for poles at the points
\(z_{1},\cdots\,z_{n}\) inside \(\gamma\) then:

\[\Large \oint_{\gamma}f(z)dz\,=2\pi i\sum_{k=1}^{N}res_{z_{k}}f\]

Example:

\(f(z)=\dfrac{e^{z}}{1+e^{z}}\) has a pole in \(z=i\pi\)

\(\implies (z-i\pi)f(z)=e^{z}\dfrac{z-i\pi}{e^{z}-e^{i\pi}}\)

\(\implies res_{i\pi}f=-e^{i\pi}\)

    \begin{tcolorbox}[breakable, size=fbox, boxrule=1pt, pad at break*=1mm,colback=cellbackground, colframe=cellborder]
\prompt{In}{incolor}{3}{\boxspacing}
\begin{Verbatim}[commandchars=\\\{\}]
\PY{k}{begin}
\PY{c}{\PYZsh{}proof that iπ is a pole}
\PY{n}{e}\PY{+w}{ }\PY{o}{=}\PY{+w}{ }\PY{n}{exp}\PY{p}{(}\PY{l+m+mi}{1}\PY{p}{)}
\PY{k}{function}\PY{+w}{ }\PY{n}{fz}\PY{p}{(}\PY{n}{z}\PY{o}{::}\PY{k+kt}{Complex}\PY{p}{)}
\PY{+w}{	}\PY{k}{return}\PY{+w}{ }\PY{n}{e}\PY{o}{\PYZca{}}\PY{p}{(}\PY{n}{real}\PY{o}{.}\PY{p}{(}\PY{n}{z}\PY{p}{)}\PY{p}{)}\PY{+w}{ }\PY{o}{/}\PY{+w}{ }\PY{n}{real}\PY{o}{.}\PY{p}{(}\PY{l+m+mi}{1}\PY{o}{+}\PY{n}{e}\PY{o}{\PYZca{}}\PY{p}{(}\PY{n}{z}\PY{p}{)}\PY{p}{)}
\PY{k}{end}
\PY{n}{printstyled}\PY{p}{(}\PY{l+s}{\PYZdq{}}\PY{l+s+si}{\PYZdl{}}\PY{p}{(}\PY{+w}{ }\PY{n}{real}\PY{o}{.}\PY{p}{(}\PY{n}{fz}\PY{p}{(}\PY{n+nb}{im}\PY{o}{*}\PY{n+nb}{pi}\PY{p}{)}\PY{p}{)}\PY{p}{)}\PY{l+s}{\PYZdq{}}\PY{p}{)}\PY{+w}{ }
\PY{k}{end}
\end{Verbatim}
\end{tcolorbox}

    \begin{Verbatim}[commandchars=\\\{\}]
Inf
    \end{Verbatim}

    \subsection{Riemann's Theorem on Removable
Singularities}\label{riemanns-theorem-on-removable-singularities}

If \(f\) is a function that is holomorphic on an open set \(\Omega\)
except possibly at one point \$z\_\{0\} \in \(\Omega\) if \(f\) is
bounded on \(\Omega-\left\{z_{0}\right\}\) then \(z_{0}\) is a
\textbf{removable} singularity

Suppose that \(f\) has an isolated singularity at the point \(z_{0}\).

Then \(z_{0}\) is a pole of \(f\)
\(\iff \lim_{z\to z_{0}}\left|f(z)\right| = \infty\)

Isolated singularities belong to one of three categories:

\begin{itemize}
\tightlist
\item
  Removable singularities (\(f\) is bounded near \(z_{0}\))
\item
  Pole singularities (\(\lim_{z\to z_{0}}\left|f(z)\right| = \infty\))
\item
  essential singularities: any singularity that is not removable or a
  pole.(\(f(z)=e^{\frac{1}{z}}\))
\end{itemize}

\textbf{Unlike Removable singularities and Pole singularities, it is
typical for a holomorphic function to behave erratically near an
essential singularity}

    \subsection{Casorati-Weierstrass
Theorem}\label{casorati-weierstrass-theorem}

Suppose \(f\) is holomorphic in the punctured disk
\(D_{r}(z_{0})-\left\{z_{0}\right\}\) and has an essential singularity
at \(z_{0}\). Then the image of \(D_{r}(z_{0})-\left\{z_{0}\right\}\)
under \(f\) is dense in the complex plane.

\[g(z)=\dfrac{1}{f(z)-w},\quad \left|g(z)\right| \le \frac{1}{\delta}\,,\,\delta >0\]

A function \(f\) on an \(\Omega\) is \textbf{meromorphic} if there
exists a sequence of points \(\left\{z_{0},z_{1},z_{2}\cdots\right\}\)
that has no limit points in \(\Omega\),and such that:

The function \(f\) is holomorphic in
\(\Omega - \left\{z_{0},z_{1},z_{2}\cdots\right\}\) and \(f\) has poles
at the points \(\left\{z_{0},z_{1},z_{2}\cdots\right\}\)

\textbf{A meromorphic function in the complex plane that is either
holomorphic at \(\infty\) or has a pole at \(\infty\) is said to be
meromorphic in the extended complex plane}

example: \(F(z)=f(\dfrac{1}{z})\)

    \subsection{Mittag-Leffler's Theorem}\label{mittag-lefflers-theorem}

The meromorphic functions in the extended complex plane are the rational
functions.

\[z_{k}\in\mathbb{C},\,\,f(z)=f_{k}(z)+g_{k}(z)\]

where \(f(z_{k})\) is the principal part of \(f\) at \(z_{k}\) and
\(g_{k}\) is holomorphic in a neighborhood of \(z_{k}\), in particular,
\(f_{k}\) is a polynominal in \(\dfrac{1}{z-z_{k}}\)

\[f(1/z)=\tilde{f_{\infty}}(z) + \tilde{g}_{\infty}(z)\] where
\(\tilde{g}_{\infty}\) is holomorphic in a neighborhood of the origin
and \(\tilde{f_{\infty}}\) is the principal part of \(f(1/z)\) at \(0\),
that is a polynominal in \(\dfrac{1}{z}\)

Let \(f_{\infty}(z)=\tilde{f}_{\infty}(1/z)\). We contend that function
\(H\).

\[H=f-f_{\infty}-\sum_{k=1}^{n}f_{k}\] is entire and bounded

    \subsection{\texorpdfstring{Riemann Sphere (one-point compactification
of
\(\mathbb{C}\))}{Riemann Sphere (one-point compactification of \textbackslash mathbb\{C\})}}\label{riemann-sphere-one-point-compactification-of-mathbbc}

Consider Euclidian space \(\mathbb{R}^{3}\) with coordinates \((X,Y,Z)\)
where \(XY-Plane\) is identified with \(\mathbb{C}\), We denote by
\(\mathbb{S}\) the sphere centered at \((0,0,1/2)\) and of
\textbf{radius}\(\,1/2\).

This sphere is of unit diameter and lies on top of the origin of the
\textbf{complex plain}. Also, let \(\mathcal{N}=(0,0,1)\) denote the
north pole of the sphere.

Given any point \(W=(X,Y,Z)\) on \(\mathbb{S}\) different from the north
pole, the line joining \(\mathcal{N}\) and \(W\) intersects the
\(XY-Plane\) in a single point, which we denote by \(w=x+iy\) is called
\textbf{Stereographic Projection} of \(W\)

Given any point \(w\) in \(\mathbb{C}\). the line joining
\(\mathcal{N}\) and \(w=(x,y,0)\) intersects the sphere at
\(\mathcal{N}\) and another point, which we call \(W\)

Giving \(w\) in terms of \(W\)

\[x=\dfrac{X}{1-Z},\quad y=\dfrac{Y}{1-Z}\]

Giving \(W\) in terms of \(w\)

\[X=\dfrac{x}{x^{2}+y^{2}+1},\quad Y=\dfrac{y}{x^{2}+y^{2}+1},\quad Z=\dfrac{x^{2}+y^{2}}{x^{2}+y^{2}+1}\]

As the point \(w\) goes to \(\infty\) in \(\mathbb{C}\)
(\(|w|\to\infty\)) the corresponding \(w\) on \(\mathbb{S}\) comes
arbitarily close to \(\quad\mathcal{N}\quad\)(\(\mathcal{N}\) points at
\(\infty\))

A meromorphic function on the \textbf{extended complex plane} can be
thought of as a map from \(\mathbb{S}\) to itself, where the image of a
pole is now a tractable point on \(\mathbb{C}\), namely \(\mathcal{N}\)

    \begin{tcolorbox}[breakable, size=fbox, boxrule=1pt, pad at break*=1mm,colback=cellbackground, colframe=cellborder]
\prompt{In}{incolor}{4}{\boxspacing}
\begin{Verbatim}[commandchars=\\\{\}]
\PY{k}{begin}
\PY{+w}{    }\PY{n}{fig}\PY{+w}{ }\PY{o}{=}\PY{+w}{ }\PY{n}{GLMakie}\PY{o}{.}\PY{n}{Figure}\PY{p}{(}\PY{n}{size}\PY{+w}{ }\PY{o}{=}\PY{+w}{ }\PY{p}{(}\PY{l+m+mi}{800}\PY{p}{,}\PY{+w}{ }\PY{l+m+mi}{600}\PY{p}{)}\PY{p}{)}
\PY{+w}{	}\PY{n}{ax}\PY{+w}{ }\PY{o}{=}\PY{+w}{  }\PY{n}{GLMakie}\PY{o}{.}\PY{n}{Axis3}\PY{p}{(}\PY{n}{fig}\PY{p}{[}\PY{l+m+mi}{1}\PY{p}{,}\PY{+w}{ }\PY{l+m+mi}{1}\PY{p}{]}\PY{p}{,}\PY{+w}{ }\PY{n}{title}\PY{+w}{ }\PY{o}{=}\PY{+w}{ }\PY{l+s}{\PYZdq{}}\PY{l+s}{Riemann Sphere}\PY{l+s}{\PYZdq{}}\PY{p}{,}\PY{n}{xticksvisible}\PY{o}{=}\PY{n+nb}{false}\PY{p}{,}\PY{n}{yticksvisible}\PY{o}{=}\PY{n+nb}{false}\PY{p}{,}\PY{n}{zticksvisible}\PY{o}{=}\PY{n+nb}{false}\PY{p}{,}\PY{n}{xticklabelsvisible}\PY{o}{=}\PY{n+nb}{false}\PY{p}{,}\PY{n}{yticklabelsvisible}\PY{o}{=}\PY{n+nb}{false}\PY{p}{,}\PY{n}{zticklabelsvisible}\PY{o}{=}\PY{n+nb}{false}\PY{p}{)}
\PY{+w}{    }\PY{c}{\PYZsh{} Create the sphere}
\PY{+w}{    }\PY{n}{θ}\PY{+w}{ }\PY{o}{=}\PY{+w}{ }\PY{n}{range}\PY{p}{(}\PY{l+m+mi}{0}\PY{p}{,}\PY{+w}{ }\PY{l+m+mi}{2}\PY{n+nb}{π}\PY{p}{,}\PY{+w}{ }\PY{n}{length}\PY{o}{=}\PY{l+m+mi}{50}\PY{p}{)}
\PY{+w}{    }\PY{n}{φ}\PY{+w}{ }\PY{o}{=}\PY{+w}{ }\PY{n}{range}\PY{p}{(}\PY{l+m+mi}{0}\PY{p}{,}\PY{+w}{ }\PY{n+nb}{π}\PY{p}{,}\PY{+w}{ }\PY{n}{length}\PY{o}{=}\PY{l+m+mi}{50}\PY{p}{)}
\PY{+w}{    }\PY{n}{x}\PY{+w}{ }\PY{o}{=}\PY{+w}{ }\PY{p}{[}\PY{n}{sin}\PY{p}{(}\PY{n}{ϕ}\PY{p}{)}\PY{+w}{ }\PY{o}{*}\PY{+w}{ }\PY{n}{cos}\PY{p}{(}\PY{n}{θ}\PY{p}{)}\PY{+w}{ }\PY{k}{for}\PY{+w}{ }\PY{n}{ϕ}\PY{+w}{ }\PY{k}{in}\PY{+w}{ }\PY{n}{φ}\PY{p}{,}\PY{+w}{ }\PY{n}{θ}\PY{+w}{ }\PY{k}{in}\PY{+w}{ }\PY{n}{θ}\PY{p}{]}
\PY{+w}{    }\PY{n}{y}\PY{+w}{ }\PY{o}{=}\PY{+w}{ }\PY{p}{[}\PY{n}{sin}\PY{p}{(}\PY{n}{ϕ}\PY{p}{)}\PY{+w}{ }\PY{o}{*}\PY{+w}{ }\PY{n}{sin}\PY{p}{(}\PY{n}{θ}\PY{p}{)}\PY{+w}{ }\PY{k}{for}\PY{+w}{ }\PY{n}{ϕ}\PY{+w}{ }\PY{k}{in}\PY{+w}{ }\PY{n}{φ}\PY{p}{,}\PY{+w}{ }\PY{n}{θ}\PY{+w}{ }\PY{k}{in}\PY{+w}{ }\PY{n}{θ}\PY{p}{]}
\PY{+w}{    }\PY{n}{z}\PY{+w}{ }\PY{o}{=}\PY{+w}{ }\PY{p}{[}\PY{n}{cos}\PY{p}{(}\PY{n}{ϕ}\PY{p}{)}\PY{+w}{ }\PY{k}{for}\PY{+w}{ }\PY{n}{ϕ}\PY{+w}{ }\PY{k}{in}\PY{+w}{ }\PY{n}{φ}\PY{p}{,}\PY{+w}{ }\PY{n}{θ}\PY{+w}{ }\PY{k}{in}\PY{+w}{ }\PY{n}{θ}\PY{p}{]}
\PY{+w}{   	}\PY{n}{GLMakie}\PY{o}{.}\PY{n}{surface!}\PY{p}{(}\PY{n}{ax}\PY{p}{,}\PY{+w}{ }\PY{n}{x}\PY{p}{,}\PY{+w}{ }\PY{n}{y}\PY{p}{,}\PY{+w}{ }\PY{n}{z}\PY{p}{,}\PY{+w}{ }\PY{n}{colormap}\PY{+w}{ }\PY{o}{=}\PY{+w}{ }\PY{l+s+ss}{:deep}\PY{p}{,}\PY{+w}{	 }\PY{n}{colorrange}\PY{+w}{ }\PY{o}{=}\PY{+w}{ }\PY{p}{(}\PY{l+m+mi}{80}\PY{p}{,}\PY{+w}{ }\PY{l+m+mi}{190}\PY{p}{)}\PY{p}{,}\PY{+w}{			}\PY{n}{shading}\PY{o}{=}\PY{n}{GLMakie}\PY{o}{.}\PY{n}{MultiLightShading}\PY{p}{)}
\PY{+w}{	 }
\PY{+w}{	}\PY{n}{GLMakie}\PY{o}{.}\PY{n}{text!}\PY{p}{(}\PY{n}{fig}\PY{p}{[}\PY{l+m+mi}{1}\PY{p}{,}\PY{+w}{ }\PY{l+m+mi}{1}\PY{p}{]}\PY{p}{,}\PY{+w}{  }\PY{n}{GLMakie}\PY{o}{.}\PY{n}{Point3f0}\PY{p}{(}\PY{l+m+mi}{1}\PY{p}{,}\PY{+w}{ }\PY{l+m+mi}{0}\PY{p}{,}\PY{+w}{ }\PY{l+m+mf}{1.1}\PY{p}{)}\PY{p}{,}\PY{+w}{ }\PY{n}{text}\PY{+w}{ }\PY{o}{=}\PY{+w}{ }\PY{l+s+sa}{L}\PY{l+s}{\PYZdq{}\PYZdq{}\PYZdq{}}\PY{l+s}{\PYZbs{}}\PY{l+s}{mathcal\PYZob{}N\PYZcb{}}\PY{l+s}{\PYZdq{}\PYZdq{}\PYZdq{}}\PY{p}{,}\PY{+w}{ }\PY{n}{color}\PY{+w}{ }\PY{o}{=}\PY{+w}{ }\PY{l+s+ss}{:red}\PY{p}{,}\PY{n}{fontsize}\PY{o}{=}\PY{l+m+mi}{34}\PY{p}{)}
\PY{+w}{	}\PY{n}{GLMakie}\PY{o}{.}\PY{n}{text!}\PY{p}{(}\PY{n}{fig}\PY{p}{[}\PY{l+m+mi}{1}\PY{p}{,}\PY{+w}{ }\PY{l+m+mi}{1}\PY{p}{]}\PY{p}{,}\PY{+w}{  }\PY{n}{GLMakie}\PY{o}{.}\PY{n}{Point3f0}\PY{p}{(}\PY{o}{\PYZhy{}}\PY{l+m+mi}{1}\PY{p}{,}\PY{+w}{ }\PY{o}{\PYZhy{}}\PY{l+m+mi}{1}\PY{p}{,}\PY{+w}{ }\PY{o}{\PYZhy{}}\PY{l+m+mf}{1.1}\PY{p}{)}\PY{p}{,}\PY{+w}{ }\PY{n}{text}\PY{+w}{ }\PY{o}{=}\PY{+w}{ }\PY{l+s}{\PYZdq{}}\PY{l+s}{0}\PY{l+s}{\PYZdq{}}\PY{p}{,}\PY{+w}{ }\PY{n}{color}\PY{+w}{ }\PY{o}{=}\PY{+w}{ }\PY{l+s+ss}{:black}\PY{p}{)}
\PY{+w}{    }\PY{n}{fig}
\PY{k}{end}
\end{Verbatim}
\end{tcolorbox}
 
            
\prompt{Out}{outcolor}{4}{}
    
    \begin{center}
    \adjustimage{max size={0.9\linewidth}{0.9\paperheight}}{output_11_0.png}
    \end{center}
    { \hspace*{\fill} \\}
    

    \subsection{The Argument Principle and Its
Applications}\label{the-argument-principle-and-its-applications}

\subsubsection{Argument Principle}\label{argument-principle}

Suppose that \(f\) is in an open set containing a circle \(\mathcal{C}\)
and its interior. If \(f\) has no poles and never vanishes on
\(\mathcal{C}\) then:

\[\dfrac{1}{2\pi i}\oint_{\mathcal{C}}\dfrac{f^{'}(z)}{f(z)}dz\,= \text{(number of zeros of f) - (number of poles of f) inside }\mathcal{C}\]

\textbf{This holds true for toy contours.}

    \subsubsection{Rouche's Theorem}\label{rouches-theorem}

Suppose that \(f\) and \(g\) are holomorphic in an open set containing a
circle \(\mathcal{C}\) and its interior.

If \(|f(z)|>|g(z)|\,\,\, \forall z\in \mathcal{C}\)

Then \(f\) and \(f+g\) have the \textbf{same number of zeros} inside the
circle \(\mathcal{C}\)

    \subsubsection{Open Mapping Theorem}\label{open-mapping-theorem}

If \(f\) is holomorphic and non-constant in a region \(\Omega\), then
\(f\) is \textbf{open}.

    \subsubsection{Maximum Modules
Principle}\label{maximum-modules-principle}

Suppose that \(\Omega\) is a region with compact closure
\(\overline{\Omega}\), if \(f\) is holomorphic on \(\Omega\) and
continuous on \(\overline{\Omega}\), then:

\[\underset{z\in\Omega}{sup}|f(z)|\le\underset{z\in\overline{\Omega}-\Omega}{sup}|f(z)|\]

then\(\implies\) \(f\) attains its \textbf{maximum} in
\(\overline{\Omega}\)

    \subsection{Homotopy and Simply Connected
Domains}\label{homotopy-and-simply-connected-domains}

Two curves are \textbf{homotopic} if one curve can be deformed into the
other by a \textbf{continuous} transformation without ever leaving
\(\Omega\)

If \(f\) is holomorphic in \(\Omega\) then
\(\Large\oint_{\gamma_{0}}f(z)dz = \oint_{\gamma_{1}}f(z)dz\) when ever
two curves \(\gamma_{0},\gamma_{1}\) are homotopic in \(\Omega\)

A region \(\Omega\) in the \textbf{complex plane} is \textbf{simply
connected} if any two pair of curves in \(\Omega\) with the same
end-points are \textbf{homotopic}

Any \textbf{homotopic} function in a \textbf{simply connected} domain
has a \textbf{primitive}.

If \(f\) is \textbf{holomorphic} in the \textbf{simply connected} region
\(\Omega\) then

\[\oint_{\gamma}f(z)dz=0\quad \forall\quad \text{closed curve}\,\,\gamma \in \Omega\]

    \begin{tcolorbox}[breakable, size=fbox, boxrule=1pt, pad at break*=1mm,colback=cellbackground, colframe=cellborder]
\prompt{In}{incolor}{9}{\boxspacing}
\begin{Verbatim}[commandchars=\\\{\}]
\PY{n}{homotopy\PYZus{}url}\PY{+w}{ }\PY{o}{=}\PY{+w}{ }\PY{l+s}{\PYZdq{}}\PY{l+s}{https://upload.wikimedia.org/wikipedia/commons/7/7e/HomotopySmall.gif}\PY{l+s}{\PYZdq{}}
\end{Verbatim}
\end{tcolorbox}

            \begin{tcolorbox}[breakable, size=fbox, boxrule=.5pt, pad at break*=1mm, opacityfill=0]
\prompt{Out}{outcolor}{9}{\boxspacing}
\begin{Verbatim}[commandchars=\\\{\}]
"https://upload.wikimedia.org/wikipedia/commons/7/7e/HomotopySmall.gif"
\end{Verbatim}
\end{tcolorbox}
        
    \textbf{Example of homotopic curves}

    \begin{tcolorbox}[breakable, size=fbox, boxrule=1pt, pad at break*=1mm,colback=cellbackground, colframe=cellborder]
\prompt{In}{incolor}{8}{\boxspacing}
\begin{Verbatim}[commandchars=\\\{\}]
\PY{n}{PlutoUI}\PY{o}{.}\PY{n}{Resource}\PY{p}{(}\PY{n}{homotopy\PYZus{}url}\PY{p}{,}\PY{+w}{ }\PY{l+s+ss}{:width}\PY{+w}{ }\PY{o}{=\PYZgt{}}\PY{+w}{ }\PY{l+m+mi}{200}\PY{p}{,}\PY{+w}{ }\PY{l+s+ss}{:autoplay}\PY{+w}{ }\PY{o}{=\PYZgt{}}\PY{+w}{ }\PY{l+s}{\PYZdq{}}\PY{l+s}{\PYZdq{}}\PY{p}{,}\PY{+w}{ }\PY{l+s+ss}{:loop}\PY{+w}{ }\PY{o}{=\PYZgt{}}\PY{+w}{ }\PY{l+s}{\PYZdq{}}\PY{l+s}{\PYZdq{}}\PY{p}{)}
\end{Verbatim}
\end{tcolorbox}

            \begin{tcolorbox}[breakable, size=fbox, boxrule=.5pt, pad at break*=1mm, opacityfill=0]
\prompt{Out}{outcolor}{8}{\boxspacing}
\begin{Verbatim}[commandchars=\\\{\}]
PlutoUI.Resource("https://upload.wikimedia.org/wikipedia/commons/7/7e/HomotopySm
all.gif", MIME type image/gif, (:width => 200, :autoplay => "", :loop => ""))
\end{Verbatim}
\end{tcolorbox}
        
    \subsection{The Complex logartihm}\label{the-complex-logartihm}

\textbf{Branch}: A branch of the complex logarithm is a continuous
function that selects one of the possible values for each complex
number. To make the function single-valued.

If \(\Omega\) is a \textbf{simply connected} with \(1\in\Omega\) and
\(0\not\in\Omega\), then there exists a \textbf{branch of logarithm}
\(F(z)=\log_{\Omega}(z)\) so that

\begin{enumerate}
\def\labelenumi{\arabic{enumi})}
\tightlist
\item
  \[F\quad \text{is holomorphic in}\quad \Omega\]
\item
  \(\mathcal{e}^{F(z)}=z\,\,\forall\,\,z\in\Omega\)
\item
  \(F(r)=\log(r)\quad \text{whenever r is a Real Number and near 1}.\)
\end{enumerate}

If \(\Omega\) is \textbf{simply connected domain} with \(1\in\Omega\)
and \(0\not\in\Omega\), then
\(\Large z^{\alpha} = \mathcal{e}^{\alpha\log(z)}\,\,\forall z\in\Omega\)

\textbf{Every non-zero complex number \(\large w\) can be written as:
\(\Large w=\mathcal{e}^{z}\)}

If \(f\) is a \textbf{nowhere vanishing holomorphic} function in a
\textbf{simply connected} region \(\Omega\) then there exists a
\textbf{holomorphic} function \(g\) on \(\Omega\) such that

\[\Large f(z)=\mathcal{e}^{g(z)}\,\,\forall\,\,z\in\Omega\]

    \begin{tcolorbox}[breakable, size=fbox, boxrule=1pt, pad at break*=1mm,colback=cellbackground, colframe=cellborder]
\prompt{In}{incolor}{6}{\boxspacing}
\begin{Verbatim}[commandchars=\\\{\}]
\PY{n}{complexlog\PYZus{}url}\PY{o}{=}\PY{+w}{ }\PY{l+s}{\PYZdq{}}\PY{l+s}{https://upload.wikimedia.org/wikipedia/commons/thumb/a/ab/Riemann\PYZus{}surface\PYZus{}log.svg/220px\PYZhy{}Riemann\PYZus{}surface\PYZus{}log.svg.png}\PY{l+s}{\PYZdq{}}
\end{Verbatim}
\end{tcolorbox}

            \begin{tcolorbox}[breakable, size=fbox, boxrule=.5pt, pad at break*=1mm, opacityfill=0]
\prompt{Out}{outcolor}{6}{\boxspacing}
\begin{Verbatim}[commandchars=\\\{\}]
"https://upload.wikimedia.org/wikipedia/commons/thumb/a/ab/Riemann\_surface\_log.s
vg/220px-Riemann\_surface\_log.svg.png"
\end{Verbatim}
\end{tcolorbox}
        
    A plot of the multi-valued imaginary part of the complex logarithm
function shows the branches

    \begin{tcolorbox}[breakable, size=fbox, boxrule=1pt, pad at break*=1mm,colback=cellbackground, colframe=cellborder]
\prompt{In}{incolor}{7}{\boxspacing}
\begin{Verbatim}[commandchars=\\\{\}]
\PY{n}{PlutoUI}\PY{o}{.}\PY{n}{Resource}\PY{p}{(}\PY{n}{complexlog\PYZus{}url}\PY{p}{,}\PY{+w}{ }\PY{l+s+ss}{:width}\PY{+w}{ }\PY{o}{=\PYZgt{}}\PY{+w}{ }\PY{l+m+mi}{200}\PY{p}{,}\PY{+w}{ }\PY{l+s+ss}{:autoplay}\PY{+w}{ }\PY{o}{=\PYZgt{}}\PY{+w}{ }\PY{l+s}{\PYZdq{}}\PY{l+s}{\PYZdq{}}\PY{p}{,}\PY{+w}{ }\PY{l+s+ss}{:loop}\PY{+w}{ }\PY{o}{=\PYZgt{}}\PY{+w}{ }\PY{l+s}{\PYZdq{}}\PY{l+s}{\PYZdq{}}\PY{p}{)}
\end{Verbatim}
\end{tcolorbox}

            \begin{tcolorbox}[breakable, size=fbox, boxrule=.5pt, pad at break*=1mm, opacityfill=0]
\prompt{Out}{outcolor}{7}{\boxspacing}
\begin{Verbatim}[commandchars=\\\{\}]
PlutoUI.Resource("https://upload.wikimedia.org/wikipedia/commons/thumb/a/ab/Riem
ann\_surface\_log.svg/220px-Riemann\_surface\_log.svg.png", MIME type image/png,
(:width => 200, :autoplay => "", :loop => ""))
\end{Verbatim}
\end{tcolorbox}
        
    \section{Fourier Series and Harmonic
functions}\label{fourier-series-and-harmonic-functions}

The coefficients of the power series expansion of \(f\) are given by

\[a_{n}=\dfrac{1}{2\pi r^{n}}\oint_{0}^{2\pi}f(z_{0}+re^{i\theta})e^{-ni\theta} d\theta\,\,,\,\forall n\ge 0,\,\, \forall\, r\,\, 0<r<R\]

Moreover if \(n<0\) then the Integral is \textbf{zero}

If f is \textbf{holomorphic} in a disk \(D_{R}(z_{0})\) then

\[f(z_{0})=\dfrac{1}{2\pi}\oint_{0}^{2\pi}f(z_{0}+re^{i\theta})d\theta ,\,\, \forall\, r\,\, 0<r<R\]

If f is \textbf{holomorphic} in a disk \(D_{R}(z_{0})\), and
\(u=Re(f)\),then

\[u(z_{0})=\dfrac{1}{2\pi}\oint_{0}^{2\pi}u(z_{0}+re^{i\theta})d\theta ,\,\, \forall\, r\,\, 0<r<R\]

Every \textbf{harmonic function} in a disk is the \textbf{real part} of
a \textbf{holomorphic} function in that disk.


    % Add a bibliography block to the postdoc
    
    
    
\end{document}
